\section{Требования к протоколу.}

В этой главе описываются требования к протоколу работы в рамках ботнета.

\subsection{Глобальные требования}
\begin{enumerate}
\item{Формализуемость без неоднозначностей.}
\item{Должен использовать шифрование.}
\item{Должен использовать скрытые каналы передачи данных.}
\item{Должен позволять масштабируемость инфраструктуры поддержки сети.}
\item{Должен позволять профилирование закладки сервером}
\item{Должен иметь возможность отчета о времени в обе стороны}
\item{Должен иметь проверки на подмену сервера.}
\item{Должен иметь проверки на подмену клиента.}
\item{Поддержка индивидуального и массового командного режима}
\end{enumerate}

\subsubsection{Шифрование}

Шифрование должно использоваться как для трафика, так и для payload.

\paragraph{Обязательно\\}

\begin{itemize}
\item{Шифрование: Данные (в сеть/из сети) не должны передаваться в открытом виде.}
\item{Должен использовать индивидуальный ключ шифрования трафика для каждой закладки}
\item{Прием/передача зашифрованных данных должны идти через скрытые каналы передачи информации}
\end{itemize}

\subsubsection{Скрытые каналы передачи данных}

\begin{itemize}
\item{ Реализация работы на, как минимум, двух каналах скрытой передачи
информации\footnote{позволит резервировать один из каналов для
информирования о нештатных ситуациях}
}
\end{itemize}


\subsection{Требования и к закладке и серверу}

\begin{enumerate}
\item{Использование в качестве транспорта стандартных протоколов на стандартных портах}
\item{Шифрование обмена данных}
\item{Скрытие канала передачи информации}
\item{Поддержка режима инкубационного периода}
\item{Поддержка режима тестового периода}
\item{Поддержка уникальных идентификаторов для модулей системы}
\item{использование уникальных идентификаторов позволяющих индексировать не
менее чем двойной объём IPv6 сети}
\end{enumerate}

\subsubsection{Требования к серверу}

\begin{enumerate}
\item{Event driven модель обработки событий\footnote{как показывает практика - быстрее многопоточной модели}}
\item{Набор проверок по отношению к клиентам}
\item{Защита от DoS атак}
\item{Классификация клиентов по задачам, которые на них можно выполнять}
\item{Классификация клиентов по уровню доверия}
\item{Достаточное быстродействие
\footnote{как показывает практика монстры вроде apache нагрузки в 200 тысяч ботов
не выдерживают несмотря на то, что мощности сервера и ширина канала позволяют обработку существенно большего}
}
\item{Поддержка включения журнала событий для каждого клиента}
\item{Поддержка детализированного журнала событий для каждого подозрительного клиента}
\item{Отдельный алгоритм работы с пойманными ботами}
\end{enumerate}

\paragraph{Защита от DoS атак} в протокольном контексте подразумевает построение протокола таким образом,
 чтобы обнаружение псевдоботов (т.е. реверсеров и закладок не корректно работающих с протоколом - закладок
 под контролем реверсеров) могло происходить на как можно более ранней стадии обмена, что позволит
 минимизировать затраты сервера и уменьшит вероятность быстрого DoS. Также часть данных может кэшироваться
 на закладке для минимизации нагрузки на сервер и исключения существенного увеличения определенных типов
 трафика в интернет.

\subsubsection{Требования к закладке}

\paragraph{Прочее\\}
\begin{enumerate}
\item{Синхронизации времени запуска payload на закладке с точкой отчета.}
\item{Должен содержать алгоритм генерации уникальных идентификаторов покрывающих не менее двойного объёма IPv6 сети}
\end{enumerate}

