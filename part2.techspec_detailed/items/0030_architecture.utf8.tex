\section{Требования к архитектуре}
\label{architecture}

\paragraph{Обзор главы.\\}
Описываются требования к реализации архитектуры ботнета -
серверной и клиентской части (для не peer2peer схемы).
Часть пунктов отсюда естественно будут дублироваться в разделах требований к
закладкам и серверу. Тут только архитектурные требования. Частности дальше.\\

\subsection{Глобальные требования}
\label{architecture_global}

\begin{enumerate}
\item{модульность}
\item{event-driven модель}
\item{поддержжка распределения нагрузки управляющей части}
\item{вынос ресурсозатратных вычислений на клиента}
\item{поддержжка профилей(схем) работы}
\item{поддержка взаимных проверок (бот-сервер, сервер-бот)}
\item{хранение и передача данных протокола в кросплатформенном формате}
\end{enumerate}

\paragraph{Event driven модель}- как показывает практика она быстрее многопоточной модели и проще программируется.

\paragraph{поддержжка профилей(схем) работы} - в т.ч. по уровням доверия (ступени инкубационный, тестовый, рабочий периоды), самостоятельное профилирование сетевых возможностей закладки в среде функционирования (в т.ч. детект
смены сети мобильным пользователем), поддержка профилирования возможностей закладки сервером.

\paragraph{вынос ресурсозатратных вычислений на клиента} - когда это не противоречит безопасности.

\paragraph{хранение и передача данных протокола в кросплатформенном формате} - независимость от
litle/big endian, будет актуально при портировании - меньше придется переделывать.

\newpage
\subsection{Требования и к закладке и серверу}

\begin{enumerate}
\item{Использование в качестве транспорта стандартных протоколов на стандартных портах}
\item{Шифрование обмена данных}
\item{Скрытие канала передачи информации}
\item{Поддержка режима инкубационного периода}
\item{Поддержка режима тестового периода}
\item{Все что перечислено в глобальных требованиях}
\end{enumerate}

\newpage
\subsection{Индивидуальные требования}

\subsubsection{Требования к закладке}

\begin{enumerate}
\item{работа, как минимум, части закладки в ring0}
\item{модульность. Как минимум - деление dropper/loader/payload}
\item{шифрование/расшифрование частей тела rootkit в памяти (<<на лету>>), части - по ключу с сервера}
\item{ступенчатая инсталляция}
\item{in memory only payload}
\item{реализация методов обнаружения запуска <<под отладчиком>>}
\item{профилирование сетевых возможностей в среде функционирования}
\item{поддержка профилирования возможностей закладки сервером}
\end{enumerate}

\paragraph{профилирование сетевых возможностей в среде функционирования} подразумевает выявление
разрешенных протоколов для работы с интернет.Необходимо, например, для того, чтобы в зараженной сети где запрещен, допустим, ICMP трафик, наружу не могла быть запущена icmp DDoS атака.
\begin{enumerate}
\item{пассивное - по сниффингу типов протоколов используемых пользователем}
\item{активное - по результатам запуска тестов на соединения}
\item{по результатам запроса к серверу}
\end{enumerate}

\paragraph{поддержка профилирования возможностей сервером}
подразумевает поддержку выставления переменных описывающих тип использования данного экземпляра сервером с использованием фильтрации на этой основе исполнения комманд выдаваемых всему ботнету.


\subsubsection{Требования к серверу}

\begin{enumerate}
\item{Event driven модель}
\end{enumerate}

