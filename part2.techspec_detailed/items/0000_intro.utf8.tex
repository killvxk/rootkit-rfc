\begin{flushleft}
\ldots\\
Уходят волки в оптике прицела..\\
И все про все - твой выстрел на удачу..\\
\ldots\\
группа Би 2.\\
\end{flushleft}

\begin{flushright}
\ldots\\
Игра ума.. \\
Кончается расстрелом.\\
И здесь и там..\\
Все та же волчья стая.\\
\ldots\\
группа Би 2.\\
\end{flushright}


\section{Обзор статьи.}
\label{section_paper_overview}


\subsection{Цель}
%
%\sl - наклонный
%\bf - bold (полужирный)
%
{\sl\bf
 Данная статья ставит своей целью техническое описание требований к качественному
 rootkit - архитектуре, протоколам, реализации. В идеале - это документ класса RFC.
}

{\sl\bf
Теоретическое обоснование в первой части данного RFC. Тут только техническая детализация
}


\paragraph{Сделано:\\}
Декларируется минимально необходимый набор требований к
реализации качественного rootkit: наборы требований к протоколам,
алгоритмам, языкам реализации, наборы требований при реализации
клиент-серверной (не peer2peer) модели взаимодействия.

\paragraph{В планах:\\}
Аналогичное описание для peer2peer модели взаимодействия.\\
Также см. стр. \pageref{paper_2do_1}, \pageref{paper_2do_2} .


\subsection{Использование.}

Статья выложена в паблик с целью создания полноценного rootkit RFC.\\

Статья может быть полезна для специалистов IT, в особенности тех из них, чья работа связана с обеспечением
безопасности IT инфраструктуры, либо с ее тестированием.\\

Статья развивалась усилиями членов клуба, а также <<усилиями третьих лиц>>.

Email текущего maintainer'а документа в рамках NetHack club:\\
grey-olli@ya.ru,\\
PGP ключ доступен к поиску в интернет: gpg --search-keys grey\_olli (сервер ключей: hkp://keys.gnupg.net)\\

Подразумевается использование в качестве ознакомительного материала для:

\begin{enumerate}
\item{Заказчика практической реализации rootkit}
\item{программиста или группы программистов, реализующих закладку класса <<rootkit>>
для сетей общего пользования, в частности, интернет}
\item{Любых заинтересованных лиц - как обзорный материал по требованиям,
 которым должен удовлетворять качественный rootkit.}
\end{enumerate}


\paragraph{Глоссарий\\}

Глоссарий пока отсутствует. Статья подразумевает,что вы понимаете используемую терминологию.
Однако Вы можете прислать ваши замечания/предложения о том что требуется внести в глоссарий.

\subsubsection{Порядок чтения}
\label{reading_notice}

Предполагается, что Вы заинтересованы прочесть все. Также предполагается, что Вы прочли теоретическое
обоснование (первую часть этого RFC).

\subsubsection{Замечания по главам}

Для того чтобы не было неоднозначностей наборы требований
могут характеризоваться так:

\begin{itemize}
\item{ <<Обязательно>> (английский синоним "MUST")}
\item{ <<Желательно>>  (английский синоним "SHOULD")}
\item{ <<Возможно>>    (английский синоним "MAY")}
\end{itemize}

\paragraph{\ref{section_paper_workflow} на стр. \pageref{section_paper_workflow}}
может быть интерсна только тем, кто участвует в развитии документа, она полностью
посвещена редакторской правке и ведению версий документа, по смысловой нагрузке
статьи там практически ничего нет.

\paragraph{Copyright\\}

Текущий copyright - Олег К. Артемьев. См. также описание \copyright в первой части этого RFC.

\paragraph{Спасибо\\}

Хочется сказать спасибо:

\begin{itemize}
\item{всем представителям компьютерномого undeground'а вообще.}
\item{virii/coding сцене.}
\item{Техническим специалистам, предоставившим свои комментарии
 к букве и сути данной работы.}
\end{itemize}
