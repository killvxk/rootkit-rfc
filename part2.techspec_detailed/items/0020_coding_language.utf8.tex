\section{Требования к языкам программирования}

\subsubsection{Логика(макро-алгоритм)}

\begin{itemize}
\item{ Для облегчения дальнейшей модификации, оптимизации, исправления ошибок,
а также с целью возможности портирования логика работы сервера и клиента
должны быть реализованы на языке среднего/высокого уровня,
предпочтительно одном из диалектов C/CPP.
}
\item{ При написании когда необходимо иметь ввиду портируемость, как клиента,
так и сервера. То есть как минимум межфункциональное взаимодействие
должно быть реализовано с возможностью замены функций, блоков функций.
}
%\item{}
\end{itemize}


\subsubsection{Низкий уровень}

\paragraph{клиент\\}
Процедуры низкого уровня на клиенте могут быть написаны на языке
ассемблера. Возможна реализация, либо в виде ассемблерных вставок в
C-функции, либо в виде отдельных функций assembler'а с интерфейсом
работы с языками среднего уровня. В любом случае  Ассемблерные вставки в
исходный код должны быть прозрачными для замены.


\paragraph{сервер\\}
Во избежание проблем с портированием серверная часть должна быть
полностью реализована на языке среднего/высокого уровня.
Предпочтительно C/CPP.

\subsection{Требования к исходному коду}

В общем-то эти требования можно применить к любому программному проекту.
Их удовлетворение конечно не обязательно, но должно существенно
упростить и сократить разработку.

\begin{enumerate}
\item{возможность переноса части функционала в библиотеку}
\item{реализация одного функционала одним кодом}
\item{написание кода с учетом возможности будущего портирования на другие платформы}
\item{Минимизация объёма кода написанного на ассемблере по сравнению с C/CPP}
\end{enumerate}
