\section{Требования к продукту в целом.}

В этой главе резюмируется что должен уметь ботнет с точки зрения его владельца.

\subsection{Требования к закладке}
\begin{enumerate}
\item{обход всех применяемых local firewalls}
\item{невидимость всем используемыми антивирусными пакетам (реакции <<подозрительно>> тоже быть не должно)}
\item{маскировка целевой сетевой активности паразитным трафиком, в частности соединений с сервером}
\item{адаптация собственной активности под активность пользователя}
\item{отработка смены сети мобильным пользователем}
\item{поддержка индивидуального и массового командного режима}
\item{реализация методов обнаружения запуска <<под отладчиком>>}
\item{профилирование сетевых возможностей в среде функционирования}
\item{поддержка профилирования возможностей сервером}
\item{поддержка индивидуального и массового командного режима}
\end{enumerate}

\paragraph{отработка смены сети мобильным пользователем} - хранение профиля активности для данной сети. Например попав в ISP dynamic ip pool одни ограничения, а внутри корпоративной сети - другие. Возможна реализация запросов
к серверу по установке профиля для данной сети.

\paragraph{профилирование сетевых возможностей в среде функционирования} подразумевает выявление
разрешенных протоколов для работы с интернет.Необходимо, например, для того, чтобы в зараженной сети где запрещен, допустим, ICMP трафик, наружу не могла быть запущена команда icmp DDoS.
\begin{enumerate}
\item{пассивное - по сниффингу типов протоколов используемых пользователем}
\item{активное - по результатам запуска тестов на соединения}
\end{enumerate}

\paragraph{поддержка профилирования возможностей сервером}
подразумевает поддержку выставления переменных описывающих тип использования данного экземпляра сервером с использованием фильтрации на этой основе исполнения комманд выдаваемых всему ботнету.



\newpage
\subsection{Требования к payload}

Тут будут перечислены классические payload и связанные с ними требования.

\paragraph{DDoS}
\begin{enumerate}
\item{реализация ICMP DDoS}
\item{реализация DNS DDoS}
\item{реализация HTTP DDoS c поддержкой keepalive}
\end{enumerate}


\newpage
\subsection{Требования к серверу}
\begin{enumerate}
\item{Защита от DoS атак}
\item{Классификация клиентов по задачам, которые на них можно выполнять}
\item{Классификация клиентов по уровню доверия}
\item{Достаточное быстродействие}
\item{Поддержка включения журнала событий для каждого клиента}
\item{Поддержка детализированного журнала событий для каждого подозрительного клиента}
\item{Отдельный алгоритм работы с пойманными ботами}
\end{enumerate}

\paragraph{Защита от DoS атак} - ботнет контроллеры часто сами подвергаются DoS атакам, соответственно реализация доллжна учитывать это. По возможности - наиболее раннее обнаружение попыток DoS и реакция, например включение хуков
на блокировку ip сетей средствами firewall, увеличение таймаутов на работу с нормальными ботами.

\paragraph{Достаточное быстродействие} - как показывает практика монстры вроде apache нагрузки в 200 тысяч ботов
не выдерживают несмотря на то, что мощности сервера и ширина канала позволяют обработку существенно большего.
Требуется нечто специализированное а-ля nginx .
