\section{Редакторское.}

% текущая состояние правки: проверена вся статья.

\label{section_paper_workflow}

эту часть полностью посвещена редакторской правке и ведению версий документа, по смысловой нагрузке статьи тут практически ничего нет.

\subsection{Замечания, комментарии, пожелания участников.}
%\subsubsection{Не подлежит правке}
%\subsubsection{Поправим когда нибудь потом..}

\subsubsection{2fix / Планируется переделать}

\label{paper_2do_1}
\paragraph{Note:} Указать, когда актуально, о каких OS ведется разговор.

\subsection{Планируется рассмотреть/сделать}
\label{paper_2do_2}
peer2peer botnets\\
\paragraph{Реорганизация:\\}
Разнести по главам чтобы получился структурированный набор - архитектура, алгоритмы, языки программиррования, протоколы, требования к <<конечному продукту>>.\\

\subsection{Please help}

Если вдруг найдете тут "я", "по моему мнению", "на мой взгляд" и т.д. - пишите - такому в  RFC не место.\\

\subsection{Изменения с предыдущей версии}

Это первая версия детальной технической спецификации требований.
