% Это основной файл, в который включаются остальные
% чтобы было можно редактировать по частям.
% Шаблон создан: Апр 12 2002

\documentclass[a4paper,11pt]{article}
%%%% document preamble start
%\usepackage[utf-8]{inputenc}
%\usepackage[koi8-r]{inputenc}
\usepackage{ucs}
\usepackage[utf8x]{inputenc}
\usepackage[english,russian]{babel}
\usepackage[dvips]{graphicx}
\usepackage{longtable}

% Index generation package require:
% 1. '\usepackage{makeidx}'
% 2. '\makeindex' in preamble (between document style definition and '\begin{dosument}' )
% 3. '\printindex' at the place where index should appear
% 4. running latex $file.tex
% 5. running latex $file.idx
% 6. running latex $file.tex
%
\usepackage{makeidx}
\makeindex


%
% очень плохая вещь! подрезаем поля по умолчанию :(
%
\addtolength{\hoffset}{-0.5cm}
\addtolength{\textwidth}{1.5cm}
\addtolength{\topmargin}{-11pt}
\addtolength{\footskip}{-10pt}
%
% конец плохой вещи ;)
%


%
% Расстановка переносов в "сложных" словах
%
\input ehyphen.tex

%разрешить перенос двух букв (недопустимо в ангельском)
\righthyphenmin=2

\author{NetHack club teamwork}
\title{ RootKit RFC, part 2\\Описание технических требований к качественному rootkit, часть 2.\\Аналитическая статья.\\Version 1.0 .}
%%%% document preamble end
\begin{document}
\maketitle
\thispagestyle{empty}
\newpage
\tableofcontents
%% пока нет таблиц и картинок это только портит содержание документа. закоментировано.
%\listoffigures
%\listoftables
\newpage
\input items/0000_intro.utf8.tex
% закоментировано за ненадобностью
%\newpage
%\input items/0001_2do,comments.utf8.tex
\newpage
\input items/0020_coding_language.utf8.tex
\newpage
\input items/0030_architecture.utf8.tex
\newpage
\input items/0040_protocol.utf8.tex
%\newpage
%\input items/.utf8.tex
%\newpage
%\input items/.utf8.tex
%\newpage
%\input items/.utf8.tex
%\newpage
%\input items/2split.utf8.tex
\newpage
\input items/9100_bibliography.utf8.tex
%\newpage
%\section{Предметный указатель}
%\printindex
\end{document}

