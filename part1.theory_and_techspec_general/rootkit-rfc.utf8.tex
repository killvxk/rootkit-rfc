% Это основной файл, в который включаются остальные
% чтобы было можно редактировать по частям.
% Шаблон создан: Апр 12 2002

\documentclass[a4paper,11pt]{article}
%%%% document preamble start
%\usepackage[utf-8]{inputenc}
%\usepackage[koi8-r]{inputenc}
\usepackage{ucs}
\usepackage[utf8x]{inputenc}
\usepackage[english,russian]{babel}
\usepackage[dvips]{graphicx}
\usepackage{longtable}

% Index generation package require:
% 1. '\usepackage{makeidx}'
% 2. '\makeindex' in preamble (between document style definition and '\begin{dosument}' )
% 3. '\printindex' at the place where index should appear
% 4. running latex $file.tex
% 5. running latex $file.idx
% 6. running latex $file.tex
%
\usepackage{makeidx}
\makeindex


%
% очень плохая вещь! подрезаем поля по умолчанию :(
%
\addtolength{\hoffset}{-0.5cm}
\addtolength{\textwidth}{1.5cm}
\addtolength{\topmargin}{-11pt}
\addtolength{\footskip}{-10pt}
%
% конец плохой вещи ;)
%


%
% Расстановка переносов в "сложных" словах
%
\input ehyphen.tex

%разрешить перенос двух букв (недопустимо в ангельском)
\righthyphenmin=2

\author{Grey Olli}
\title{ RootKit RFC, part 1\\Рекомендации по созданию програмных закладок.\\Аналитическая статья.\\Version 1.1 .}
%%%% document preamble end
\begin{document}
\maketitle
\thispagestyle{empty}
\newpage
\tableofcontents
%% пока нет таблиц и картинок это только портит содержание документа. закоментировано.
%\listoffigures
%\listoftables
\newpage
\input items/0000_intro2rootkits.utf8.tex
% глава с комментариями и туду закоментирована - с тех пор как статья поддерживалась прошло слишком много времени - это не актуальное
%\newpage
%\input items/0001_2do,comments.utf8.tex
\newpage
\input items/0002_rootkits_in_short.utf8.tex
\newpage
\input items/0005_rootkit_detection.utf8.tex
\newpage
\input items/0010_rootkit_evasion_of_detection.utf8.tex
\newpage
\input items/0020_rk_against_honeypots.utf8.tex
\newpage
\input items/0022_rk_against_avers.utf8.tex
\newpage
\input items/0025_rk_reincarnation.utf8.tex
\newpage
\input items/0030_diffrent_comments.utf8.tex
\newpage
\input items/0040_botnet_architecture.utf8.tex
\newpage
\input items/0050_botnet_treat_model.utf8.tex
\newpage
\input items/0400_good_rootkit_musthave.utf8.tex
\newpage
\input items/0600_rootkit_payloads.utf8.tex
\newpage
\input items/0650_rootkits_applience_ranges.utf8.tex
\newpage
\input items/9000_glossary.utf8.tex
\newpage
\input items/9100_bibliography.utf8.tex
\newpage
\section{Предметный указатель}
\printindex
%пока не написан автогенератор диффов по главам закоментировано.
%\input items/9999_diff_from_previous_version_verbatim.utf8.tex
%\newpage
\end{document}
