\section{Diff с предыдущей версии.}
\label{diff_from_prev_ver}
Чтобы не читать все заново. Но пока без разбивки по главам. В 2do -
сделать скрипт для автогенерации этой главы. К сожалению, сюда пока
попадают несмысловые изменения в форматировании исходного текста latex.

Помимо того что тут - сделаны новые главы с описанием:\\
 планов по правке статьи,\\
 модели угроз на разных уровнях ботнета,\\
 противодействию антивирусному ПО,
 глава \ref{tech_spec} вынесена в отдельную часть rfc (пока оставлена, в дальнейшем
будет минимизирована или вовсе убрана из этой части rfc.

\begin{verbatim}
 \section{Обзор статьи.}
@@ -22,10 +22,32 @@

 \subsection{Использование.}

-На данный момент статья не распространяется иначе как лично с согласия
-автора и на условиях нераспространения, в узком кругу заинтересованных
-лиц и экспертов способных внести вклад в ее развитие. Фраза <<Do not
-distribute>> полностью  соответствует режиму работы с данным
+На данный момент статья распространяется исключительно в узком кругу заинтересованных
+лиц - специалистов IT, в особенности среди тех из них, чья работа связана с обеспечением
+безопасности IT инфраструктуры, либо с ее тестированием.\\
+
+Статья развивается усилиями членов клуба, а также <<усилиями третьих лиц>>.
+Особенности схемы развития и публикации таковы, что для получения очередной
+версии необходимо прислать комментарии к текущей версии. Вам на комментарии
+могут сразу отправить одну из последних версий, но, возможно, придется
+откомментировать все, начиная с самой первой. Политика клуба по отношению
+к данному документу не запрещает участие в развитии данного документа
+представителей антивирусных и прочих организаций (в т.ч. занимающихся
+защитой информации), однако если таковые участвуют в развитии документа -
+на них распространяются все требования к получателям текущей версии:
+\begin{itemize}
+\item{не распространять документ в public(включая передачу представителям других организаций)}
+\item{присылать комментарии к текущей версии для получения следующей}
+\item{использовать только зашифрованные каналы передачи информации в рамках работы с документом}
+\end{itemize}
+При этом, после индивидуальных консультаций, можно выработать легитимную (с точки зрения
+клуба) схему работы с этим документом, которая позволит использовать его в
+рамках организации не только ее представителем присылющим комментарии.\\
+
+Email текущего релизера документа в рамках NetHack club: grey-olli@ya.ru,\\ PGP ключ доступен к поиску в интернет: gpg --search-keys grey\_olli , сервер ключей: hkp://keys.gnupg.net \\
+Пожалуйста присылайте Ваши комментарии исключительно в зашифрованном pgp виде.\\
+
+Фраза <<Do not distribute>> полностью  соответствует режиму работы с данным
 материалом.\\

 Подразумевается использование в качестве ознакомительного материала для:
@@ -35,15 +57,15 @@
 \item{Управляющего персонала среднего и высшего звена (главы: \ref{section_paper_overview},
  \ref{section_rk_payloads} ) }

-\item{Персонала ответственного за безопасность сетевой
+\item{Персонала ответственного за безопасность сетевой
 ( internet(IPv4/IPv6), ЛВС (ethernet / wifi LANs / Corporate WANs) ) инфраструктуры.
 (особое внимание главам \ref{section_rk_detection} и \ref{section_rk_evasion}).
 }

 \item{Заказчика практической реализации описываемых в данной статье принципов - стоит
- просмотреть все уделив особое внимание главам
-\ref{section_rk_detection} (со стр. \pageref{section_rk_detection}),
-\ref{honeypot_section} (со стр. \pageref{honeypot_section}),
+ просмотреть все уделив особое внимание главам
+\ref{section_rk_detection} (со стр. \pageref{section_rk_detection}),
+\ref{honeypot_section} (со стр. \pageref{honeypot_section}),
 \ref{section_rk_payloads} (со стр. \pageref{section_rk_payloads}) .}

 \item{программиста или группы программистов, реализующих закладку класса <<rootkit>>
@@ -51,7 +73,7 @@

 \item{Любых заинтересованных лиц - как
  обзорный материал по возможным способам скрытия и обнаружения закладок,
-использующих сети общего доступа, в частности, интернет (если нет
+использующих сети общего доступа, в частности, интернет (если нет
 конкретной цели и достаточно времени, то лучше просмотреть все ).}

 \end{enumerate}
@@ -63,19 +85,19 @@
 мало знакомого с предметом, в том числе с интернет. Однако, в нем
 присутствуют и термины, которые могут быть новыми и для некоторых
 IT-специалистов, особенно <<прикладников>>. Так что пролистать глоссарий
-(\ref{glossary} стр. \pageref{glossary} - \pageref{glossary_end})
-рекомендую всем. Для удобства сделан предметный указатель страниц
+(\ref{glossary} стр. \pageref{glossary} - \pageref{glossary_end})
+рекомендуем всем. Для удобства сделан предметный указатель страниц
 по встречающимся терминам (в основном в глоссарии).

 \subsubsection{Порядок чтения}
-
+\label{reading_notice}
 В случае если Вы не знакомы с предметом, или не уверены, что знакомы
 достаточно хорошо, имеет смысл начать с просмотра глоссария.\\

 \paragraph{Ответственным за сетевую безопасность} желательно просмотреть все, уделив
 особенное внимание обзору в главах \ref{section_rk_detection} и \ref{section_rk_evasion}\\

-\paragraph{Техническим специалистам} рекомендуется уделить внимание главе \ref{tech_spec}
+\paragraph{Техническим специалистам} рекомендуется уделить внимание главе \ref{tech_spec}
 (стр. \pageref{tech_spec} - \pageref{tech_spec_end}), все остальное, если не возникает
 вопросов по аргументации, можно лишь бегло просмотреть.

@@ -83,12 +105,15 @@

 \paragraph{ Глава \ref{tech_spec} } (стр. \pageref{tech_spec} - \pageref{tech_spec_end})
 является техническим резюме рассмативаемых аспеектов написания и инсталляции
-закладок и предназначенна исключительно для специалистов, поэтому комментарии к
+закладок и предназначенна исключительно для специалистов, поэтому комментарии к
 используемым в ней терминам отсутствует в глоссарии.

-\paragraph{ Глава \ref{applience_limits}} (стр. \pageref{applience_limits}) дает
+\paragraph{ Глава \ref{applience_limits}} (стр. \pageref{applience_limits}) дает
 краткий обзор ограничений присущих rootkits.

+\paragraph{ Глава \ref{section_paper_workflow}} (стр. \pageref{section_paper_workflow})
+может быть интерсна только тем, кто участвует в развитии документа, она полностью посвещена редакторской правке и ведению версий документа, по смысловой нагрузке статьи в ней практически
+ничего нет..

 \subsection{Цель}
 %
@@ -101,11 +126,25 @@
 качественному rootkit, то есть документ класса RFC.
 }

+{\slДля разработки качественного ПО, требуются соответствующие подходы
+ - выработка требований, построение архитектуры, планирование, а не
+ кодинг на коленке сломя голову. Это все применимо и к руткитам, и
+даже в большей степени - из-за специфичности разработки - код работающий
+с ядром ОС (существенная часть кода rootkit) требует внимательости и продуманности.}\\
+
+Данная статья развивается чтобы заполнить нишу в части описания правильной архитектуры и
+выработки требований к реализации.
+
+
+\paragraph{Сделано:\\}
 Рассматриваются методы обнаружения (вкратце) и методы скрытия (в общих
 чертах).  Декларируется минимально необходимый набор требований к
 реализации качественного rootkit: наборы требований к протоколам,
 алгоритмам, языкам реализации, наборы требований при реализации
-клиент-серверной (не peer2peer) модели взаимодействия.
+клиент-серверной (не peer2peer) модели взаимодействия.
+
+\paragraph{В планах:\\}
+см. \pageref{paper_2do_1}, \pageref{paper_2do_2} .

 \paragraph{Что такое rootkit\\}

@@ -116,11 +155,18 @@
 сторонних производителей. Такой уровень невидимости достигается путём
 перехвата внутренних функций ОС на уровне ядра\footnote{самый низкий,
 или <<самый внутренний>> уровень работы ОС}, т.е. перехватом функций к
-которым обращаются как процессы прикладного режима, так и функции ОС).
+которым обращаются как процессы прикладного режима, так и функции ОС.
+Существуют различные подходы к классификации руткитов, они будут вкратце рассмотрены
+далее.\footnote{в частности, встречается понятие 'user-mode' rootkit. С учетом
+развития современных средств противодействия нежелательному ПО такие rootkit'ы,
+по большому счету, за rootkit'ы считать уже нельзя.}
+

 \paragraph{Copyright\\}

-Some person you shouldn't be aware of.
+Текущий \copyright - NetHack club .
+Первоначальный вариант статьи был написан в 2006 году by grey\_olli, в
+мае 2009 статья перешла в team work.

 \paragraph{Спасибо\\}

@@ -129,9 +175,6 @@
 \begin{itemize}
 \item{всем представителям компьютерномого undeground'а вообще.}
 \item{virii/coding сцене.}
-\item{Техническим специалистам, предоставившим мне свои комментарии
- к букве и сути данной работы.\footnote{данная статья распространялась
-среди огранниченного круга лиц на условиях нераспространения.}
-}
+\item{Техническим специалистам, предоставившим свои комментарии
+ к букве и сути данной работы.}
 \end{itemize}
-
--- items.ver_1.0/0005_rootkit_detection.utf8.tex	2009-05-26 03:29:39.000000000 +0400
+++ items/0005_rootkit_detection.utf8.tex	2009-06-06 17:18:48.000000000 +0400
@@ -5,8 +5,8 @@
 Под сетевым обнаружением понимается и активное и пассивное сетевое
 обнаружение.

-\subparagraph{Активное сетевое обнаружение закладок}\index{активное
-сетевое обнаружение} это либо сканирование
+\subparagraph{Активное сетевое обнаружение закладок}\index{активное
+сетевое обнаружение} это либо сканирование
 компьютеров подключённых к защищаемой сети на предмет несоответсвия
 открытых портов используемым на них сервисам, либо определение
 нестандартной реакции на проходящий мимо тестируемых компьютеров
@@ -15,28 +15,32 @@
 компьютеру, хотя и доступному на нем, например по  resolving'у таких
 адресов}.

-\subparagraph{Пассивное сетевое обнаружение закладок}
+\subparagraph{Пассивное сетевое обнаружение закладок}
 \index{пассивное сетевое обнаружение} подразумевает под
  собой анализ трафика в сети. Либо на прокси, маршрутизаторах и
-коммутаторах по статистическим аномалиям, либо на IDS (статистические
-аномалии плюс сигнатуры на трафик определённого типа). HoneyPots  также
-используют пассивное определение по трафику, по крайней мере, на
+коммутаторах по статистическим аномалиям и сигнатурам, либо на IDS
+(статистические аномалии плюс сигнатуры на трафик определённого типа).
+HoneyPots  также используют пассивное определение по трафику, по крайней мере, на
 первоначальном этапе.

 \paragraph{Обзор\\}
 \index{локальное обнаружение}
 Вариантов локального обнаружения по большому счету два. Первый -
-обнаружение загрузки кода после загрузки с внешнего к системе носителя,
-например CDROM, так можно провести, например, сравнение чтения реестров
-- <<offline>> и <<online>>. Второй - обнаружения несоответствий
-результатов вызова тех или иных функций API на уровне пользователя
-(ring3) и на уровне ядра (ring0).
+обнаружение инструкций для загрузки кода\footnote{например ключей реестра и т.п.}
+и прочих изменений после загрузки с внешнего к системе носителя, например CDROM,
+так можно провести, например, сравнение чтения реестров - <<offline>> и <<online>>.
+Второй - обнаружения несоответствий результатов вызова тех или иных функций API
+на уровне пользователя (ring3) и на уровне ядра (ring0). Возможен также вариант с
+опросом через различные службы в пределах ring3 на предмет неких данных о системе
+и сравнение полученного, однако это ненадёжно.

 \subsection{Локальное обнаружение rootkit}
 \index{локальное обнаружение}

 \subsubsection{Используемое готовое ПО}

+На данный момент рассматриваются только Windows ситемы. Желающие дополнить unix-like аналогами - присылайте краткие описания релизеру статьи.
+
 \paragraph{RKDetect\\}
 RKDetect - утилита обнаружения Windows rootkit по поведению.

@@ -46,17 +50,15 @@
 руткитами типа Hacker Defender. Утилита чрезвычайно проста - она
 перечисляет службы  удалённого компьютера через WMI (на пользовательском
 уровне) \footnote{Windows Management Instrumentation (WMI) - интерфейс,
- обеспечивающий взаимодействие с компонентами системы, в общем случае
-доступными лишь через особые механизмы. WMI можно использовать в
-различных целях, в частности для управления компьютерами с помощью
-сценариев.} и через Service Control Manager (на уровне ядра), сравнивает
-результаты и отображает различия. Таким образом находятся скрытые
-службы, используемые обычно для запуска rootkit. Такой же подход может
-быть использован для обнаружения процессов, разделов реестра и всего,
-что может скрыть rootkit.
+ обеспечивающий взаимодействие с компонентами системы, в общем случае
+доступными лишь через особые механизмы. WMI можно использовать в
+различных целях, в частности для управления компьютерами с помощью
+сценариев.} и через Service Control Manager (приложение работает на уровне пользователя - services.exe), сравнивает результаты и отображает различия. Таким образом находятся скрытые
+службы, используемые обычно для запуска rootkit. Такой же подход может быть использован
+для обнаружения процессов, разделов реестра и всего, что может скрыть rootkit.

 \paragraph{Антивирусный монитор Adinf\\}
-\index{Adinf}
+\index{Adinf}
 Утилита устанавливается в проверяемую систему сразу же после её
 инсталляции и создает контрольные суммы файлов имеющихся на диске
 (конфигурабельно). В дальнейшем при запусках  контрольные суммы
@@ -71,18 +73,34 @@
 (конфигурабельно). В дальнейшем при запусках  контрольные суммы
 сравниваются с оригинальными.

-\paragraph{Антивирусные мониторы AVZ, DrWEB, Kaspersky, NortonAntivirus, Panda antivirus и другие.\\}
+\paragraph{AVZ,GMER}
+\label{antirootkittool}
 \index{AVZ}
+\index{GMER}
+Наиболее серьезные из известных нам утилит специально нацеленных на обнаружение rootkit, активно развиваются\footnote{весна 2009го года.} авторами.
+
+\paragraph{Антивирусные мониторы DrWEB, Kaspersky, NortonAntivirus, Panda antivirus и другие.\\}
+\label{antiviruses}
 \index{DrWEB}
 \index{Kaspersky}
 \index{NortonAntivirus}
 \index{Panda}
-Всерьёз почти не рассматриваются, поскольку 99.(9) их функциональности
-состоит в обнаружении известных вирусов, троянцев, червей и прочего
-деструктивного ПО. Обнаружение новых закладок у всех современных
-антивирусов отнюдь не на лучшем уровне - максимум, что может сказать
-антивирусный продукт - <<подозрительно>>. О методах противодействия
-можно сказать <<в двух словах>> следующее:
+На 2006-2007й годы всерьёз почти не рассматривались как антируткит-средства,
+поскольку 99.(9) их функциональности состояло в обнаружении известных вирусов,
+троянцев, червей и прочего деструктивного ПО. Обнаружение новых закладок у всех
+антивирусов отнюдь не на лучшем уровне - максимум, что может сказать
+антивирусный продукт - <<подозрительно>>.\\
+
+Тем не менее, в 2009м антируткит-технологии различного качества предоставляет большинство
+антивирусов. Например, на конец 2007го - KAV детектит скрытые процессы как руткит, DrWeb
+стал детектить и лечить новый MBR Rootkit. Качество их работы требует отдельного рассмотрения.
+Поскольку цель данной главы лишь обзор средств борьбы с руткитами - антивирусные продукты
+упомянуты весьма поверхностно.\\
+
+Также отдельного рассмотрения заслуживает проактивная защита рекламируемая многими производителями
+антивирусов.\\
+
+О методах противодействия классическим сигнатурным анализаторам можно сказать <<в двух словах>> следующее:

 \begin{itemize}
 \item{использование алгоритмов полиморфизма}
@@ -90,22 +108,20 @@
 \end{itemize}

 \index{полиморфизм}
-Кстати, наиболее правильным является полиморфизм при компиляции, когда
+Кстати, наиболее правильным является полиморфизм реализуемый при компиляции, когда
 антивирусные компании не получают для последующего реверсинга алгоритм
 полиморфизма и, таким образом, не способны выпустить версию антивируса,
-который будет ловить следующий <<морф>> rootkit'а.
+который будет ловить следующий <<морф>> rootkit'а и его дроппера.

 \subsection{Сетевое обнаружение rootkit}

-Обнаружение любых закладок и вредоносного кода использующего возможности
-сети достаточно просто. При этом обнаружение можно разделить на два
-категории:\\
+Обнаружение любых закладок и вредоносного кода использующего возможности сети достаточно
+просто, если не используются скрытые каналы передачи данных с маскировкой паразитного
+трафика под легитимный. При этом сетевое обнаружение можно разделить на две
+категории:\\

 \begin{itemize}
-\item {обнаружение локальное(средствами  Personal Firewall, как то: Kerio,
-Agnitum, Outpost, интегрированные в antivirus-software  программные
-продукты предоставляющие personal firewall в качестве одной из опций
-(например  в комплекте у Norton, Panda, Kaspersky).}
+\item {обнаружение локальное - средствами  Personal Firewall, как то: Kerio, Outpost и др., а также интегрированными в antivirus-software продукты программными модулями, предоставляющими функциональность personal firewall в качестве одной из опций (например  в комплекте у Norton, Panda, Kaspersky), функционал аналогичный Personal Firewall имеется у HIPS (host intrusion prevention systems), например Cisco Security Agent\index{Cisco Security Agent}}
 \item{обнаружение на транзитном сетевом оборудовании.}
 \end{itemize}

@@ -114,17 +130,23 @@

 \subsubsection{Обнаружение на локальных firewall'ах}

-Доступно любому пользователю ПК, однако ненадёжно, поскольку использует
-перехват обращений к функциям реализующим передачу данных по сети, то
-есть, если rootkit осуществил перехват раньше personal firewall, то
-весь трафик rootkit'а идёт мимо firewall (не блокируется, не
-замечается) .
-
-Существуют закладки, в которых реализована "необнаружимость"
-персональными firewall'ами любых производителей (на данный момент).
-Однако доступа к их исходным кодам у меня нет. Тем не менее с учетом
-того, что на подобных закладках работает достаточно приличный по
-объёмам ботнет - это достаточно достоверные данные.
+Доступно любому пользователю ПК, однако ненадёжно, поскольку использует
+перехват обращений к функциям реализующим передачу данных по сети, то
+есть, если rootkit осуществил перехват раньше personal firewall, то
+весь трафик rootkit'а идёт мимо firewall (не блокируется, не
+замечается) . В идеале очередность загрузки значения не имеет - руткит
+за счет нестандартных приемов скрытия активности (в т.ч. сетевой) может
+оставить персональный firewall <<в дураках>>.
+
+Существуют закладки, в которых реализована "необнаружимость"
+персональными firewall'ами любых производителей.\footnote{Это было абсолютно
+справедливо в 2006м году - на подобных закладках работал достаточно приличный по
+объёмам ботнет, rootkit-часть которого тестировалась на большинстве personal firewalls
+того времени - достоверные данные <<из первых рук>>. В 2009м это утверждение можно
+ считать спорным (особенно с учетом появления комплексных систем с использованием
+ технологии гипервизоров), однако де факто борьба firewall vs malware в этом
+контексте - классическая вечная борьба щита и меча - после появления новых способов
+ взлома со временем появляются способы защиты от них.}

 \subsubsection{Обнаружение на транзитных сетевых устройствах}

@@ -133,20 +155,20 @@

 \begin{itemize}
 \item{обнаружение асимметрии трафика}
-\item{обнаружение аномальной активности (нетипичные объёмы, порты, протоколы )}
+\item{обнаружение аномальной активности (нетипичные объёмы трафика, порты, протоколы)}
 \item{обнаружение на основе сигнатур IDS}
 \end{itemize}


 \paragraph{Асимметрия трафика\\}
 \index{асимметрия трафика}
-Протоколы используемые для работы с сетью имеют статистические
-характеристики доступные на точках через которые проходит трафик.
-Например, http протоколу характерна нормальная асимметрия, когда
-пользователь получает данных больше, чем отправляет\footnote{фактически
-трафик пользователя это запросы, а ответный трафик (контент запрошенный
-сервера) обычно существенно больше по объёму данных}. Существуют
-анализаторы асимметрии трафика, способные сигнализировать нетипичную
+Протоколы используемые для работы с сетью имеют статистические
+характеристики доступные на точках через которые проходит трафик.
+Например, http протоколу характерна нормальная асимметрия, когда
+пользователь получает данных больше, чем отправляет\footnote{фактически
+трафик пользователя это запросы, а ответный трафик (контент запрошенный
+сервера) обычно существенно больше по объёму данных}. Существуют
+анализаторы асимметрии трафика, способные сигнализировать нетипичную
 статистику обмена данными по различным протоколам.

 \paragraph{маршрутизаторы\\}
@@ -161,43 +183,50 @@
 Такое представление данных весьма наглядно и позволяет узнать о
 установленных закладках по нетипичному трафику (нестандартные порты,
 протоколы не используемые в данной организации), либо по пикам трафика
-определённого типа. Обнаружение нестандартного поведения по графикам
-расходования трафика . Для пример можно указать на маршрутизаторы
-компании cisco systems и пакет netflow tools. Аналоги программных
-комплектов визуализаторов трафика есть для любых производителей
-маршрутизаторов. Даже в случае, если маршрутизатором является обычный ПК
-у не слишком ленивого администратора есть все необходимое чтобы добиться
-такой же наглядности  представления трафика, как и при использовании
-маршрутизаторов той же cisco. \\
+определённого типа, а также по графикам расходования трафика .
+Для примера можно указать на маршрутизаторы компании cisco systems и
+пакет netflow tools. Аналоги программных комплектов визуализаторов
+трафика есть для любых производителей маршрутизаторов. Даже в случае,
+если маршрутизатором является обычный ПК - у не слишком ленивого
+администратора есть все необходимое чтобы добиться такой же наглядности
+ представления трафика, как и при использовании маршрутизаторов той же
+ cisco. \\

 Обнаружение нетипичного поведения на основе трафика весьма опасно для
 владельцев rootkit и может послужить поводом для дальнейших
 разбирательств, что рано или поздно приведёт к обнаружению и, в, худшем
-для владельцев, rootkit'а случае, реверсингу алгоритма работы,
-инсталляции и передачи информации с возможным выходом на владельцев,
-вплоть до выявления личности со всеми вытекающими.
-
-
+для владельцев, rootkit'а случае, реверсингу алгоритмов работы,
+инсталляции и протоколов передачи информации . Это может привести
+к выходу на владельцев, вплоть до выявления личности с вытекающими
+из этого судебными преследованиями.

 \paragraph{прокси\\}
 \index{прокси}
 \index{proxy}
 Часть сетей (значительная часть сетей организаций) предоставляют доступ
-к web ресурсам (и, иногда, к другим ресурсам) через proxy. Прокси
-могут, как и маршрутизаторы, быть совмещены с программным комплексом по
-визуализации трафика, то есть возможности обнаружения в общих чертах те
-же, что и у маршрутизаторов. Кроме этого прокси обладают возможностью
-хранения трафика проксируемого протокола, избирательной блокировки
-ресурсов доступных в сети по проксируемому протоколу, а также позволяют
-реализовать запуск программы по факту доступа к тому или иному ресурсу
-доступному через проксируемый протокол. То есть потенциально более
-опасны для владельцев rootkit.
+к web ресурсам (и, иногда, к другим ресурсам) через proxy. Прокси
+могут, как и маршрутизаторы, быть совмещены с программным комплексом по
+визуализации трафика, то есть возможности обнаружения в общих чертах те
+же, что и у маршрутизаторов. Кроме этого прокси обладают возможностью
+хранения трафика проксируемого протокола, избирательной блокировки
+ресурсов доступных в сети по проксируемому протоколу, а также позволяют
+реализовать запуск программы по факту доступа к тому или иному ресурсу
+доступному через проксируемый протокол. То есть потенциально более
+опасны для владельцев rootkit.


-\paragraph{IDS\\}
+\paragraph{IDS и IPS\\}
 \index{IDS}
-Устанавливаются в любой организации всерьёз относящейся к собственной
-безопасности. Представляют из себя мониторы трафика, <<заточенные>> под
+\index{IPS}
+
+Основным отличием IDS от IPS является место установка в сети: IDS прослушивает трафик, а
+IPS пропускает трафик через себя, что позволяет на IPS использовать правила фильтрации
+для блокирования атак и изоляции атакующего. Преимуществом IDS является то, что он не
+вносит задержки в сеть. Недостатком IDS является единственно возможный метод блокирования
+атак – посылка пакетов TCP RST.\\
+
+IDS и(или) IPS устанавливаются в любой организации всерьёз относящейся к собственной
+безопасности. Представляют из себя мониторы трафика, <<заточенные>> под
 выявление неправомерных действий в сети, как то:

 \begin{itemize}
@@ -207,40 +236,57 @@
 \item{запрещённых типов трафика(по портам, протоколам, содержимому)}
 \end{itemize}

-Классические примеры - snort\index{snort} (доступен всем, бесплатен) и cisco IDS
-(дорого, но очень производительно, доступно только организациям
+Классические примеры:\\
+snort\index{snort} (доступен всем, бесплатен\footnote{плюс доп. поддержка за деньги}),
+cisco IDS (дорого, но очень производительно, доступно только организациям
 способным потратить порядка 100 тыс долларов и затем тратить деньги на
-обновления сигнатур). Так же широко известна IDS от Internet Security Systems
-<<RealSecure>>:
-система RealSecure - это интеллектуальный анализатор пакетов с расширенной
-базой сигнатур атак действующий
-в реальном масштабе времени (real-time packet analysis), она относится к
-системам обнаружения атак, ориентированных на защиту целого сегмента сети.
+обновления сигнатур),Ourmon\index{Ourmon}(доступен всем, бесплатен, специально нацелен на
+противодействие ботнетам).\\
+Так же широко известны:\\ IDS от Internet Security Systems
+<<RealSecure>>:\\
+система RealSecure - это интеллектуальный анализатор пакетов с расширенной
+базой сигнатур атак действующий в реальном масштабе времени (real-time packet
+ analysis), она относится к системам обнаружения атак, ориентированных на защиту
+ целого сегмента сети.\\
+
+IBM RealSecure Network IDS\\
+Используемый сенсором модуль обнаружения атак Protocol Analysis Module (PAM) аналогичен модулю используемому в аппаратных устройства Proventia Network IPS. Кроме того, RealSecure Network Sensor может переконфигурировать правила межсетевого экрана Checkpoint Firewall-1 используя протокол OPSEC.

 Также существуют различные коммерческие IDS доступные для фирм <<средней
 руки>>.

-IDS весьма опасны для владельцев rootkit. Для IDS существуют утилиты
-агрегации и обработки статистики, которые могут выдать отчёт о событиях
-связанных с безопасностью в аналогичном наглядном виде, как для
-маршрутизаторов и прокси. Это позволяет например обращать внимание на
-регулярные нарушения, которые в единственном экземпляре не привлекли бы
-внимания.
+IDS и IPS весьма опасны для владельцев rootkit. Для них существуют утилиты
+агрегации и обработки статистики, которые могут выдать отчёт о событиях
+связанных с безопасностью в аналогичном наглядном виде, как для
+маршрутизаторов и прокси. Это позволяет например обращать внимание на
+регулярные нарушения, которые в единственном экземпляре не привлекли бы
+внимания. Также возможна запись трафика по выявлении атак для последующего анализа.
+
+\paragraph{DarkNets}
+\label{darknet}
+ представляют из себя области адресации внутри ЛВС, в которых
+ни один адрес не занят компьютером или другим оборудованием способным отвечать по
+сети. Для DarkNets действуют нормальные правила маршрутизации, но трафик идет
+через сенсор IDS. Наличие unicast трафика в darknet с большой вероятностью говорит
+ о том, что хост инициировавший трафик заражен.

 \paragraph{коммутаторы, концентраторы}
 \index{коммутаторы}\index{концентраторы}
-Существенным интеллектом для обнаружения не обладают, но могут быть
-использованы для подключения монитора трафика (в простейшем случае -
+Существенным интеллектом для обнаружения не обладают, но могут быть
+использованы для подключения монитора трафика (в простейшем случае -
 tcpdump) и записи его на хранение для последующего анализа. Так, все
 коммутаторы компании cisco поддерживают port mirror - зеркалирование
-траффика одного порта в другой.
+траффика одного порта в другой. Некоторые IPS и IDS могут предоставлять возможность
+динамического управления конфигурацией по событиям (такие функции
+также есть у специализированного ПО cisco). Последнее время концентраторы
+ практически не выпускаются (сняты с производства).

 \paragraph{honeypots\\}
 \index{honeypot}
 \index{хонипот}
 Самый эффективный и опасный для владельцев root-kit вариант. HoneyPot
 переводится как <<ловушка>>. Это программа или компьютер, работающие в
-качестве приманки.
+качестве приманки.

 Наиболее продвинутые варианты honeypot'ов делаются профессионалами,
 причем есть проекты, где машина ловушка не просто пассивно ждет, когда
@@ -253,24 +299,28 @@
 \index{интерактивный хонипот}
 \index{интерактивный honeypot}
 Под интерактивным honeypot здесь понимается компьютер, которые
-используется для обычного серфинга сети\footnote{Тут стоит отметить,
-что моя классификация honeypot'ов отличается от ставшей классической
-Spitzner'овской классификации, где honeypot'ы деляться на <<High-interaction>>
-и <<Low-interaction>> - высокоинтерактивные и малоинтерактивные соответственно.
-Спицнер подразумевает под интерактивностью взаимодействие программ -
-низкоинтерактивный хонипот отличается у него от высокоинтерактивного глубиной
- эмуляции сервиса атакуемого злоумышленниками, я же
-использую термин интерактивность для того чтобы отделить honeypot'ы без
+используется для обычного серфинга сети\footnote{Тут стоит отметить,
+что классификация honeypot'ов в этой статье отличается от ставшей классической
+Spitzner'овской классификации, где honeypot'ы деляться на <<High-interaction>>,
+<<medium-interaction>> и <<Low-interaction>> - по глубине эмуляции сервиса.
+Спицнер подразумевает под интерактивностью взаимодействие программ -
+низкоинтерактивный хонипот отличается от высокоинтерактивного глубиной
+ эмуляции сервиса атакуемого злоумышленниками, я же использую термин
+ интерактивность для того чтобы отделить honeypot'ы без
 активного участия человека (неинтерактивные) от honeypot'ов предполагающих
- серфинг сети пользователем. Как показано в данной статье на не серфящий
-сеть компьютер руткит может просто не попасть, поскольку алгоритм заражения
- предполагает наличие пользователя за компьютером}. Таким образом на этот компьютер
-попадают не только активные члены сети в виде червей, вирусов и
+ серфинг сети пользователем (или эмуляцию поведения пользвателя скриптами).
+Как показано в данной статье на не серфящий сеть компьютер руткит может
+просто не попасть (практический пример - распространение так называемого
+ bootkit'а (классификация команды Kaspersky lab) - заражения происходили
+ только для тех компьютеров, которые кликали по линкам на сайтах с вредоносным
+ контентом - открыть сайт было недостаточно), поскольку алгоритм заражения
+ предполагает наличие пользователя за компьютером}. Таким образом, на
+ этот компьютер попадают не только активные члены сети в виде червей, вирусов и
 незадачливых хакеров, но так же закладки устанавливаемые исключительно
 при входе на какой нибудь сайт (например многие сайты с порно продают
 хостинг для спамеров, которые распространяя через этот хостинг закладки
 организуют botnet'ы, которые затем используются для распространения
-спама).
+спама, фишинга и кражи персональных данных).

 Примером такой интерактивной системы honeypots может служить открытый
 недавно микрософтом проект: http://research.microsoft.com/HoneyMonkey/ .
@@ -278,11 +328,14 @@
 После заражения компьютер поступает на анализ к команде специалистов,
 которая  занимается, как минимум,выявлением методов заражения. В худшем
 случае для владельцев root-kits пойманный экземпляр подвергается полному
-реверсингу (дизассемблирование, выявление алгоритма работы, используемых
+реверсингу - дизассемблирование, выявление алгоритма работы, используемых
 для обмена информацией серверов и ресурсов сети.

-Позволить себе интерактивные хонипоты могут только большие компании либо
-большие  госучереждения. Типичные <<заинтересованные лица>> в создании
+Позволить себе интерактивные хонипоты с участием людей могут только большие
+компании либо большие  госучереждения. Однако системы с эмуляцией поведения человека
+не так дороги в обслуживании. Различные организации поддерживают также автоматизированные
+ системы анализа malware (например CWSandbox, )
+Типичные <<заинтересованные лица>> в создании
 хонипотов:

 \begin{itemize}
@@ -294,18 +347,17 @@

 \subparagraph{Реакция на обнаружение закладки}

-В случае попадания в ловушку закладка будет обнаружена по трафику
-гарантированно. Если за время существования в режиме <<подопытного
-кролика>> закладка не проявила видимых  деструктивных действий, то
-дальнейшее его исследование зависит от типа honey-pot'а. Во многих
-honeypot'ах проигнорируют безвредного серфера, или же просто занесут его
-в базу adware ПО. В редких антивирусных компаниях возможно займутся
-боле-менее плотным reversing'ом и в конечном итоге закладка появится в
+В случае попадания в ловушку с мониторингом на внешних к хонипоту устройствах,
+закладка будет обнаружена по трафику гарантированно. Если за время
+существования в режиме <<подопытного кролика>> закладка не проявила
+видимых  деструктивных действий, то дальнейшее его исследование
+зависит от типа honey-pot'а. Во многих honeypot'ах проигнорируют
+безвредного серфера, или же просто занесут его в базу adware ПО.
+В редких антивирусных компаниях, возможно, займутся боле-менее плотным
+reversing'ом и в конечном итоге закладка появится в
 наборе сигнатур какой либо антивирусной компании(а значит, рано или
 поздно - у всех разработчиков антивирусов). Крайне редким, на мой взгляд
 будет случай детального разбора закладки с выяснением подробностей
 алгоритма  её работы. Большинство антивирусных компаний удовлетворится
 списком перехватываемых функций и созданием сигнатуры в базу - для
 лечения на других компьютерах.
-
-
--- items.ver_1.0/0010_rootkit_evasion_of_detection.utf8.tex	2009-05-26 03:29:39.000000000 +0400
+++ items/0010_rootkit_evasion_of_detection.utf8.tex	2009-06-03 04:25:59.000000000 +0400
@@ -8,12 +8,12 @@
 средствами (антивирусным ПО, anti-spyware, antiadware and so on) и
 сетевыми средствами - на маршрутизаторах, прокси, IDS.\\

-Задача противодействия honeypots является задачей комплексной, в ней
-присутствуют элементы локального и сетевого характера, поэтому она
-выделена в отдельную главу номер \ref{honeypot_section} (см. на
+Задача противодействия honeypots является задачей комплексной, в ней
+присутствуют элементы локального и сетевого характера, поэтому она
+выделена в отдельную главу номер \ref{honeypot_section} (см. на
 странице \pageref{honeypot_section_page} ).\\

-Задача по сокрытию rootkit делится на самом деле, как минимум, на три
+Задача по сокрытию rootkit делится на самом деле, как минимум, на три
 задачи:

 \begin{itemize}
@@ -26,22 +26,45 @@

 \subsubsection{Общие рекомендации}

-\paragraph{алгоритмы, проверки, противодействие сетевой трассировке}
+\paragraph{алгоритмы, проверки, противодействие сетевой трассировке\\}

+\paragraph{Со стороны управляющего сервера\\}
 Для наиболее успешного сокрытия результатов проверки на honeypots и
 прочее контролирующий сервер должен внешне одинаково реагировать как на
-<<плохого>>, так и на <<хорошего>> клиента.
-
-Более того, в случае, если на передачу закладкой данных, согласно
-используемому транспортному протоколу возможно выдавать ошибку - это
-следует делать.\footnote{Это поможет откреститься от обвинений обмена
+<<плохого>>, так и на <<хорошего>> клиента.\\
+
+Алгоритм противодействия антивирусным продуктам не должен быть реализован их отключением.
+Отключение работы HIPS/Antivirus может быть замечено самим пользователем.
+
+Более того, в случае, если на передачу закладкой данных, согласно
+используемому транспортному протоколу возможно выдавать ошибку - это
+следует делать.\footnote{Это поможет откреститься от обвинений обмена
 данными с закладкой}

-По возможности часть алгоритма работы закладки должна быть реализована
-на сервере.\footnote{это усложнит и в ряде случаев сделает практически
-невозможным реверсинг}
+При работе с управляющим сервером закладка должна проверяться на точное
+соблюдение инструкций сервера. Например, закладке можно по приходу за
+заданием отдавать команду прийти еще раз через определенный промежуток времени.
+Это позволит выявить закладки в процессе реверсинга и отрабатывая их по отдельному
+алгоритму скрыть действительный алгоритм работы ботнета и полный протокол обмена
+с сервером.
+
+\paragraph{Со стороны бота}
+
+По возможности часть алгоритма работы закладки должна быть реализована
+на сервере - это усложнит и в ряде случаев сделает практически невозможным реверсинг,
+ в частности при возможности установления соединения на timeoute'ах можно
+реализовать протокол проверки на отладку.
+
+В случае если закладка модульная необходимо шифрование модулей, при этом необходимо
+использовать отдельный ключ для каждого модуля. При этом сервер должен выдавать
+только ту часть ключей, которой достаточно для выполнения текущего набора
+задач. Так делается невозможным реверсинг неиспользуемых модулей. Возможна
+реализация последовательного получения ключей с сервера по мере прохождения
+ проверок.
+
+Для усложнения реверсинга возможна реализация интерпретатора своего байт-кода.\footnote{при этом однако всегда есть смысл делать оценку затраты/эффективность. То есть затраты оправданы когда труд противодействующей стороны умножается от вашего в разы, а лучше десятки и сотни раз.}

-\paragraph{работа с файловой системой\\}
+\subparagraph{работа с файловой системой\\}

 Желательно создать следующую схему:

@@ -60,53 +83,60 @@
 \end{enumerate}

 Таким образом можно замаскировать работу с диском под работу самого
-пользователя, либо под работу  screen-saver'а.
+пользователя, либо под работу  screen-saver'а - такая маскировка полезна,
+поскольку активная работа с диском может быть замечена пользователем компьютера
+(светодиодная индикация работы с диском на многих ноутбуках сделана на виду
+у пользователя).


 \subsubsection{Сокрытие при загрузке с внешнего носителя}

-Бороться с обнаружением при загрузке с внешнего носителя можно следующим
-образом:
+Бороться с обнаружением при загрузке с внешнего носителя можно
+следующим образом:
+\begin{itemize}
+\item{Сохранение в BIOS компьютера или комплектующих - пока проверкой этих областей проверяющее ПО не озаботилось.}
+\item{Сохранение легитимных ссылок на места в сети, открытие которых вызовет загрузку дроппера и повторное заражение. То есть сам рукит перезагрузку не переживает.}
+\end{itemize}
+
+Если будет найдена уязвимость в проверяющем файловую систему ПО возможно хранение загрузчика спец. областях файловой системы с загрузкой по факту проверки FS при старте. Однако это легко отслеживается, если проверяется отличие всего диска, а не только файлов в рамках FS.

-\subsubsection{Отсутствие внешней маскировки}
+
+\subsubsection{Вариант 1: Отсутствие внешней маскировки}
 \paragraph{Описание}

+Под отсутствием маскировки можно понимать такое поведение, при котором работа с
+файловой системой и реестром не прячется а исполнение реализуется так, что нет
+отдельного процесса и т.о. закладка не отображается в стандартном мониторе процессов.
+В т.ч. позволяется произвести деинсталляцию закладки (с возможными вариантами в духе
+перехода на другой метод маскировки работы в системе). Маскируется только трафик работы
+с сетью, причем создается также легитимный трафик, который пользователю позволяется разрешить
+или запретить. В идеале закладка должна нести также некую минимальную полезную нагрузку.
+
 \subparagraph{Реестр\\}

 Этот метод состоит в том, что загружающая rootkit часть не прячется от
-просмотра во время работы rootkit. Таким образом при проверке чтения
-реестра разницы просто нет. Такой метод работы подразумевает, что после
-инсталляции rootkit реестр не меняется вообще. Это не может не создавать
-определённых неудобств программирующему rootkit.
-
-Впрочем, возможно размещение остальных <<сторонних>> ветвей реестра в
-отдельном файле с их последующей загрузкой после старта rootkit и
-предъявлением читающим  реестр функциям оригинального реестра.
-
-То есть, в таком случае при работе rootkit всем вызовам предъявляется
-оригинальный реестр без <<сторонних>> ветвей. В любом случае это не
-очень удачный вариант, так как пользователь может устанавливать
-программы, то есть нужно учитывать это в логике обработки реестра. Кроме
-того этот вариант просто трудно программировать.
-
+просмотра во время работы rootkit, также не прячутся временные файлы.
+Таким образом при проверке чтения реестра разницы просто нет. Такой метод
+работы подразумевает, что после инсталляции rootkit реестр не меняется вообще.
+Это не может не создавать определённых неудобств программирующему rootkit.
+
 \subparagraph{файловая система\\}

-Данные rootkit возможно размещать в swap файле, временных системных
-файлах которые изменяются при каждой загрузке ОС. Вариантов достаточно
-много. \footnote{например, можно использовать потоки связанные с
-файлами, если файловая система NTFS. Ну и многое другое.:)}.
-
-Сам rootkit можно  разместить среди прочих драйверов, которых в
-современных ОС более чем достаточно. Можно даже предусмотреть
-деинсталляцию rootkit штатными для ОС средствами деинсталляции драйверов
+Временные данные rootkit возможно размещать в swap файле,
+временных системных файлах которые изменяются при каждой
+загрузке ОС. Вариантов достаточно много.
+
+Сам rootkit размещается среди прочих драйверов, которых в
+современных ОС более чем достаточно. Предусматривается
+деинсталляция штатными для ОС средствами деинсталляции драйверов
 с последующей отработкой какой нибудь подпрограммы на это
-<<недружественное>> действие.
+<<недружественное>> действие.

 \paragraph{Плюсы\\}

 \begin{itemize}
-\item{При правильной реализации должно быть очень похоже на какой нибудь системный драйвер.}
-\item{Возможно совместить с методом сокрытия от мониторов контрольных сумм}
+\item{При правильной реализации должно быть очень похоже на какой нибудь обычный системный драйвер.}
+\item{Возможно совместить с методом сокрытия от програмных мониторов контрольных сумм}
 \end{itemize}

 \paragraph{Минусы\\}
@@ -154,9 +184,9 @@
 \paragraph{Плюсы\\}

 \begin{itemize}
-\item{При правильной реализации данные хранимые rootkit должны
+\item{При правильной реализации данные хранимые rootkit должны
  попадать в список исключений из мониторинга.}
-\item{Возможно совместить с методом сокрытия от проверки с внешнего носителя}
+\item{Возможно совместить с методом сокрытия от проверки с внешнего носителя}
 \end{itemize}

 \paragraph{Минусы\\}
@@ -169,7 +199,7 @@
 \end{itemize}


-\subsubsection{Маскировка под известные закладки.}
+\subsubsection{Маскировка под известные закладки.}
 \paragraph{Описание\\}
 Такой метод базируется на отказе от скрытия факта различий между
 <<offline>> и  <<online>> реестром, т.е. покуда ОС загружена - ключ
@@ -179,7 +209,8 @@
 похожим на какой нибудь известный относительно безобидный вирус. На мой
 взгляд этот метод не заслуживает существенного внимания, хотя может
 применяться, когда необходимо создать у пользователя представление о
-том, что утечка данных произошла через вирусное ПО.
+том, что утечка данных произошла через вирусное ПО или же, что он словил
+относительно безобидное adware.

 \paragraph{Плюсы\\}

@@ -196,7 +227,8 @@

 Основоной метод борьбы в данном случае - устранение несоответствий, что
 может быть достигнуто только при перехвате функций на уровне ядра (а не
-только на уровне hook'ов <<обёрток>> к вызову ядерных функций).
+только на уровне hook'ов <<обёрток>> к вызову ядерных функций) и модификации
+внутренних структур данных с которыми оперируют функции ядра.


 \subsection{Сетевое скрытие rootkit}
@@ -209,7 +241,7 @@
 поведении исполняемого кода на самом компьютере - жертве). Характерным
 для всех продуктов реализующих защиту от заранее неизвестных проблем
 является довольно большой порог срабатывания, так как поведение
-пользователя в различные дни может колебаться. По моей оценке двух/пяти
+пользователя в различные дни может колебаться. По моей оценке двух/пяти
 процентные несоответствия должны быть незаметны на общем фоне в
 большинстве систем. Таким образом медленные атаки (т.е. сцеживание
 информации в час по чайной ложке) имеют гораздо больший шанс остаться
@@ -218,8 +250,8 @@
 за рамки типичного поведения. Огрублённые оценки <<темплейта работы
 пользователя>> на мой взгляд реализуемы достаточно просто. Фактически
 речь идёт о наборе правил, на основании которых должна регулироваться
-работа с сетью (трафик), причем набор  правил надо формировать на основе
-наблюдения за пользователем.
+работа с сетью (трафик), причем набор  правил желательно формировать динамически
+на основе наблюдения за пользователем.

 \subsubsection{Доступ к сетевым дискам}

@@ -240,7 +272,7 @@
 практику можно применить по отношению к локальным дискам, однако
 необходимостью для локальной работы она станет только в случае
 установленных программ реализующих криптодиски и установленных программ
-мониторов работы приложений - второй случай очень редко, но может
+мониторов работы приложений - второй случай относительно редко, но может
 встретиться.


@@ -270,19 +302,33 @@
 При этом характер допустимой работы с сетью можно получить мониторингом
 сетевой активности самого пользователя.

+Вообще говоря асимметрия трафика актуальна только для тунелирования
+в протоколах не использующих шифрование. Внутри ssl заботиться об асимметрии нет смысла.
+
 \paragraph{Шифрование трафика\\}

 Требования к алгоритму шифрования просты: скорость, минимальный размер.
 Выбор остается за разработчиком соответствующего rootkit, могу лишь
-сказать, что заслуживает внимания idea всвязи с
-высокой скоростью его реализации на
-mmx-совместимых процессорах\footnote{таких сейчас большинство}.
+сказать, что заслуживает внимания idea всвязи с
+высокой скоростью его реализации на
+mmx-совместимых процессорах - таких сейчас большинство.
+Кроме того возможно применение разных алгоритмов на различных данных, т.е. если актуальность
+данных для анализа закладки/ботнета в случае их расшифровки ничтожно мала - можно применять на таких
+данных банальный xor. Разумеется payload должен быть зашифрован с использовавнием серьезных алгоритмов
+шифрования.
+
+\paragraph{Получение payload через скрытый канал\\}
+payload передается закладке в зашифрованном виде. Ключ к расшифровке может быть
+передан по условию (например после проверки на sanbox/honeypot и прочих проверок).
+Желательна передача по каналам организованным согласно рекомендациям в \ref{hidden_tunnel} на стр. \pageref{hidden_tunnel}, при этом необходимо чтобы получить их с сервера иначе невозможно было невозможно.\\
+Необходимо исключить возможность записи на диск расшифрованного payload
+и ключа расшифрования.

 \paragraph{Скрытие канала передачи данных\\}
 \label{hidden_tunnel}
 \index{скрытие канала передачи данных}
-Вполне подходящим местом для организации скрытого канала передачи данных
-являются картинки. Так, например, в стандарте на формат jpeg
+Одним из возможных способов для организации скрытого канала передачи данных закладке
+является прием модифицированных картинок. Так, например, в стандарте на формат jpeg
 определяются поля, которые могут использоваться только специальным
 ориентированным на графику софтом (как то adobe photoshop, gimp и
 подобные специальные редакторы графики), причем некоторые - только после
@@ -294,21 +340,22 @@
 Еще один момент, который стоит учитывать - передача информации наружу может,
 но не должна быть организована через картинки, поскольку upload файлов на сервер
 по http происходит весьма редко. Для отправки данных на сервер можно использовать
-запросы с зашифрованными данными внутри запроса (например POST). Чтобы это было
+запросы с зашифрованными данными внутри запроса (например POST). Чтобы это было
 незаметно на фоне остального - подобные запросы, но со случайными данными должны
 идти на случайные сервера.

-Для того чтобы наличие шифротекста было тяжело отличить от просто
-измененной картинки соответствующее поле в картинке должно
-инициализироваться случайной строкой, тогда не зная алгоритма (то есть
-не дизассемблировав закладку) невозможно будет отделить  .
+Для того чтобы наличие шифротекста было тяжело отличить от просто
+измененной картинки соответствующее поле в картинке должно
+инициализироваться случайной строкой, тогда не зная алгоритма (то есть
+не дизассемблировав закладку) невозможно будет отделить мусор от значащих данных.

 Однако, следует понимать, что очистка картинок от таких дополнительных
 полей легко автоматизируется (например есть утилита jpegclean). В то же
 время существуют утилиты которые прячут данные в поля цветности картинки
-без заметного глазу ухудшения качества. Выбор конкретной реализации
+без заметного глазу ухудшения качества. Выбор конкретной реализации
 скрытого канала остается за реализатором закладки. По крайней мере очистка
 jpeg от информации в специальных полях не является сейчас стандартным пунктом
 в настройке прокси систем.

-
+Также при организации тунелирования через картинки необходимо использовать картинки
+с статистически часто попадающимися параметрами (размер, разрешение, цветность и т.д.).
\ No newline at end of file
--- items.ver_1.0/0020_rk_against_honeypots.utf8.tex	2009-05-26 03:29:39.000000000 +0400
+++ items/0020_rk_against_honeypots.utf8.tex	2009-06-06 17:18:48.000000000 +0400
@@ -2,13 +2,13 @@
 \label{honeypot_section}
 \label{honeypot_section_page}

-\subsubsection{Общие замечания.}
+\subsection{Общие замечания.}

-Продиводействие honeypots с существенным успехом возможно только при
+Противодействие honeypots с существенным успехом возможно только при
 взаимодействии с внешним по отношению к honeypot ресурсом. Дело здесь, в
 первую очередь, в том, что honeypot, направленный на выявление закладок,
 может быть абсолютно прозрачен, т.е. может использовать сетевой метод
-выявления закладок и offline'овый метод их анализа. Разумеется в случае
+выявления закладок и offline'овый метод их анализа. Разумеется в случае
 offline'овых средств исследования защититься практически невозможно.
 Конечно можно использовать шифрование, однако ключ расшифрования
 придётся хранить на машине, на которой исполняется программа. Даже в
@@ -25,14 +25,31 @@
 \subsubsection{Возможности}

 Резюмируя вышесказанное,  можно сказать, что противодействие honeypots
- можно разбить на два направления:
+ можно разбить на следующие направления:

 \begin{itemize}
-\item{регистрация известных honeypots на сервере управляющем сетью root-kits}
-\item{максимальное усложнение процесса реверсинга для того чтобы сделать его
-слишком дорогим для рядовых исследователей держателей хонипотов}
+\item{регистрация известных honeypots на управляющем сервере}
+\item{максимальное усложнение процесса реверсинга}
+\item{протокол работы с одновременными двусторонними проверками}
+\item{противодействие дизассемблированию}
+\item{аккуратная работа с LAN}
 \end{itemize}

+\paragraph{регистрация известных honeypots на управляющем сервере}
+часть IP ханипотов можно выявить через публичные services - часть проектов предоставляют free services for malware checks. Также после определенного количества ошибок коммуникации отдельно взятый бот должен признаваться хонипотом и отрабатываться отдельно, в т.ч. с записью в базу как минимум IP.
+
+\paragraph{максимальное усложнение процесса реверсинга}
+ для того чтобы сделать его слишком дорогим для рядовых исследователей
+ держателей хонипотов.
+
+\paragraph{протокол работы с одновременным двусторонними проверками\\}
+Подразумевает разделение протокола работы закладка-сервер на взаимодополняющие части, одновременную эмуляцию которых на одной стороне реализовать невозможно либо крайне трудно.
+
+\paragraph{аккуратная работа с LAN\\}
+Один из вариантов распространения внутри LAN - атаковать только хосты которые уже инициализировали трафик к зраженному, либо хосты к которым заражённый хост соединялся,
+т.е. отказ от активности типа сканирования сети для распространения. Это позволит избежать
+ детектирования с использованием darknet (см. \ref{darknet}, стр. \pageref{darknet}).
+
 \subsubsection{Выявление}

 \paragraph{Обзор методики\\}
@@ -56,7 +73,7 @@

 \begin{itemize}
 \item{Ограничение доступа в сеть для закладки.}
-\item{Ограниченность комплектующих и соответственно парка компьютеров
+\item{Ограниченность комплектующих и соответственно парка компьютеров
 используемых как ловушки.}
 \end{itemize}

@@ -66,8 +83,17 @@
 \item{Атипичная конфигурация софта.}
 \item{Атипичная конфигурация железа.}
 \item{Работа ОС honeypotа в виртуальной машине}
+\item{работа honeypot как usermode или kernelmode rootkit}
 \end{itemize}

+\subparagraph{работа honeypot как usermode или kernelmode rootkit} - наличие перехватов системных таблиц и usermode rootkit - тоже возможно ханипот.
+
+
+\subsubsection{Блокировка части алгоритмов работы закладки}
+
+Невозможность коммуникации с драйвером/модулем процедура инсталяции которого вернула success - один из признаков сандбокса.
+
+
 \subsubsection{Блокировка доступа к сети}

 Блокируют на хонипотах заражённую машину обычно либо сразу по проявлению
@@ -76,28 +102,36 @@
 создаваемому ей подозрительному трафику. Эта особенность объясняется
 тем,  что по законодательству многих стран на деструктивные действия
 производимые с его сетевых адресов на  владельца хонипота могут подать
-иск в суд, причем заведомо выигрывая его, поскольку владельцу хонипота
+иск в суд, причем заведомо выигрывая его, поскольку владельцу хонипота
 сложно будет доказать, что он не знал о деструктивном характере трафика
 закладки, равно как и то что он  не мог его заблокировать. Срок в,
 минимум, две недели объясняется тем, что  немногие владельцы хонипотов
 могут позволить себе ждать проявления активности более длительное время.

 \paragraph{Факт обращения в сеть},
-это очевидно, регистрируется даже если трафик шифруется и, пусть даже
+это очевидно, регистрируется даже если трафик шифруется и, пусть даже
 расшифровать его не представляется возможным, но  сам факт обращений в
-сеть и их характер позволяют отследить управляющий сервер (или несколько
-серверов, что не принципиально, так как число серверов конечно).
+сеть и их характер могут позволить отследить управляющий сервер (или несколько
+серверов, что не принципиально, так как число серверов конечно). Однако
+существуют весьма эффективные способы маскировки управляющих серверов -
+использование паразитного трафика того же типа (но никому не
+предназначенного) и использование средств анонимизации (цепочки анонимных
+прокси, специализированные сети анонимизации работы с интернет, например
+ TOR network \index{TOR}\index{TOR network}). Также могут быть
+использованы способы частичной маскировки, когда при выяснении A-записи
+ управляющего сервера выдаются каждый раз разные данные, например за счет
+ротации в DNS, либо за счет того, что используется доменное имя
+генерируемое по псевдослучайному алгоритму.
 Выявление управляющих серверов приведет, рано или поздно, к их закрытию
-в силу жалоб владельцев honeypot и пострадавшихвладельцев honeypot и
-пострадавших.
+в силу жалоб владельцев honeypot и пострадавших от деятельности ботнета.

 \subparagraph{Однако,}

 это ещё не повод отказываться от зашифрования трафика, поскольку оно
 существенно усложняет проблему реверсирования алгоритма  работы закладки
 и делает необходимым  её дизассемблирование, что
-ресурсозатратно\footnote{Реверсинг  вообще времяемкая и отнюдь не
-простая процедура. =) }. Решение проблемы маскировки сервера будет
+ресурсозатратно - реверсинг вообще времяемкая и отнюдь не
+простая процедура. Решение проблемы маскировки сервера будет
 рассмотрено ниже.

 \subsubsection{Набор железа}
@@ -108,12 +142,11 @@
 закладка за счет неумелого поведения владельцев хонипота сумела
 определить что машина, на  которой она установлена является хонипотом,
 то в дальнейшем сервер может отрабатывать её по отдельному алгоритму.
-Как будет показано далее это позволит и в дальнейшем выявлять установку
+Как будет показано далее, это позволит и в дальнейшем выявлять установку
 на хонипоты собранные с участием того же железа. Поскольку количество
 железа доступного любой конкретной компании ограничено, то хонипоты
 будучи пересобраны для другой конфигурации будут, весьма вероятно,
-содержать комплектующие от предыдущих сборок\footnote{Хотя современный
-компьютер и есть большой компьютер, не стоит забывать, что у каждой
+содержать комплектующие от предыдущих сборок\footnote{Не стоит забывать, что у каждой
 используемой в компьютере железки есть свой серийный номер доступный для
 считывания программным способом.}. Это, в свою очередь, позволяет после
 первого обнаружения узнавать о любых других конфигурациях, в которых
@@ -130,7 +163,7 @@
 также несоответствие здравому смыслу. Возвращаясь к вышеописанному
 примеру, можно заострить внимание на том, что если у пользователя
 нашлись деньги на несколько гигагерцовый Pentium4, то очень странно, что
-при этом у него не было денег хотя бы на 256 Мб памяти и современный
+при этом у него не было денег хотя бы на 256 Мб памяти и современный
 жесткий диск в десятки гигабайт .

 \subsubsection{Атипичная конфигурация ПО}
@@ -146,7 +179,7 @@

 \subsubsection{Работа внутри виртуальной машины}

-Некоторые примитивные хонипоты могут работать внутри виртуальных машин.
+Некоторые хонипоты могут работать внутри виртуальных машин.
 Это может показаться очень удобным владельцу хонипота, однако
 обнаружение таких хонипотов не составляет труда - в сети присутствуют
 примеры исходного кода.
@@ -155,7 +188,8 @@
 \subsection{Решение.}

 \subsubsection{Выявление виртуальных машин}
- В числе прочих проверок необходимо проверять тип машины и, если это
+\label{vm_detection}
+В числе прочих проверок необходимо проверять тип машины и, если это
 виртуальная машина (например ОС  выполняется внутри {\bfэмулятора}
 физической машины, а не на {\bfфизическом} компьютере), то
 предусматривать альтернативное исполнение. Причем чтобы не возникало
@@ -163,6 +197,91 @@
 алгоритма должно происходить на сервере. Реализация выявления
 виртуальной машины проста, в сети есть примеры.

+Однако часть пользователей использует виртуальные машины с windows в
+повседневной работе. Т.е. сам факт работы в виртуальной машине еще не 100\%
+доказательство, что это honeypot. Закладка должна хранить данные о своем
+состоянии, в частности если закладка в прошлом выполнялась в нормальной
+среде, а затем оказалась в виртуальной машине - это скорее всего хонипот
+ или среда для проверки вредоносного ПО в антивирусной компании или у
+энтузиаста-исследователя, поскольку современные производители виртуальных
+ машин не предоставляют возможности загрузки виртуальной машины с физического
+ диска\footnote{так, насколько нам известно в vmware и sun virtual box,
+ранее в vmware была возможность загрузки с физического диска}.
+
+\subsubsection{противодействие дизассемблированию}
+\label{antihoneypot_crypto}
+Шифрование блоков закладки. Нет смысла противодействовать на уровне ином как отсутствие ключей
+расшифрования у авера, поскольку любой другой способ лишь оттянет получение кода - в рамках общей модели угроз\ref{bot_treat_model}.
+
+\subsubsection{обнаружение работы под отладчиком}
+Нет смысла противодействовать отладке, однако ее надо пытаться обнаружить и отрабатывать работу в отладчике отдельно, а именно отрапортовать о хонипоте и не расшифровывать payload.
+
+Из полезных для обнаружения отладки способов можно перечислить следующие:
+\begin{enumerate}
+\item{Проверка CRC исполняемого кода}
+\item{проверка и использование указателя стека}
+\item{перехват int1, int3, int0}
+\item{использование SEH}
+\item{Win32 API IsDebuggerPresent()}
+\item{ключи реестра,процессы,семафоры}
+\item{использование указателя на стек при расшифровании блоков кода}
+\item{расшифрование от конца к началу}
+\item{entry point tricks (TLS)}
+\end{enumerate}
+
+\paragraph{Проверка CRC исполняемого кода\\}
+В x86 архитектуре отладчик в пошаговом режиме использует модификацию исполняемого кода трассируемого приложения вставкой инструкций INT3. Код закладки должен проверять CRC
+исполняющихся участков кода и в случае изменения отрабатывать функционал <<работа под отладчиком>>.
+
+\paragraph{проверка указателя стека\\}
+Отладчик может хранить данные об отладке в стеке отлаживаемого приложения. Для проверки
+запоминаем текущий указатель на стек, кладем в стек любое значение, вынимаем его из стека,
+затем сравниваем указатель на стек с сохраненным - если не равно - работа под отладчиком:
+\begin{verbatim}
+проверка указателя стека:
+MOV     BP,SP  ; Let's pick the Stack Pointer
+PUSH    AX     ; Let's store any AX mark on the stack
+POP     AX     ; Pick the value from the stack
+CMP     WORD PTR [BP-2], AX ; Compare against the stack
+JNE     DEBUG  ; Debugger detected!
+-end{verbatim}
+\subparagraph{использование указателя на стек при расшифровании блоков кода} существенно затруднит отладку поскольку int1 использует стек. Т.е. под отладчиком блок не будет
+расшифрован.
+
+\paragraph{перехват int1, int3, int0\\}
+Прерывания int1 и int3 используются отладчиками. Перехваченные прерывания можно использовать для
+расшифровки блоков кода, а также для рапорта о хонипоте. Перехват int1 и int3 легко
+обходится в отладчиках использующих виртуальные машины, виртуальные машины следует
+распозавать отдельно (см. стр. \pageref{vm_detection}). Некоторые отладчики некорректно
+отрабатывают перехват int0.
+
+\paragraph{использование SEH\\}
+
+SEH - structured exception handling предоставляет из себя использование блоков типа:
+\begin{verbatim}
+try {
+// код который может вызывать exception
+}
+catch {
+// код который обрабатывает возможные exceptions
+}
+-end{verbatim}
+идея в том, чтобы намеренно создавать исключения в блоке try гарантированно
+получая упраление в блоке catch. Есть шанс, что в процессе отладки реверсер может
+пропустить часть кода, которая выполняется в обработчике эксцепшена. Т.е. проверки
+CRC например можно разместить в блоке catch.
+
+\paragraph{ключи реестра,процессы,семафоры\\}
+Отладчиков не так уж и много, соответственно можно обнаруживать соответствующие им ключи реестра, имена процеесов, глобальные системные семафоры и прочие объекты (например имена драйверов). Далее закладка может иметь некий индивидуальный подход к установленному отладчику.Впрочем наличие установленного в системе отладчика еще не говорит о том, что он применяется именно к боту. Наиболее безопасным был бы отказ работы закладки в случае если в системе установлен отладчик, однако таким образом пропускаются компьютеры многих программистов.В зависимости от целей работы данного ботнета можно отрабатыавть ситуацию по разному.
+
+\paragraph{расшифрование от конца к началу\\}
+использование процедур расшифрования таким образом чтобы начало расшифровывваемого блока записывалось последним немного улучшает шансы на выполнение части кода без контроля
+отладчиком - реверсер не может поставить breakpoint на начало блока до его расшифрования
+ - int3 перезапишется данными в процессе расшифрования.
+
+\paragraph{entry point tricks (TLS)\\}
+Windows specific: PE формат позволяет использовать более 1й точки входа при запуске -
+Thread Local Storage (TLS) позволит выполнить код до основной точки входа.

 \subsubsection{Скрытие управляющего сервера}

@@ -175,7 +294,7 @@
 \paragraph{Реализация серфинга}
 должна быть сделана внимательно, корректным образом, поскольку в случае
 ошибок могут возникнуть проблемы со стороны поисковых машин и рядовых
-пользователей.
+пользователей.

 \subparagraph{Во первых,}
  не следует искать случайные наборы символов. Резкое увеличение
@@ -189,35 +308,50 @@
 часто искомых слов, что тоже плохо. Кроме того, если не позаботиться  о
 распределении запросов случайным образом, то нагрузка на поисковые
 системы также сильно возрастет, что  может привести к незапланированным
-отказам в обслуживании, что тем более привлечет внимание.
+отказам в обслуживании, что тем более привлечет внимание.

 \subparagraph{В третьих,}
  следует составлять поисковую строку так, чтобы находить достаточное
 количество  результатов, то есть не более трех-пяти слов в запросе,
 количество должно меняться от запроса к запросу.
-
+
 \paragraph{Инициализация серфинга}

 возможна списком поисковых машин (google, yandex, rambler, yahoo и
 другие ), а набор слов для поиска закладка  может брать из файлов
-носителя либо из собственного словаря в случайном порядке.
-
+носителя либо из собственного словаря в случайном порядке.

-\subsubsection{Инкубационный период}

+\subsubsection{Инкубационный период}
+\label{incubation_period}
 Пользуясь природными аналогиями можно вспомнить, что у многих
 заболеваний есть инкубационный период, когда анализы не дают возможности
 установить факт заболевания. С целью скрытия rootkits от владельцев
 хонипотов следует прибегнуть к тому же средству: в момент инсталляции
 закладка может не проявлять вообще никакой сетевой активности.

-Большой инкубационный период нужен для того чтобы, во первых, обойти
+Так же актуально наличие паузы в несколько минут перед тем, как закладка
+ прописывает себя на диск - часть хонипотов используют раннее обнаружение -
+через несколько секунд (или минут) делается холодная перезагрузка хонипота,
+после чего используются методы offline анализа.
+
+Большой инкубационный период нужен для того чтобы, во первых, обойти
 краткосрочные проверки на очень многих honeypots.

 Инкубационный период, равный месяцу, а тем более двум,  позволит
 избежать выявления закладки во многих интерактивных хонипотах.

-\paragraph{Учет возможности переустановки времени}
+Также возможно деление инкубационного периода на стадии - например:
+\begin{enumerate}
+ \item {пауза от начала выполнения - никакой активности }
+ \item {пауза перед инсталляцией в автозапуск на диск - сбор информации о системе,
+тесты на хонипот без участия сервера}
+ \item {регистрация на сервере с отправкой информации о системе}
+ \item {двусторонние тесты (участвуют и сервер и бот) на хонипот и отладку}
+\end{enumerate}
+
+
+\paragraph{Учет возможности переустановки времени}

 на компьютере является необходимым условием соблюдения сроков
 инкубационного периода. При установке закладки она должна запомнить
@@ -227,24 +361,28 @@
 закладка должна зафиксировать это и, возможно, отреагировать какими либо
 действиями.

-Например, если скачки времени вперед на существенную величину
-производятся постоянно(например, более нескольких раз
+Сервер должен иметь возможность выдать клиенту текущее время,  равно как и клиент серверу. Закладка таким образом может детектить сетевой honeypot за счет выявления разницы между данными полученными по NNTP и по скрытому каналу от сервера. Сервер может сопоставлять
+время на боте с прочими параметрами отрапортованными им выявляя реверсера.
+
+Например, если скачки времени вперед на существенную величину
+производятся постоянно (например, более нескольких раз
 подряд за короткий срок\footnote{<<несколько>> и <<подряд>> - существенные факторы в том
 смысле, что бывает, что люди переустанавливают время на компьютере чтобы
 обмануть регистрационную программу устанавливаемого платного ПО. Однако не стоит
 забывать, что человеку свойственно бывать в отпусках, так что выключение компьютера
-может оказаться вполне нормальным}), при
-наличии доступа в сеть закладка может отрапортовать конфигурацию железа,
-которое с большой вероятностью является honeypot'ом. \footnote{Проверка
+может оказаться вполне нормальным}), также нормальными являются скачки если используется компьютер с несколькими ОС - есть шанс, что пользователь все это время работал в другой ОС. Однако скачки времени вперед будут однозначно характеризовать хост как нетипичный если реальное время (полученное по NTP и от управляющего сервера) существенно отличается.\\
+
+При наличии доступа в сеть закладка может отрапортовать конфигурацию железа,
+которое с большой вероятностью является honeypot'ом. Проверка
 времени по NTP может не дать достоверных результатов, так как обращения
-по NNTP могут перенаправляться, так что сервер должен иметь возможность
-выдать клиенту текущее время, равно как и клиент серверу. Второе - для
-возможности  анализа поведения на сервере.}
+по NNTP и другим стандартным протоколам могут перенаправляться и модифицироваться.
+ Так,  например, CWSandbox\index{CWSandbox} умеет сэмулировать отправку по SMTP
+ не отправляя данных вовне, а лишь представляясь удаленной системой.

 \subsubsection{Тестовый период}
 \label{test_period}

-После окончания инкубационного периода закладка должна пройти тестовый
+После окончания инкубационного периода закладка должна пройти тестовый
 период.

 Во время работы в тестовом режиме закладка   должна серфить сеть также
@@ -260,7 +398,7 @@
 структуре закладки необходимо сделать загрузку закладки ступенчатой.

 \begin{enumerate}
-\item{-я ступень инсталлируется <<как обычно>> - при входе на заражённый
+\item{1-я ступень инсталлируется <<как обычно>> - при входе на заражённый
 web server\footnote{либо в индивидуальном порядке человеком временно получившим доступ к атакуемому
  компьютеру - скачиванием и запуском инсталлятора первой ступени, что в других случаях выполняет эксплойт
  к браузеру}. Начинается инкубационный период.}
@@ -275,11 +413,11 @@

 \begin{enumerate}
 \item{Получение payload через скрытый канал в зашифрованном виде\footnote{поток данных в скрытом канале
-зашифровывается в штатном порядке, то есть ключ и payload будут зашифрованы дважды, но это не
+зашифровывается в штатном порядке, то есть ключ и payload будут зашифрованы дважды, но это не
 существенно - так проще для единообразия}}
 \item{Получение данных для работы payload с сервера через скрытый канал в зашифрованном виде}
 \item{Получение ключа для расшифровки payload через скрытый канал в зашифрованном виде}
-\item{Получение отдельного ключа для расшифровки данных для работы payload через скрытый
+\item{Получение отдельного ключа для расшифровки данных для работы payload через скрытый
 канал в зашифрованном виде}
 \item{хранение payload и ключа его расшифровки только в памяти без записи на диск}
 \item{минимизация количества инструкций payload хранимых в расшифрованном виде}
@@ -290,6 +428,8 @@
  зрения ресурсоемкости
   }
 }
+\item{хранение внутри закладки информации о конфигурации компьютера на момент исполнения с отрабаткой
+ситуаций по перемещению на другой ПК как нештатной.}
 \end{enumerate}


@@ -313,10 +453,10 @@
 \begin{itemize}

 \item{ Использование in memory only хранения payload и ключа его расшифрования
-позволит значительно затруднить реверсинг, а также позволит выдавать ложную
-информацию о задачах которые были поставлены закладке, если реверсирующая
-сторона была обнаружена. Более того, при использовании получения ключа для
-расшифровки исполняемого кода и данных с сервера возникает гораздо больше
+позволит значительно затруднить реверсинг, а также позволит выдавать ложную
+информацию о задачах которые были поставлены закладке, если реверсирующая
+сторона была обнаружена. Более того, при использовании получения ключа для
+расшифровки исполняемого кода и данных с сервера возникает гораздо больше
 узловых точек, в которых реверсер может обнаружить свое присутствие в силу
 несоблюдения протокола обмена данными, в первую очередь - по временным параметрам}

@@ -325,19 +465,20 @@
  утечки информации с носителя rootkit, поскольку, после того как файл признан
  необходимым к хранению очень мало возможных ситуаций, когда для владельца rootkit
  будет осмысленным его расшифрование на клиенте вместо отправки файла на сервер и
- дальнейшей обработки локально, в безопасных условиях, когда нет необходимости
- прятать свои действия от пользователя и проверяющего ПО.
+ дальнейшей обработки локально, в безопасных условиях, когда нет необходимости
+ прятать свои действия от пользователя и проверяющего ПО.
 }
-
+
 \end{itemize}

 \subsubsection{Реализация шифрования}

+\paragraph{Шифрование сетевого трафика\\}
 Некоторые компании имеют маленький интернет трафик и делают полное
 журналирование работы с интернет за день и, возможно, более. Для того
-чтобы исключить расшифровку трафика обмена с сервером для любой закладки
+чтобы исключить расшифровку трафика обмена с сервером для любой закладки
 за счет утечки ключа полученного реверсингом одной закладки необходимо
-реализовать:
+реализовать:

 \begin{itemize}
 \item{индивидуальность сеансового ключа для каждой закладки}
@@ -348,6 +489,7 @@
 их дальнейшего использования для обнаружения установленных закладок в
 других местах.

+
 \paragraph{Сеансовый ключ.\\}

 Под сеансовым ключом понимается ключ используемый для
@@ -358,44 +500,66 @@
 протоколе - набор команд боту или набор данных для последующей
 дообработки - в данном случае не важно.)}

+\paragraph{использование нескольких алгоритмов шифрования\\}
+Первичный запуск должен использовать bruteforce схему, payload должен быть зашифрован с использованием стойких алгоритмов с длиннами ключей не поддающимися bruteforce в
+разумные сроки (сто и более лет). См. также \ref{random_enc_key} на стр.
+ \pageref{random_enc_key} и \ref{decryptor_as_a_part_of_key} на стр. \pageref{decryptor_as_a_part_of_key}.

 \subsection{Реакция}
 \subsubsection{Удачность периода}

-Окончательные выводы о любом периоде, за исключением инкубационного
-\footnote{ Поскольку в инкубационном периоде закладка не проявляет сетевой
-активности - у сервера просто нет данных для выявления его  удачности
-или неудачности.}, должны осуществляться сервером.
+Окончательные выводы о любом периоде в процессе которого идет обмен
+данными с сервером, должны осуществляться сервером.
+Разумеется, если в инкубационном периоде закладка не проявляет
+ сетевой активности -  у сервера просто нет данных для выявления
+ его  удачности или неудачности.

 Поскольку сервер отдает обновления для каждой ступени, то по факту
-обнаружения несоблюдения алгоритма он может выдавать различные файлы с
+обнаружения несоблюдения алгоритма он может:
+\begin{itemize}
+\item{зарегистрировать во внутренней базе данных информацию о характеристиках противодействующей стороны - IP,OS,идентификаторы железа, уникальный идентификатор пойманной закладки и прочее}
+\item {выдавать различные файлы с
 мусором вместо зашифрованных данных в области картинки реализующей
-скрытый транспортный протокол передачи данных .
+скрытый транспортный протокол передачи данных (только если нет средств
+проверки валидности данного типа данных на закладке, иначе передача мусора
+будет говорить о том, что сервер обнаружил противодействие).}
+\item{выдать payload не соответствующий реальному использованию остальных закладок, т.е. фактически
+ скормить любой алгоритм, в следовании которому данного типа закладок
+ надо убедить реверсера/автоматическую систему анализа malware.}
+\item{блокировать обмен данными с закладкой на уровне сети или на уровне протокола в рамках ботнета}
+
+\end{itemize}

 \subsubsection{Варианты реакции на обнаружение хонипота}

+В случае запуска в сандбоксе основная задача - сдетектить и отстучать на C\&C для внесения в базу очередного хонипота.
+
+
 \paragraph{Возможности:\\}
 Возможны различные варианты ответов на выявление неудачности каждого из
-периодов. Их гораздо больше чем описанных здесь. Но наа мой взгляд
+периодов. Их гораздо больше чем описанных здесь, но по мнению авторов
 наиболее разумным является следующее поведение:

 \begin{itemize}
-\item{Выявление неудачи инсталляции на момент инкубационного периода должно
-повлечь, в первую очередь,  попытку отправки информации о хонипоте на
-сервер. Отправка информации должна происходить одновременно  по
-нескольким каналам - smptp(email)/http.}
+\item{Выявление хонипота на момент инкубационного периода должно
+повлечь, в первую очередь,  попытку отправки информации о нем на
+сервер. Отправку информации желательно организовать одновременно  по
+нескольким каналам, например smptp(email)/http/https/dns, как минимум
+ двум}

-\item{Выявление неудачи в тестовом режиме на мой взгляд, должно
+\item{Выявление хонипота в тестовом режиме на мой взгляд, должно
 повлечь выдачу на хост заведомо неправильного бинарника, а если
 точнее - бинарника с совсем другими, почти безобидными функциями.}

-\item{Выявление хонипота в рабочем режиме должно повлечь за собой, как
-минимуму, выдачу, в дальнейшем, закладке, находящейся в хонипоте
-случайного файла с мусором в зашифрованной области, где предполагаются
-данные для обмена с ботом.}
+\item{Выявление хонипота в рабочем режиме должно повлечь за собой
+ блокирование работы с данной закладкой - в нормальной ситуации
+ хонипот должен быть обнаружен на более ранних этапах.}
 \end{itemize}

-\subparagraph{Реакция на обнаружене в инкубационном периое:}
+Следует также по разному отрабатывать хонипот и реверсинг. В частности, при обнаружении
+реверсинга закладки желательно инициировать смену части ключей используемых в ботнете.
+
+\subparagraph{Реакция на обнаружене в инкубационном периоде:}
 Неплохо было бы использование dns туннеля в числе прочих методов.
 Есть немалая вероятность того, что DNS трафик, который по сути своей
 деструктивным быть почти не может (кроме попыток DoS), не будет
@@ -406,22 +570,22 @@
 находится реверсер,он может знать определенную часть общей схемы. Так
 если выявлено, что закладка находится в тестовом режиме, то можно
 предположить, что реверсер уже выявил в ней возможности класса
-<<downloadn and execute>> используемые для обновления. Разумно в таком
-случае отдать реверсеру каку нибудь  безобидную закладку, которая не
+<<download and execute>> используемые для обновления. Разумно в таком
+случае отдать реверсеру какую нибудь  безобидную закладку, которая не
 будет делать ничего особенно неприятного, а например, всего лишь,
 рапортовать список установленного в системе ПО на какой нибудь сервер.
-% 2do: переделать этот консноязычный footnote.
-Однако это подразумевает поддержку такого сервера и другого протокола
+Однако это подразумевает поддержку дополнительного сервера и другого протокола
 обмена между закладками и сервером (чтобы не вызывать подозрений), и,
 таким образом, может быть слишком накладным. Поэтому более подходящим
-может оказаться более подходящей просто выдача, в том или ином виде,
-ошибки соединения.
+может оказаться просто блокирование работы с пойманной закладкой, хотя вариант с
+выдачей некоего <<левого>> payload гораздо вероятнее приведет на
+противодействующей стороне к решению, что закладка уже достаточно изучена и
+дальнейшее исследование можно прекратить.

 \subparagraph{Реакция на обнаружение в рабочем режиме:\\}
-Для начала неплохо бы чтобы владелец хонипота не догадался, что его
-обнаружили. Поэтому можно, например, продолжить работу с данной
-закладкой, но скармливать ей какую нибудь чушь. Вообще говоря
-подозрительные rootkits должны переводиться на один сервер.
+Обнаружение хонипота в рабочем режиме, а не ранее может говорить о том,
+что часть алгоритма работы ботнета раскрыта, так что этот вариант надо отрабатывать
+особым образом.


 \paragraph{База данных клиентов\\}
@@ -429,6 +593,8 @@
 базу данных по железу  на котором исполняется закладка. На  основании
 имеющейся базы данных и ряда проверок сервер\footnote{не клиент!} сможет
 определить\footnote{например используя повторяемость ситуации} с
-достаточно высокой вероятностью имеет ли он дело с honeypot или
-реверсером энтузиастом, или же нет.
+достаточно высокой вероятностью имеет ли он дело с honeypot,
+реверсером энтузиастом, профессиональным реверсер или же с нормальным
+пользователем.

+%По окончании инкубационного периода закладка должна запросить на сервере очередной морф
--- items.ver_1.0/0025_rk_reincarnation.utf8.tex	2009-05-26 03:29:39.000000000 +0400
+++ items/0025_rk_reincarnation.utf8.tex	2009-05-30 12:18:38.000000000 +0400
@@ -9,7 +9,7 @@
 определённых приложений.  Скачивание происходит раз в месяц плюс минус
 несколько дней по случайному принципу. После того как rootkit скачан он
 обращается к серверу за инструкциями и, в зависимости от решения
-принимаемого на сервере получает к исполнению тот или иной бинарь или
+принимаемого на сервере получает к исполнению тот или иной бинарь или
 получает инструкцию выгрузиться из памяти.


@@ -24,7 +24,7 @@
 инфицирование новой  версией rootkit данной машины через достаточный
 промежуток времени. Разумеется только сервер всегда сможет отдать любой
 бинарь, вместо текущей версии опираясь на собственные данные о
-запрашивающей машине.
+запрашивающей машине.

 \subsubsection{Предостережение.}

@@ -35,8 +35,8 @@
 путь. Единственный вариант, при котором такое может быть нужным, если сеть
 закладок осталась <<на автопилоте>> без контроля и за время, которое она так
 работала все или часть антивирусных пакетов вдруг прозрели и стали определять
-наличие закладки и лечить ее, в том числе вылечив \footnote{частично, раз
-независимая закладка-реинкарнатор сохранилась (если это не ловушка конечно)}
+наличие некоторых модулей закладки, производя лечение не полностью, раз
+независимая закладка-реинкарнатор сохранилась (если это не ловушка конечно).



@@ -45,8 +45,8 @@
 К возвращению rootkit на компьютер, с которого он был ранее удален
 требует особенно жестких проверок. Ведь, в данном случае
 {\bfчерезвычайно велик риск напороться на активное
-противодействие и подготовленную ловушку}. Решение о возврате должно
-приниматься индивидуально, а хосты на которые rootkit инсталируется
+противодействие и подготовленную ловушку}. Решение о возврате желательно
+принимать индивидуально, а хосты на которые rootkit инсталируется
 повторно должны быть, во первых, под особым контролем, а во вторых,
 обслуживаться на  отдельных серверах.

@@ -58,7 +58,7 @@

 Это требование связано с тем, что если за сеть закладок возьмутся
 всерьез, то соответствующие  заинтересованные госструктуры могут обязать
-к принятию мер, как минимум, следующих  юридических лиц:
+к принятию мер, как минимум, следующих  юридических лиц:

 \begin{itemize}
 \item{Провайдера хостинга}
@@ -67,17 +67,22 @@

 \subparagraph{провайдера хостинга} можно обязать  предоставить
 физический доступ к данным на компьютере , что решаемо со стороны
-владельца  rootkit использованием зашифрованных файловых
-систем.\footnote{ Тем более это решаемо в случае, если покупается
-хостинг физического компьютера\footnote{датчик открытия корпуса
-завязывается на мобильник, который висит на зарядке питающейся от Б/П.
-По срабатыванию датчика отправляется СМС. Однако, отдельный след:
-предоставить компьютер должен либо живой человек, либо  служба доставки,
-которая, соответственно должна принять его отнюдь не у виртуального
-персонажа, который может иметь лишь идентификатор web-money или pay-pal,
-а у реального человека с паспортом той или иной страны. В общем это на
-самом деле слабое звено, поскольку анонимизация доступна лишь ОПГ и
-различным боле-менее курпным компаниям и госструктурам.}}
-
-
-
+владельца ботнета использованием зашифрованных файловых систем (частично,
+поскольку если сервер виртуальный - может быть применен дамп памяти, то
+есть так  можно получить ключи к файловой системе, следовательно к данным).
+Тем более это решаемо в случае, если покупается хостинг физического
+компьютера - например датчик открытия корпуса завязывается на мобильник,
+ который  висит на зарядке питающейся от Б/П. По срабатыванию датчика
+отправляется СМС. Однако, это отдельный след в real life: предоставить
+компьютер должен либо живой человек, либо  служба доставки, которая,
+соответственно, должна принять его отнюдь не у виртуального персонажа,
+а у реального человека с паспортом той или иной страны - это слабое звено,
+поскольку анонимизация в реальном мире доступна лишь ОПГ\index{ОПГ} и различным
+боле-менее курпным компаниям и специфическим госструктурам.
+
+\subparagraph{провайдера трафика} могут обязать, с какого либо момента, сохранять
+лог обращений и даже трафик работы с сервером по некоторым или всем протоколам
+(СОРМ\index{СОРМ} это предусматривает), таким образом это может помочь в выявлении управляющего
+сервером персонала и, затем, владельца ботнета. Противодействовать этому можно лишь
+используя для административного доступа специальные средства анонимизации - цепочки
+ анонимных прокси и сети подобные TOR\index{TOR}.
--- items.ver_1.0/0030_diffrent_comments.utf8.tex	2009-05-26 03:29:39.000000000 +0400
+++ items/0030_diffrent_comments.utf8.tex	2009-05-30 12:23:17.000000000 +0400
@@ -1,34 +1,31 @@
 \section{Различные комментарии}

-Данная глава на данный момент - разрозненные заметки на полях. Поэтому
-многое  закомментировано и попадет в следующую версию этого документа
-только после значительной доработки.\footnote{2do}.
+Данная глава на данный момент - разрозненные заметки на полях.


 \label{payload_term}

 Термин payload я подобрал в журнале 29A\index{29A}, в котором он использовался для
-обозначения функций выполняемых вирусом не относящихся к заражению и
+обозначения функций выполняемых вирусом не относящихся к заражению и
 распространению. В нашем случае этот термин употребляется как
 обозначение задач rootkit выполняемых по указанию владельца в
 автоматическом режиме (то  есть без специальных на то команд от
 владельца). Пользуясь случаем, хочу  высказать огромное спасибо 29A и
 всем участникам virus scene, кто помог мне своими опубликованными
 идеями.\footnote{Журнал 29A это один из современных (на лето 2004)
-журналов вирусной сцены, на момент написания этой статьи с ним можно
-было ознакомится по url: http://www.29a.host.sk/\label{29A_mag} }
-ВНИМАНИЕ! ЗАХОДИТЬ ТУДА с помощью windows и особенно internet explorer
-ЧРЕВАТО подхватыванием троянца! Впрочем как и на любой другой сайт связанный с
-hacking/virii scene.
+журналов вирусной сцены, на момент начала написания этой статьи
+с ним можно было ознакомится по url: http://www.29a.host.sk/\label{29A_mag}. На 2009й год
+команда 29A это уже, к сожалению, история сцены . }

 \subsection{Reversing}

 Reversing - восстановление алгоритма по <<бинарю>> (см. \ref{glossary}).
 Очевидно, что владелец rootkit заинтересован в максимально возможной
 секретности алгоритма  его работы, так как знание алгоритма помогает
-противодействию экземпляров rootkit и их сетей (kitnet), а также
+противодействию экземпляров rootkit и их сетей (kitnet),
 помогает понять цели установки и исполняемых им в процессе работы
-действий.
+действий а также может помочь в преследовании владельца ботнета
+по законам страны проживания.

 \subsubsection{Обнаружение трассировки}

--- items.ver_1.0/0400_good_rootkit_musthave.utf8.tex	2009-05-26 03:29:39.000000000 +0400
+++ items/0400_good_rootkit_musthave.utf8.tex	2009-05-30 15:14:16.000000000 +0400
@@ -1,13 +1,21 @@
 \section{Набор свойств необходимых к реализации в качественном rootkit}
 \label{tech_spec}
+
+%
+{\sl\bf
+  Внимание: данная глава оставлена на обзорном уровне.
+  Детальная техническая разработка требований к качественному rootkit
+  вынесена в отдельную статью, которая является логическим продолжением
+  данной. Это сделано, в том числе, для облегчения развития статьи.
+}
+
 \paragraph{Обзор главы.\\}

-В этой главе я резюмирую для технических специалистов обсуждавшиеся
-ранее  возможности, которыми, на мой взгляд должна обладать качественная
-реализация rootkit.
+В этой главе резюмируются для технических специалистов обсуждавшиеся
+ранее  возможности, которыми должна обладать качественная реализация rootkit.

-Для того чтобы не было неоднозначностей каждый набор требований поделен
-на части, которые соответственно называются:
+Для того чтобы не было неоднозначностей наборы требований характеризуются
+так:

 \begin{itemize}
 \item{ <<Обязательно>> (английский синоним "MUST")}
@@ -19,7 +27,7 @@

 Эта глава исключительно технического характера и предназначена для
 технических специалистов. Вы \underline{ не } найдете пояснений к
-используемым здесь терминам в глоссарии.
+используемым здесь терминам в глоссарии.

 \subsection{Общие требования}

@@ -67,7 +75,7 @@
 }
 \item{ При написании когда необходимо иметь ввиду портируемость, как клиента,
 так и сервера. То есть как минимум межфункциональное взаимодействие
-должно быть реализовано с возможностью замены функций, блоков функций.
+должно быть реализовано с возможностью замены функций, блоков функций.
 }
 %\item{}
 \end{itemize}
@@ -82,30 +90,43 @@
 работы с языками среднего уровня. В любом случае  Ассемблерные вставки в
 исходный код должны быть прозрачными для замены.

-%
-%\subsubsection{Средний уровень}
-%Возможно, актуальным был бы интерфейс для использования payload/вызова функций языков
-%высокого уровня?
-%

 \paragraph{сервер\\}
 Во избежание проблем с портированием серверная часть должна быть
 полностью реализована на языке среднего/высокого уровня. Предпочтительно
-C/CPP.
+C/CPP.
+
+\subsubsection{Требования к протоколу работы и архитектуре}

-\subsubsection{Требования к протоколу работы}
+Часть пунктов отсюда естественно будут дублироваться в разделах требований к закладкам и серверу.

+\paragraph{Глобальные требования}
 \begin{enumerate}
-\item{Протокол работы должен быть формализуем.}
-\item{Протокол работы должен содержать алгоритм генерации уникальных идентификаторов}
-\item{Протокол должен использовать индивидуальный ключ шифрования для каждой закладки}
-\item{Протокол не должен позволять неоднозначной трактовки.}
-\item{Протокол работы должен быть масштабируемым.}
-\item{Протокол работы должен иметь возможность синхронизации времени.}
-\item{Протокол работы должен иметь проверки на подмену сервера.}
-\item{Протокол работы должен иметь проверки на подмену клиента.}
+\item{Формализуемость без неоднозначностей.}
+\item{Должен использовать шифрование.}
+\item{Должен позволять масштабируемость инфраструктуры поддержки сети.}
+\item{Должен позволять профилирование закладки сервером}
+\item{Должен иметь возможность отчета о времени в обе стороны}
+\item{Должен иметь проверки на подмену сервера.}
+\item{Должен иметь проверки на подмену клиента.}
+\item{Поддержка индивидуального и массового командного режима}
 \end{enumerate}

+\paragraph{Уточнения\\}
+
+\subparagraph{Шифрование\\}
+Как для трафика, так и для payload.
+\begin{enumerate}
+\item{Должен использовать индивидуальный ключ шифрования трафика для каждой закладки}
+\end{enumerate}
+
+\subparagraph{Прочее\\}
+\begin{enumerate}
+\item{Синхронизации времени запуска payload на закладке с точкой отчета.}
+\item{Должен содержать алгоритм генерации уникальных идентификаторов покрывающих не менее двойного объёма IPv6 сети}
+\end{enumerate}
+
+
 \subsubsection{Требования к распределению нагрузки}

 \begin{enumerate}
@@ -121,16 +142,19 @@
 \item{Скрытие канала передачи информации}
 \item{Поддержка режима инкубационного периода}
 \item{Поддержка режима тестового периода}
+\item{Поддержка уникальных идентификаторов для модулей системы}
+\item{использование уникальных идентификаторов позволяющих индексировать не
+менее чем двойной объём IPv6 сети}
 \end{enumerate}

 \subsubsection{Требования к исходному коду}

 В общем-то эти требования можно применить к любому программному проекту.
 Их удовлетворение конечно не обязательно, но должно существенно
-упростить и сократить разработку.
+упростить и сократить разработку.

 \begin{enumerate}
-\item{возможность вынести частей в библиотеку}
+\item{возможность переноса части функционала в библиотеку}
 \item{реализация одного функционала одним кодом}
 \item{написание кода с учетом возможности будущего портирования на другие платформы}
 \item{Минимизация объёма кода написанного на ассемблере по сравнению с C/CPP}
@@ -144,31 +168,54 @@
 \item{работа, как минимум, части закладки в ring0}
 \item{обход всех применяемых local firewalls}
 \item{невидимость всем используемыми антивирусными пакетам (реакции <<подозрительно>> тоже быть не должно)}
-\item{модульность. Как минимум - деление loader/payload}
+\item{модульность. Как минимум - деление dropper/loader/payload}
 \item{шифрование/расшифрование тела rootkit в памяти (<<на лету>>) по ключу с сервера}
 \item{ступенчатая инсталляция}
 \item{in memory only payload}
-\item{маскировка соединений с сервером паразитным трафиком}
+\item{маскировка целевой сетевой активности паразитным трафиком, в частности соединений с сервером}
 \item{адаптация собственной активности под активность пользователя}
 \item{реализация методов обнаружения запуска <<под отладчиком>>}
+\item{профилирование сетевых возможностей в среде функционирования}
+\item{поддержка профилирования возможностей сервером}
+\item{поддержка индивидуального и массового командного режима}
+\end{enumerate}
+
+\paragraph{профилирование сетевых возможностей в среде функционирования} подразумевает выявление
+разрешенных протоколов для работы с интернет.Необходимо, например, для того, чтобы в зараженной сети где запрещен, допустим, ICMP трафик, наружу не могла быть запущена команда icmp DDoS.
+\begin{enumerate}
+\item{пассивное - по сниффингу типов протоколов используемых пользователем}
+\item{активное - по результатам запуска тестов на соединения}
 \end{enumerate}

-\subsubsection{Требования к серверу}

+\paragraph{поддержка профилирования возможностей сервером}
+подразумевает поддержку выставления переменных описывающих тип использования данного экземпляра сервером с использованием фильтрации на этой основе исполнения комманд выдаваемых всему ботнету.

+\subsubsection{Требования к payload}
+
+\paragraph{DDoS}
+\begin{enumerate}
+\item{реализация ICMP DDoS}
+\item{реализация DNS DDoS}
+\item{реализация HTTP DDoS c поддержкой keepalive}
+\end{enumerate}
+
+
+\subsubsection{Требования к серверу}

 \begin{enumerate}
 \item{Event driven модель обработки событий\footnote{как показывает практика - быстрее многопоточной модели}}
 \item{Набор проверок по отношению к клиентам}
 \item{Защита от DoS атак}
-\item{Классификация клиентов по задачам которые на них можно выполнять}
+\item{Классификация клиентов по задачам, которые на них можно выполнять}
 \item{Классификация клиентов по уровню доверия}
 \item{Достаточное быстродействие
 \footnote{как показывает практика монстры вроде apache нагрузки в 200 тысяч ботов
-не выдерживают}
+не выдерживают несмотря на то, что мощности сервера и ширина канала позволяют обработку существенно большего}
 }
 \item{Поддержка включения журнала событий для каждого клиента}
 \item{Поддержка детализированного журнала событий для каждого подозрительного клиента}
+\item{Отдельный алгоритм работы с пойманными ботами}
 \end{enumerate}


--- items.ver_1.0/0600_rootkit_payloads.utf8.tex	2009-05-26 03:29:39.000000000 +0400
+++ items/0600_rootkit_payloads.utf8.tex	2009-05-30 14:00:31.000000000 +0400
@@ -6,10 +6,10 @@
 \paragraph{Массовое единомоментное открытие соединений\\}
 Сотня тысяч компьютеров послав одновременно запрос на использование
 затруднят работу любого сервиса. Использование сетей зомбированных
-компьютеров для организации DDoS атак уже давно стало привычным
+компьютеров для организации DDoS атак уже давно стало привычным
 событием.

-\subparagraph{Раскрытие}
+\subparagraph{Раскрытие}
 взаимодействия в рамках сети зомбированных компьютеров - очевидное
 следствие участия в DDoS атаках. Поэтому, если сохранение rootkit на
 данном компьютере существенно, то его не следует использовать при
@@ -19,36 +19,36 @@

 \subsubsection{Distributed Net ( dnet ) }\label{dnet} - широко известная
  инициатива  интернет сообщества по оценке стойкости алгоритмов шифрования и
-алгоритмов хеширования. Суть её состоит в том, что каждый желающий
+алгоритмов хеширования. Суть её состоит в том, что каждый желающий
 может заставить свой компьютер работать над некой вычислительной задачей
 совместно с компьютерами других энтузиастов. По умолчанию этот проект
 использует остаточную вычислительную мощность компьютера, т.е. то, что
 осталось после выполнения задач ОС и пользователя.

 Пример организации distributed net энтузиастами на территории exUSSR
-можно посмотреть на http://bugtraq.ru/dnet/ .
+можно посмотреть на http://bugtraq.ru/dnet/ .

 Инфицированные компьютеры, точно так же, как и любые другие,
 используются большую часть времени не на полную мощность, так что
 возможно написать модуль к rootkit, который будет использовать время
-простоя этих машин на пользу владельцу root-kit сети.
+простоя этих машин на пользу владельцу root-kit сети.

 \subsubsection{Intelligent Distributed Calculations}

 Совершенно необязательно производить перебор <<подряд>>, как это
 делается в реализации Distributed Net\footnote{такой перебор называют перебором грубой силы - <<bruteforce>>} (см. выше, стр. \pageref{dnet}
 глава \ref{dnet}). Дело в том, что проект Distributed Net
-не слишком заинтересован в оптимизации алгоритма вычислений и
-поддерживается энтузиастами, которые зачастую почти не имею
+не слишком заинтересован в оптимизации алгоритма вычислений и
+поддерживается энтузиастами, которые зачастую почти не имеют
 отношения к криптографии. Однако, согласно Шнайеру, при вычислениях
 связанных, например, с выявлением закрытых ключей существенную роль
 может играть объем доступной памяти. RootKit'у вполне по силам незаметно
 отъесть десяток и более мегабайт памяти на многомегабайтной системе
-\footnote{редкий современный компьютер не имеет хотя бы 128 мегабайт памяти},
+\footnote{редкий современный компьютер не имеет хотя бы 128 мегабайт памяти},
 а на системах со старенькими компьютерами можно использовать меньше памяти.
 Допустим на каждой зараженной машине отъедается, в среднем, 10Mb ОЗУ. Таким образом,
 при объемах <<ботнета>> порядка 300 тысяч компьютеров\footnote{не самая большая сеть}
-количество доступного ОЗУ составляет 3 терабайта. Однако, с учетом необходимых
+количество доступного ОЗУ составляет 3 терабайта. Однако, с учетом необходимых
 накладных расходов (резервирование участвующих вычислительных единиц), существенно меньшая
 скорость доступа к распределенной памяти выигрышь может быть не столь значителен или вовсе
 может отсутствовать - это требует математической оценки.
@@ -57,20 +57,23 @@

 Весьма выгодная деятельность, описанная мной в начале этой статьи в
 качестве примера зачаточной реализации rootkit (как спамерского
-узкоспециализированного бота) и в других статьях. Однако,  следует
-учитывать, что если данный компьютер участвует в рассылке СПАМа это
+узкоспециализированного бота). Спамерство как одна из движущих сил экономической
+оправданности разработки ПО для организации ботнетов обсуждается во многих источниках.
+Однако,  следует учитывать, что если данный компьютер участвует в рассылке СПАМа это
 очень быстро будет  выявлено\footnote{существует множество служб
 выявления спама, борьбы со спамом, включая  регистрацию хостов в
-публично доступных базах данных. Жалобы на рассылку и прочее. Фактически
+публично доступных базах данных. Жалобы на рассылку и прочее. Фактически
 спамеры постоянно покупают новые загрузки, как минимум два-три раза в
 месяц} и с большой  вероятностью повлечёт переустановку ОС
-пользователем. Что в свою очередь повлечёт потерю закладки.
+пользователем. Что в свою очередь повлечёт потерю закладки. Также массовая рассылка спама
+достаточно быстро приведет к пристальному вниманию к самой закладке и значительным усилиям
+по реверсингу ее алгоритма, выявления управляющих серверов (с попытками блокирования серверов и попытками отследить управляющий серверами персонал, а через них выйти на создателя ПО с дальнейшим уголовным преследованием).

 \subsection{Network Analysing}

 Возможность получить из удалённой точки информацию о маршрутизации
 иногда бывает очень кстати.  Вообще для разных IP-сетей некоторый
-хост может иметь разный список правил в своем firewall software,
+хост может иметь разный список правил в своем firewall software,
 причем это может быть реализовано в том числе на уровне принадлежности
 к определенным автономным системам.

@@ -78,14 +81,15 @@

 Передача файлов пользователя без его ведома - вполне доступная для
 rootkit задача. Даже если пользователь использует зашифрованные диски.
-\footnote{Однако, если используется шифрование придется дождаться
-когда система сама (по требованию пользователя вводящего пароль) обратится
-к зашифрованному диску, то есть дождаться входа пользователя в систему с
-монтированием криптодисков}
+\footnote{Однако, если используется шифрование и зашифрованные данные
+не подмонтированы на момент заражения - придется дождаться когда
+система сама (по требованию пользователя вводящего пароль или предоставившего
+носитель с ключами) обратится к зашифрованному диску, то есть дождаться
+входа пользователя в систему с монтированием криптодисков}
 Вообще говоря любые файлы компьютера, на котором установлен rootkit
 помимо пользователя компьютера становятся доступны владельцу
 rootkit.\footnote{см. также \ref{local_data_access} на стр.
-\pageref{local_data_access}}
+\pageref{local_data_access}}


 \subsection{Account Grabs}
@@ -93,12 +97,14 @@
 Современные пользователи чрезвычайно доверчивы и хранят на компьютерах
 очень много информации связанной с доступом в различные места, включая
 то, что не имеет отношения непосредственно к интернет. Владельцу rootkit
-доступна и эта информация. Кроме того rootkit может брать пароли
+доступна и эта информация. Кроме того rootkit может брать пароли
 непосредственно из памяти приложения когда оно запущено\footnote{Например,
 в Mozilla есть функция мастер пароль для хранения паролей к сайтам требующим
-ввода пароля. Пока не введен мастер пароль все пароли хранятся на диске в
+ввода пароля. Пока не введен мастер пароль все пароли хранятся на диске в
 зашифрованном виде. Когда мастер пароль введен пароли расшифровываются и
-загружаются в память, которую, в свою очередь, уже может прочитать закладка}.
+загружаются в память, которую, в свою очередь, уже может прочитать закладка}.
+Кроме того закладка может считывать пароль в момент ввода (если пользователь
+не хранит пароли в легко расшифровываемом хранилище).

 \paragraph{Примеры}
 \begin{itemize}
@@ -115,29 +121,33 @@

 Некоторые компьютеры подключены к хабам, а не коммутаторам, что
 позволяет видеть трафик предназначенный другим компьютерам. Реализация
-такой функции возможна. Также (в индивидуальном порядке) возможно
+такой функции возможна. Также (в индивидуальном порядке) возможно
 применение атак на коммутирующее оборудование для получения доступа к
-транзитному трафику\footnote{Имеются ввиду атаки на ARP и STP.}
+транзитному трафику\footnote{Имеются ввиду атаки на ARP и STP.} Однако
+в сетях крупных компаний часто используются интеллектуальные устройства
+коммутации и маршрутизации совместно с программами мониторинга, т.е. выполнение
+атак подобного рода в автоматическом (да и вручную) режиме может быстро
+выявить наличие заражения.


 \subsubsection{KeyBoard sniffing}

 Очень часто применяемый метод слежения за пользователем - запись всего
 что он набирает на клавиатуре. Помимо прочего так можно красть пароли
-во время ввода.
+во время ввода.

 \subsection{User Stats}

 Возможно получать различную статистику по действиям пользователя.
 Например, список и статистику загружаемых программ (список, время,
 частота использования), историю его серфинга интернет (список, пароли
-доступа и прочее).
+доступа и прочее).

 \subsubsection{Personal Data} \label{local_data_access}

 Очень часто пользователи доверяют компьютеру хранение личных данных.
 Разумеется, если информация есть в компьютере - она доступна для rootkit
-тем или иным образом.
+тем или иным образом.

 \paragraph{Доступность данных на зашифрованных носителях}

@@ -154,7 +164,7 @@
 }

 \item{Во вторых, даже если программа организующая криптодиск ограничивает
-доступ  к нему определенным набором программ - всегда можно либо открыть
+доступ  к нему определенным набором программ -  можно либо открыть
 собственный поток в разрешенном приложении или просто перехватив пароль
 при вводе эмулировать <<законный>> доступ.}

@@ -164,20 +174,20 @@
 rootkit очевидна, необходимо  тестирование доступных программных
 продуктов во избежание досадных ошибок из за которых пользователь сможет
 по неадекватному поведению приложений понять, что компьютер находиться
-под контролем.
+под контролем.

 \subsection{Сбор информации с multimedia устройств компьютера}

-При доступе к системе на уровне OS/драйвера\footnote{ring0 access} не
+При доступе к системе на уровне OS/драйвера (ring0 access) не
 составляет проблем доступ к, например, видеокамере или фотоаппарату
 подключенным к компьютеру. Вопрос лишь в том чтобы написать
 соответствующий модуль к rootkit, который смог бы воспользоваться
-удаленным устройством.
+удаленным устройством.

 \paragraph{Список устройств которые могут быть использованы rootkit\\}

 Такие устройства на мой взгляд следует поделить на те, к которым следует
-обращаться косвенно и те к которым можно обращаться напрямую.
+обращаться косвенно и те, к которым можно обращаться напрямую.

 \subparagraph{Устройства, которые можно использовать непосредственно\\}

@@ -194,7 +204,11 @@

 При этом следует осознавать, что факт доступа к некоторым утройствам
 может журналироваться,  поэтому желательно использовать косвенный доступ
-к удаленным устройствам (см. ниже).
+к удаленным устройствам (см. ниже).\\
+
+Также устройства типа web камеры часто имеют светодиодную индикацию, по которой
+пользователь может понять, что камера используется не им. Для таких устройств
+нужно совмещать доступ с использованием устройства самим пользователем.

 \subparagraph{Устройства, доступ к которым можно осуществлять косвенным образом}

@@ -208,13 +222,16 @@

 \item{ Сканеры(любые) - возможно получение результатов сканирования с последующей их обработкой и возможной отправкой }

+\item{Все устройства с которыми можно работать непосредственно}
+
 \end{itemize}

-Необходимость работы с этими устройствами косвеенным образом очевидна -
+Необходимость работы с этими устройствами косвенным образом очевидна -
 обращение к таким устройствам заметно (индикация на корпусе, шум работы,
 движение механических элементов). Весьма желательно осуществлять
 косвенным образом и доступ к удаленным multimedia устройствам, поскольку
-обращение к ним может журналироваться.
+обращение к ним может журналироваться. При необходимости можно использовать
+косвенный доступ и к устройствам которые можно использовать непосредственно.

 \subsection{ScreenShots}

@@ -238,16 +255,17 @@

 Существовала когда то контора которая организовывала платный сервис
 доставки пакетов с гарантией  анонимности. Ныне контора
-перепрофилировалась и больше таких услуг не поставляет. Идея проста -
-каждый пакет шифруеутся на 3х ключах, отправляется на ближайший сервер
-сети zks . Сервер расшифровывает пакет, отправляет следующему известному
+перепрофилировалась и больше таких услуг не поставляет - есть
+мнение - <<большой брат настоял>>. Идея проста - каждый пакет
+шифруеутся на 3х ключах, отправляется на ближайший сервер сети
+zks . Сервер расшифровывает пакет, отправляет следующему известному
 ему серверу. Не зная адреса и порта назначения. Второй сервер делает то
 же самое. Также не имея понятия куда это все идет и откуда. Далее только
 на 3м сервере пакет  отправляется уже на хост назначения.\\

 Резюме: очень сложно выяснить пути пакета, отправителя и получателя,
-поскольку сервера раскиданы на разных континентах и под разной
-юрисдикцией.
+поскольку сервера раскиданы на разных континентах и под разной
+юрисдикцией.


 \paragraph{Подробнее}
@@ -277,7 +295,7 @@
 отправителя сообщения, и его содержимое. Фактически, Zero-Knowledge
 разместила сервера по всему миру, чтобы максимально затруднить для
 отдельно взятого правительства возможность изъятия содержимого всех трех
-серверов, задействованных в пересылке конкретного сообщения.
+серверов, задействованных в пересылке конкретного сообщения.

 =========
 -end{verbatim}
--- items.ver_1.0/9000_glossary.utf8.tex	2009-05-26 03:29:39.000000000 +0400
+++ items/9000_glossary.utf8.tex	2009-05-30 15:34:37.000000000 +0400
@@ -3,6 +3,10 @@

 \begin{description}

+\item[СОРМ]
+\index{СОРМ}
+- Система Оперативно Розыскных Мероприятий. Для провайдеров соблюдение требований СОРМ обязательно.
+
 \item[ компьютер, машина, комп, ПК, ПЭВМ, ЭВМ ]
 \index{компьютер}
 \index{машина}
@@ -10,39 +14,42 @@
 \index{ПК}
 \index{ЭВМ}
 \index{ПЭВМ}
- - синонимичные понятия, которые,
-надеюсь, не требуют пояснений. Тем не менее, в данной статье (в том числе в этом
+ - синонимичные понятия, которые,
+надеюсь, не требуют пояснений. Тем не менее, в данной статье (в том числе в этом
 глоссарии) термин компьютер
 используется иногда для описания любого устройства существенно умнее калькулятора (то
 есть имеющего возможность обеспечить работу операционной системы, пусть даже специализированной,
 как, например, CISCO IOS. )
-( alternate4fun: computer - a device designed to speed up and automate errors. ;) )
+( alternate4fun: computer - a device designed to speed up and automate errors. )

 \item[ИТ, IT, information technologies]
 \index{ИТ}
 \index{IT}
 \index{information technologies}
- - информационные технологии. Термин очень широко
+ - информационные технологии. Термин очень широко
 распространившийся в современности и не менее широко обобщаемый. Зачастую под IT понимают
 вообще все что связано с компьютерами и, в частности, компьютерными сетями.

+\item[ОПГ]
+\index{ОПГ}
+- Организованная Преступная Группа .

 \item [ interface, интерфейс ]
 \index{interface}
 \index{интерфейс}
  - набор характеристик для взаимодействия между чем либо. Например,
-для взаимодействия человека и компьютера чаще всего используется интерфейс в виде совокупности
+для взаимодействия человека и компьютера чаще всего используется интерфейс в виде совокупности
 монитора, манипулятора типа <<мышь>>, клавиатуры и набора программ для обслуживания событий
-(нажатие кнопки на клавиатуре, например) с этих устройств. В свою очередь, человек, для
+(нажатие кнопки на клавиатуре, например) с этих устройств. В свою очередь, человек, для
 работы с этим интерфейсом,должен иметь интерфейс, как минимум, в виде рук и глаз . В IT
 контексте под интефрейсом подразумевается, обычно, одно из следующих:
 \begin{enumerate}
 \item{внешний вид и характеристики(ток, сопротивление, вольтаж )разъемов на устройстве
 (hardware context)}
 \item{набор правил работы с программой реализуемых ее внешним видом - меню и т.п. (user context)}
-\item{набор функций приложения или операционной системы доступных для программиста}
+\item{набор функций приложения или операционной системы доступных для программиста}
 \end{enumerate}
-Примеры интерфейсов:
+Примеры интерфейсов:
 \begin{enumerate}
 \item{сетевые интерфейсы (плата ethernet, modem(в том числе adsl))}
 \item{физические интерфейсы (ethernet (IEEE802.3), wifi(IEEE 802.11b, IEEE802.11g)), ps2, COM(RS232)}
@@ -75,8 +82,7 @@
  - термин подразумевает, в первую очередь, отсутствие подключения к сети.
 Часто применяется по отношению к программе до запуска (например, когда диск компьютера подключается
 к другому компьютеру для исследования его содержимого, все программы на подключаемом диске находятся
- в offline'овом
-режиме.
+ в offline'овом режиме).

 \item[online, онлайн]
 \index{online}
@@ -89,25 +95,17 @@
 \index{протокол}
  - набор соглашений об обмене информации. В it-контексте
 подразумевается обычно протокол обмена данными по компьютерной сети или между
-программными модулями (в т.ч. определяет объем данных и их логическое
+программными модулями (в т.ч. определяет объем данных и их логическое
 представление на каждых приёме/передаче).
 То есть протокол определяет форму
-запрос/ответ или, иными словами, как понимать данные, которые приходят по
-сети и как их в сеть <<говорить>> (в общем-то, человеческая речь это тоже
+запрос/ответ или, иными словами, как понимать данные, которые приходят по
+сети и как их в сеть <<говорить>> (в общем-то, человеческая речь это тоже
 протокол)

 \item[smtp]
 \index{smtp}
  - simple mail transfer protocol, протокол для передачи почтового трафика.

-\item[traffic, трафик]
-\index{traffic}
-\index{трафик}
- - поток данных, чаще всего имеется ввиду объем данных в единицу
-времени (например загрузка канала), причем трафик именуется зачастую по
-названию протокола согласно которому\footnote{исторически принято говорить
-<<по которому>>} передаются данные, например http-трафик.
-
 \item[сервис,service]
 \index{сервис}
 \index{service}
@@ -120,13 +118,13 @@
 \item[сервер, server]
 \index{сервер}
 \index{server}
- - с точки зрения пользователя - <<какой-то>> компьютер, стоящий
-<<где-то>> (хоть в соседней комнате, хоть на другом континенте), к которому
-можно обратится по сети при помощи той или иной программы. В общем случае
-термин применяется как к компьютеру в целом, так и к программе, которая на
+ - с точки зрения пользователя - <<какой-то>> компьютер, стоящий
+<<где-то>> (хоть в соседней комнате, хоть на другом континенте), к которому
+можно обратится по сети при помощи той или иной программы. В общем случае
+термин применяется как к компьютеру в целом, так и к программе, которая на
 нем установлена (работает) и предназначена для обслуживания клиентов, т.е.
-<<предоставления сервисов>>.  Исторически слолжилось, что также как сервисы
-и клиенты сервера именуются созвучно названию протокола - например http-сервер.
+<<предоставления сервисов>>.  Исторически сложилось, что аналогично сервисам
+и клиентам, сервера именуются созвучно названию протокола - например http-сервер.
 Бывает также, что слово сервер опускают, как бы, подразумевая его. Например
 когда говорят <<прокси>> подразумевают прокси сервер.

@@ -136,7 +134,7 @@
  - с точки зрения пользователя - он сам. В общем случае,
 в it-контексте, это либо компьютер, либо программа, которые сами сервисов
 не предоставляют, но к ним обращаются. См. <<сервер>>, <<сервис>>.
-Исторически слолжилось, что также как сервисы и сервера, клиенты именуются
+Исторически слолжилось, что также как сервисы и сервера, клиенты именуются
 созвучно названию протокола - например http-клиент.

 \item[атака, attack]
@@ -145,7 +143,7 @@
  - в IT -  воздействие на некоторую вычислительную систему или сеть
 или набор данных или алгоритм их обработки. Атаки используются для:
 \begin{enumerate}
-\item{нарушения работы данной
+\item{нарушения работы данной
 системы;}
 \item{модификации алоритма
 ее работы;}
@@ -159,8 +157,8 @@
 \index{Denial of Service}
 \index{отказ в обслуживании}
  - широко распространенный вариант атак
-в сети, суть которых - вывод из строя некоторого сервиса для того чтобы он,
-временно, не мог обслуживать клиентов.
+в сети, суть которых - вывод из строя некоторого сервиса для того чтобы он,
+временно, не мог обслуживать клиентов.


 \item[DDoS, Distributed DoS, Distriuted Denial of Service]
@@ -171,33 +169,33 @@
 большим количеством
 компьютеров через сеть. Чаще всего DoS осуществляется либо за счет превышения пропускной
 способности канала в интернет атакуемого сервера, либо за счет превышения максимального
-поддерживаемого сервером количества открытых соединений.
+поддерживаемого сервером количества открытых соединений.

 \item[транспорт, transport]
 \index{транспорт}
 \index{transport}
- - почти как и <<в миру>> - средство переноса, в it-контексте
-речь идёт о переносе данных. Сеть интернет задумана как иерархическая
+ - почти как и <<в миру>> - средство переноса, в it-контексте
+речь идёт о переносе данных. Сеть интернет задумана как иерархическая
 структура, в том числе это касается и передачи данных, а именно: классическая
-(как у письма) структура каждого протокола
-может быть представлена как две части - заголовок (с адресом) и данные
-(все что осталось); в свою очередь интерпретировать эти данные можно как
+(как у письма) структура каждого протокола
+может быть представлена как две части - заголовок (с адресом) и данные
+(все что осталось); в свою очередь интерпретировать эти данные можно как
 угодно - в том числе можно придумать какой нибудь дополнительный протокол
-для обработки этих данных. Если данные передаются по протоколу <<А>> и для
-обработки полученных данных используется протокол <<Б>>, то
+для обработки этих данных. Если данные передаются по протоколу <<А>> и для
+обработки полученных данных используется протокол <<Б>>, то
 получается, что один протокол как бы вложен в другой. В таких случаях
-говорят что протокол A является транспортным для <<Б>> или <<Б>> ходит
+говорят что протокол A является транспортным для <<Б>> или <<Б>> ходит
 поверх <<А>>.

 \item[туннелирование, tunneling]
 \index{туннелирование}
 \index{tunneling}
  - под этим термином понимается вложение данных одного
-протокола в другой (см. пример для транспорт). Туннели применяются как в
+протокола в другой (см. пример для транспорт). Туннели применяются как в
 <<мирных целях>> в повседневной практике (например для создания защищённых
 сетей установлением соединений поверх шифрующего протокола), так и в целях
 получения/передачи информации из сетей с ограниченным доступом\footnote{чаще
-всего это делают в нарушение правил} - например, если пользователь может
+всего это делают в нарушение правил} - например, если пользователь может
 легально ходить в интернет только по протоколу http, написав соответствующую
 программу он сможет сконвертировать в http трафик любой другой.\footnote{детальное
 описание этого процесса можно найти в интернете}
@@ -223,20 +221,19 @@
 \item[программная закладка]
 \index{программная закладка}
  - исполняемый модуль, созданный для <<нелегальной>>
-работы на компьютере, как то: слежение за <<законным>> пользователем
-компьютера, воровство идентификационной информации, использование
-процессорного времени без ведома владельца, воровство файлов владельца
+работы на компьютере, как то: слежение за <<законным>> пользователем
+компьютера, воровство идентификационной информации, использование
+процессорного времени без ведома владельца, воровство файлов владельца
 компьютера и прочее.

 \item[ОС, Операционная Система]
 \index{ОС}
 \index{Операционная Система}
- - <<системный>> набор программных модулей,
-обеспечивающих минимум возможностей для работы с их помощью других программ и,
-собственно, пользователя. Типичные примеры: MS DOS, Free DOS, MS Windows 3.11,
-MS Windows 95, MS Windows 98, MS Windows XP, MS  Windows 2000, Debian Linux,
-Slackware Linux, Red Hat Linux, ALT Linux, ASP Linux, Solaris, IBM DOS,
-IBM OS/2, IBM AIX.
+ - <<системный>> набор программных модулей, обеспечивающих минимум возможностей для работы с
+их помощью других программ и, собственно, пользователя.
+Типичные примеры: MS DOS, Free DOS, MS Windows 3.11, MS Windows 95, MS Windows 98, MS Windows XP,
+MS  Windows 2000, Debian Linux, Slackware Linux, Red Hat Linux, ALT Linux, ASP Linux, Solaris, IBM DOS,
+IBM OS/2, IBM AIX, Free BSD, Open BSD, Net BSD .

 \item[ПО, Программное Обеспечение, софт ]
 \index{ПО}
@@ -256,22 +253,25 @@
 \index{платформа}
  - термин используемый как для определения сродства компьютерной
 техники (по типу и организации комплектующих), так и для определения сродства
-операционных систем (по набору и организации предоставляемых пользователю и
-программам возможностей, производителю ПО, ОС). Например UNIX и WINDOWS
+операционных систем (по набору и организации предоставляемых пользователю и
+программам возможностей, производителю ПО, ОС). Например UNIX и WINDOWS
 платформы. Платформы различных производителей зачастую имеют несовместимый
-формат исполняемых файлов, например исполняемый файл Linux не запускается в
+формат исполняемых файлов, например исполняемый файл Linux не запускается в
 Windows.

 \item[кроссплатформенный]
 \index{кроссплатформенный}
- - используемый на нескольких платформах.
+ - используемый на нескольких платформах, ПО работающее более чем на одной аппаратной
+ платформе и/или операционной системе. Кроссплатформенность может быть либо на уровне
+ компиляции (из исходного кода можно собрать бинарь более чем для одной платформы или ОС),
+ либо на уровне выполнения - исполняемый файл может быть запущен на разных ОС (платформах).

 \item[червь, worm]
 \index{червь}
 \index{worm}
  - самораспространяющийся, после запуска на компьютере подключённом
-к ЛАН или интернет, программный модуль, использующий для распространения сеть
-(в частности - интернет) и ошибки в ПО обрабатывающем информацию из сети на
+к ЛАН или интернет, программный модуль, использующий для распространения сеть
+(в частности - интернет) и ошибки в ПО обрабатывающем информацию из сети на
 конечных точках сети (как серверах, так и рабочих станциях). В общем случае
 черви - кроссплатформенное явление. Разумеется, возникает не сам по себе, а как
 результат творчества программиста.
@@ -293,12 +293,12 @@
  -  сеть
 из компьютеров, территориально расположенная на большом (географически) пространстве.
 Классические примеры - сеть internet, связанные внутренние сети корпораций с филиалами
-в разных странах.
+в разных странах.

 \item [ ARP, Address Resolution Protocol ]
 \index{ARP}
 \index{Address Resolution Protocol}
-протокол используемый в локальных сетях ethernet для преобразования IP адресов в адреса
+протокол используемый в локальных сетях ethernet для преобразования IP адресов в адреса
 сетевых плат. Атаки с использованием протокола ARP на коммутирующее оборудование
 используют тот факт, что таблицы коммутации имеют конечный размер и после переполнения
 коммутации больше не производится - свитч начинает работать как хаб.
@@ -306,11 +306,11 @@
 \item [ internet, интернет ]
 \index{internet}
 \index{интернет}
- - глобальная (WAN) сеть публичного доступа, объединяет
+ - глобальная (WAN) сеть публичного доступа, объединяет
 компьютеры частных лиц и организаций, используется для досуга, работы, рекламы, публикации
 информации технического и гуманитарного характера, как развлекательного, так и познавательного
 характера. Вряд ли есть область деятельности человека не упомянутая в интернет. Большинство
-организаций имеет подключения к сети интернет в том или ином виде с разной степенью
+организаций имеет подключения к сети интернет в том или ином виде с разной степенью
 ограничений на доступ к сети и извне сети - в зависимости от политики безопасности
 данной организации. Сеть интернет зачастую используется для объединения частных сетей организаций
 через так называемые туннели.
@@ -319,7 +319,7 @@
 \index{RFC}
 \index{Request for Comments}
  - общеупотребительное название стандартов на составляющие
-при помощи которых организована сеть Internet. RFC свободно доступны на соответствующих
+при помощи которых организована сеть Internet. RFC свободно доступны на соответствующих
 серверах интернета и я вляются, по сути, открытыми стандартами, в поиске и исправлении ошибок
 в которых может принять участие каждый желающий.

@@ -328,9 +328,9 @@
 \index{virus}
  - самораспространяющийся после запуска на компьютере программный
 модуль, использующий для распространения модификацию других программных модулей,
-в том числе поставляемых в комплекте ОС и в комплекте с ПО сторонних
-производителей. В последнее время в связи с широким распространением ЛВС
-используют в том числе методы распространения червей (сеть). Разумеется,
+в том числе поставляемых в комплекте ОС и в комплекте с ПО сторонних
+производителей. В последнее время в связи с широким распространением ЛВС
+используют в том числе методы распространения червей (сеть). Разумеется,
 возникает не сам по себе, а как результат творчества программиста.

 \item[троян, троянский конь]
@@ -359,12 +359,12 @@
 \index{address}
 \index{адрес}
  - значение слова практически такое же, как и <<в миру>>, однако в
-ИТ-контексте необходимо понимать следующее: в любой широко используемой
+ИТ-контексте необходимо понимать следующее: в любой широко используемой
 компьютерной сети как глобального, так и локального масштаба используются
-<<адреса>>, отличие от обычного смысла - адрес указывает не физическое
-положение компьютера, а его место в логике сети (её логической структуре).
+<<адреса>>, отличие от обычного смысла - адрес указывает не физическое
+положение компьютера, а его место в логике сети (её логической структуре).
 Адреса в IT области ориентированы не на людей, а на работу программ.
-Также, необходимо понимать, что поскольку компьютер может участвовать в
+Также, необходимо понимать, что поскольку компьютер может участвовать в
 различных сетях - адресов у него может быть несколько, кроме того они
 могут со временем меняться (типичный пример - дозвон поставщику услуг интернет
 - чаще всего при повторном дозвоне провайдер назначит уже другой адрес).
@@ -374,7 +374,7 @@
 \item[Mbit]
 \index{Mbit}
 \index{Мегабит}
- - Мегабит, сокращение используемое для описания пропускной
+ - Мегабит, сокращение используемое для описания пропускной
 способности. Для сравнения типичное для Москвы пользовательское соединение
 с интернет типа <<ADSL>> предоставляет максимальную скорость приёма до 7Mbit,
 но в то же время максимальная скорость отправки всего лишь в 0.7 Mbit - именно
@@ -399,14 +399,19 @@
 в сети зарегистрировано удобное для человека представление. Процесс resolving'а подразумевает
 обращение к серверу DNS.

-\item[трафик]
+\item[traffic, трафик]
+\index{traffic}
 \index{трафик}
- - объем передаваемых данных. Основной доход от деятельности
-провайдеры услуг интернет получают за счет взимания платы за объем переданных
-данных через своё оборудование к подписчикам услуг (то есть оплачивается как
-трафик клиентов, так и трафик серверов) - расходы обсчитываются на каждый адрес
-(из списка адресов принадлежащих сети провайдера) используемый при работе
-в сети интернет.
+ - объем передаваемых данных. Основной доход от деятельности провайдеры
+доступа в интернет получают за счет взимания платы за объем переданных
+данных через своё оборудование для подписчиков услуг (то есть оплачивается как
+трафик клиентов, так и трафик серверов) - расходы обсчитываются на каждый адрес
+(из списка адресов принадлежащих сети провайдера) используемый при работе
+в сети интернет.\\
+ - поток данных, чаще всего имеется ввиду объем данных в единицу
+времени (например загрузка канала), причем трафик именуется зачастую по
+названию протокола согласно которому\footnote{исторически принято говорить
+<<по которому>>} передаются данные, например http-трафик.

 \item[функция]
 \index{function}
@@ -414,7 +419,7 @@
 \index{процедура}
  - некая часть программы, которую можно использовать (вызывать)
 многократно. Функции зачастую называют <<процедурами>>. Современные ОС реализуют
-для программ возможность использовать как свои собственные функции, так и
+для программ возможность использовать как свои собственные функции, так и
 функции находящиеся внутри программ ОС и других программ.

 \item[машинное представление]
@@ -425,7 +430,7 @@
 \index{железо}
 \index{harware}
  - любая аппаратная часть компьютера либо компьютер <<отдельно
-от программ>>.
+от программ>>.

 \item[ядро]
 \index{ядро}
@@ -437,24 +442,24 @@
 \item[низкий, низкоуровневый]
 \index{низкий}
 \index{низкоуровневый}
- - в контексте этой статьи, чаще всего, - максимально
- приближенный
-к внутреннему, например низкоуровневое программирование - либо программирование
-очень близкое к работе с ядром ОС, либо программирование на инструкциях процессора
-(то есть фактически <<в коде>> (циферками то есть), а не командами, которые затем
-переводятся в код (цифры в том или ином машинном представлении).
+ - в контексте этой статьи, чаще всего, - максимально  приближенный к внутренним
+ особенностям работы ОС или компьютера, например  низкоуровневое программирование -
+ либо программирование очень близкое к работе с ядром ОС, либо программирование в
+ непосредственно инструкциях процессора (то есть фактически <<в коде>> (циферками
+ то есть), а не командами, которые затем переводятся в код (цифры в том или ином
+ машинном представлении)).

 \item[rootkit, руткит, RootKit]
 \index{rootkit}
 \index{руткит}
 \index{RootKit}
  - <<самый продвинутый>> вариант реализации программных закладок.
-Для того чтобы удовлетворять классу rootkit закладка должна реализовывать
+Для того чтобы удовлетворять классу rootkit закладка должна реализовывать
 невидимость своего присутствия как для средств обнаружения программ включённых
 в комплект ОС, так и для утилит сторонних производителей. Такой уровень
-невидимости достигается путём перехвата внутренних функций ОС на уровне ядра (самый
+невидимости достигается путём перехвата внутренних функций ОС на уровне ядра (самый
 низкий, или <<самый внутренний>> уровень работы ОС - функций к которым
-обращаются как процессы прикладного режима, так и функции ОС).
+обращаются как процессы прикладного режима, так и функции ОС).

 \item[машинные коды(машинные инструкции)]
 \index{машинные коды}
@@ -485,10 +490,10 @@
 \index{программа}
  - исполняемый модуль программы в виде инструкций процессора на данной
  вычислительной
-системе. Не путать со скриптами и исходными текстами, которые тоже
+системе. Не путать со скриптами и исходными текстами, которые тоже
 есть набор инструкций,
- но не для
-процессора, а для некоей программы (компилятора или интерпретатора).
+ но не для
+процессора, а для некоей программы (компилятора или интерпретатора).
 Синонимичные определения: программа, приложение.

 \item[исходник, исходный текст, source code, source, сырец ]
@@ -513,9 +518,21 @@
 \index{payload}
  - нагрузка, ради которой работает rootkit (или закладка вообще). См. \ref{payload_term}

+\item[сцена]
+\index{сцена}
+В контексте этой статьи - понятие объединяющее людей интересующихся определенной информацией
+и создающих некий контент интересный, в основном, в рамках этого круга. Существует демо-сцена,
+vx-сцена и другие типа сцены. В рамках таких объединений люди творят ради искусства,
+самореализации и обмена идеями -  <<just for fun>> .
+
+\item[vx-сцена,вирусная сцена]
+\index{vx-сцена}
+\index{вирусная сцена}
+Сцена вирмейкеров. Людей, которые пишут вирусы. Не для того (по крайней мере не обязательно) чтобы навредить другим, а <<just for fun>>.
+
 \item[29A]
 \index{29A}
- - журнал вирусной сцены, см. сноску в \ref{29A_mag}
+ - журнал команды вирмейкеров. Широко известен в рамках вирусной сцены, см. сноску в \ref{29A_mag}

 \item[алгоритм]
 \index{алгоритм}
@@ -533,7 +550,10 @@
 \item[отладчик, debugger]
 \index{отладчик}
 \index{debugger}
- - программа для трассировки. Позволяет пошагово выполнять машинные инструкции трассируемого приложения.
+ - программ для пошагового выполнения некоего кода, не обязательно машинного (в т.ч. бывают отладчики
+для скриптовых языков программирования) с возможностью просмотра значений переменных используемых в программе - используется для исправления ошибок (отладки) программ.\\
+ - программа для трассировки. Позволяет пошагово выполнять машинные инструкции трассируемого
+ приложения с просмотром значений в регистрах процессора, памяти.

 \item[трассировка, отладка, debugging]
 \index{трассировка}
@@ -541,8 +561,10 @@
 \index{debugging}
  - процесс пошагового прохождения работы программы с использованием
 специальной программы отладчика(debugger'а), которая в свою очередь использует для перевода исполняемого
-файла в пошаговый режим специальные возможности ОС и процессора. Существуют методики определения программой
-того, что её трассируют, равно как и методы сокрытия этого факта от программы.
+файла в пошаговый режим специальные возможности ОС и процессора. Существуют методики определения программой того, что её трассируют, равно как и методы сокрытия этого факта от программы. Вообще говоря трассировка и отладка - не одно и тоже, но часто эти понятия используются именно в указанном контексте.
+Точные значения этих понятий таковы:\\
+Отладка - этап разработки компьютерной программы, на котором обнаруживают, локализуют и устраняют ошибки.\\
+Трассировка - пошаговое выполнение программы с остановками на каждой команде (assembler) или строке (языки более высокого уровня).

 \item[реверсер (reverser)\label{reverser}]
 \index{реверсер}
@@ -552,10 +574,18 @@
 \item[jpeg, джипег]
 \index{jpeg}
  - название формата хранения картинок.
+ - формат файла, который содержит сжатые данные обычно также называют именем JPEG, наиболее
+ распространённые расширения для таких файлов .jpeg, .jfif, .jpg . формат jpg использует сжатие с
+ потерями, при сохранении JPEG-файла можно указать степень качества, а значит и степень сжатия, которую
+ обычно задают в некоторых условных единицах, например, от 1 до 100 или от 1 до 10. Большее число
+ соответствует лучшему качеству, но при этом увеличивается размер файла. Обыкновенно, разница в
+ качестве между 90 и 100 на глаз уже практически не воспринимается. Следует помнить, что побитно
+ восстановленое изображение всегда отличается от оригинала.

 \item[гиф, gif]
 \index{gif}
  - название формата хранения картинок.
+- формат GIF способен хранить сжатые данные без потери качества в формате до 256 цветов. Независящий от аппаратного обеспечения формат GIF был разработан в 1987 году (GIF87a) фирмой CompuServe для передачи растровых изображений по сетям. В 1989-м формат был модифицирован (GIF89a), были добавлены поддержка прозрачности и анимации. GIF использует LZW-компрессию, что позволяет неплохо сжимать файлы, в которых много однородных заливок (логотипы, надписи, схемы).

 \item[ресурс]
 \index{ресурс}
@@ -579,15 +609,25 @@
 \index{browser}
  - http клиент. смотрелка страниц в интернете.

-\item[ exploit, эксплойт ]
+\item[ exploit, эксплойт, сплойт ]
 \index{exploit}
 \index{эксплойт}
+\index{сплойт}
  - программный код для использования уязвимости.

+\item[dropper, дроппер]
+\index{дроппер}
+\index{dropper}
+чаще всего загрузка и запуск закладки осуществляется небольшим программным модулем, который включается в себя функционал эксплойта, downloader'а (загрузчика), executor (выполнение скачанного основного модуля).
+
 \item[IE, Internet Explorer, ИЕ]
 \index{IE}
 \index{Internet Explorer}
- - браузер используемый в windows системах <<из коробки>>, то есть, если пользователь не инсталлировал другой бродилки по сети IE окажется бродилкой <<по умолчанию>>. Де факто самый используемый в сети браузер. Де факто один из самых <<дырявых>> браузеров, под IE пишется большинство exploits, что, в прочем, объясняется в первую очередь его популярностью.
+ - браузер используемый в windows системах <<из коробки>>, то есть, если пользователь не
+ инсталлировал другой бродилки по сети IE окажется бродилкой <<по умолчанию>>. Де факто
+ самый используемый в сети браузер. Де факто один из самых <<дырявых>> браузеров: под
+IE и расширения к нему пишется большинство exploits. Однако, это объясняется в
+первую очередь его популярностью - в других браузерах тоже находят уязвимости.

 \item[ урл, url, линк, ссылка, сайт, web-ресурс ]
 \index{урл}
@@ -601,17 +641,25 @@
 \item[баннер]
 \index{баннер}
  - небольшая картинка рекламного или информационного характера. Баннеры используются,
-для рекламы, в том числе контекстной, а так же для привлечения внимания и как легко заметные ссылочные элементы (в таком случае к баннер <<привязывается>> web-дизайнером некий ссылочный url.
+для рекламы, в том числе контекстной, а так же для привлечения внимания и как легко заметные ссылочные элементы (в таком случае к баннеру <<привязывается>> web-дизайнером некий ссылочный url).

 \item[баннерная сеть]
 \index{баннерная сеть}
- - термин используется для группировки размещаемых в сети баннеров по некоторому общему признаку. В частности баннерные сети организуют компании предоставляющие услуги рекламного характера и услуги по номинированию рейтингов сайтов. За использование баннерной сетью может взиматься плата. Создание и использование бесплатных баннерных сетей может окупаться за счет показа рекламы в теле баннер(изображение обновляется с серверов баннерной сети). Баннерные сети работают поверх протокола http и его расширений.
+ - термин используется для группировки размещаемых в сети баннеров по некоторому общему признаку.
+ В частности баннерные сети организуют компании предоставляющие услуги рекламного характера и услуги
+ по номинированию рейтингов сайтов. За использование баннерной сетью может взиматься плата. Создание
+ и использование бесплатных баннерных сетей может окупаться за счет показа рекламы в теле баннер
+(изображение обновляется с серверов баннерной сети). Баннерные сети работают поверх протокола http
+ и его расширений.

 \item[ нулевое кольцо, ring3, ring0 ]
 \index{нулевое кольцо}
 \index{ring3}
 \index{ring0}
- - термины характерные для описания программ работающих на intel PC - совместимых процессорах (x86, начиная с 286). Указывают на уровень привилегий которые использует в работе программа. ring3 - самый бесправный код, ring0 - самый привилегированный. В ring0 обычно работает ядро ОС, в ring3 - пользовательское ПО и часть ПО ОС не требующая привилегий ring0.
+ - термины характерные для описания программ работающих на intel PC - совместимых процессорах (x86,
+ начиная с 286). Указывают на уровень привилегий которые использует в работе программа. ring3 - самый
+ бесправный код, ring0 - самый привилегированный. В ring0 обычно работает ядро ОС, в ring3 -
+ пользовательское ПО и часть ПО ОС не требующая привилегий ring0.

 \item[ конфигурабельно, настраиваемо ]
 \index{конфигурабельно}
@@ -620,29 +668,37 @@
 \item [ система, системный ]
 \index{система}
 \index{системный}
- - в контексте этой статьи система это чаще всего ОС, системный - относящийся к встроенным в ОС функциям, базам данных, файлам.
+ - в контексте этой статьи система это чаще всего ОС, системный - относящийся к встроенным в ОС функциям, базам данных, файлам.

-\item[ API, Application Programm Interface, АПИ ]
+\item[ API, Application Programming Interface, АПИ ]
 \index{API}
-\index{Application Programm Interface}
+\index{Application Programming Interface}
 \index{АПИ}
- - набор функций предоставляемых ОС, в частности, для пользовательских программ. Бывают системные API (для использования модулями ОС) и пользовательские API (для использования прикладными программами).
+ - набор функций предоставляемых ОС, в частности, для пользовательских программ. Бывают системные
+ API (для использования модулями ОС) и пользовательские API (для использования прикладными программами).

 \item [ реестр ]
 \index{реестр}
- - в контексте данной статьи и вообще в контексте Windows совместимых систем: системная база данных, для использования которой каждое приложение может использовать пользовательское API. Для реестра windows характерно использование значений в машинном представлении (то есть значений не понятных не профессионалу). Реестр активно используется большинством windows приложений.
+ - в контексте данной статьи и вообще в контексте Windows совместимых систем: системная база данных,
+ для использования которой каждое приложение может использовать пользовательское API. Для реестра
+ windows характерно использование значений в машинном представлении (то есть значений не понятных не
+ профессионалу). Реестр активно используется большинством windows приложений.

 \item[драйвер, driver]
 \index{драйвер}
 \index{driver}
- - ПО из комплекта ОС или поставляемое сторонним производителем для поддержки работы ОС с некоторым устройством (железом).
+ - ПО из комплекта ОС или поставляемое сторонним производителем для поддержки работы ОС с
+ некоторым устройством (железом).

 \item[полиморфизм, полиморфность, мутации кода, пермутация]
 \index{полиморфизм}
 \index{полиморфность}
 \index{мутации кода}
 \index{пермутация}
- - подразумевается методика изменения бинарного кода основанная на том, что <<на языке процессора можно сделать одно и то же разными словами>>. Практически представляет собой вариацию бинарного кода с заменой инструкции и блоков инструкций на эквивалентные по смыслу (действию).
+ - подразумевается методика изменения бинарного кода основанная на том, что
+ <<на языке процессора можно сделать одно и то же разными словами>>. Практически
+ представляет собой вариацию бинарного кода с заменой инструкции и блоков
+инструкций на эквивалентные по смыслу (действию).

 \item[морф]
 \index{морф}
@@ -651,36 +707,53 @@
 \item[ фича, feature, возможность, наворот, фичастость ]
 \index{фича}
 \index{feature}
- - подразумевается наличие неких <<продвинутых>>, расширенных возможностей, по сравнению с чем то типичным. Может применяться как к софту, так и к железу.
+ - подразумевается наличие неких <<продвинутых>>, расширенных возможностей, по сравнению
+ с чем то типичным. Может применяться как к софту, так и к железу.

 \item[firewall, файервол, brandmouer, брэндмауэр]
 \index{firewall}
 \index{файервол}
 \index{brandmouer}
 \index{брэндмауэр}
- - средство контроля трафика в сети. Позволяет, в зависимости от уровня разработки (<<фичастости>>) продукта контролировать работу с транспортными протоколами (IP,udp), так и работу на уровне приложений (http, ftp клиенты и прочее).
+ - средство контроля трафика в сети. Позволяет, в зависимости от уровня разработки
+(<<фичастости>>) продукта контролировать работу с транспортными протоколами (IP,udp),
+ так и работу на уровне приложений (http, ftp клиенты и прочее).

 \item[proxy,прокси, проксик, прокся]
 \index{proxy}
 \index{прокси}
- - некоторая программа, которая работает в режиме посредника при работе с сетью. Чаще всего software решение. Прокси классифицируются по типам протоколов. Наиболее распространены http прокси. Классический прокси требует настройки браузера для того чтобы его использовать.
+ - некоторая программа, которая работает в режиме посредника при работе с сетью.
+Чаще всего software решение. Прокси классифицируются по типам протоколов.
+Наиболее распространены http прокси. Классический прокси требует настройки
+браузера для того чтобы его использовать.

 \item[прозрачный прокси]
 \index{прозрачный прокси}
- - прокси, который работает с перенаправлением трафика и таким образом не требует настройки в браузере. Классическое использование - на серверах через которые организация подключена к интернет.
+ - прокси, который работает с перенаправлением трафика и таким образом не требует
+ настройки в браузере. Классическое использование - на серверах через которые
+ организация подключена к интернет.

 \item[personal firewall, персональный файрволл]
 \index{personal firewall}
 \index{персональный файрволл}
- - программное обеспечение контролирующее обращения программ к ресурсам сети, настройка доступна администратору компьютера, в простейшем случае настраивает тот кто пользуется. Настройка производится через графический интерфейс, чаще всего по факту обращения программы в сеть или обращения к ресурсам компьютера из сети.
+ - программное обеспечение контролирующее обращения программ к ресурсам сети,
+ настройка доступна администратору компьютера, в простейшем случае настраивает
+ тот кто пользуется. Настройка производится через графический интерфейс, чаще
+ всего по факту обращения программы в сеть или обращения к ресурсам компьютера из сети.

 \item[botnet]
 \index{botnet}
- - сеть образуемая инфицированными компьютерами. Управляется владельцем ботнет. Чаще всего владелец ботнет является его создателем. Но при этом отнюдь не всегда владелец ботнета автор закладок при помощи которых botnet функционирует.
+ - сеть образуемая инфицированными компьютерами. Управляется владельцем ботнет.
+Чаще всего владелец ботнет является его создателем. Но при этом отнюдь не
+всегда владелец ботнета автор закладок при помощи которых botnet функционирует.

 \item[маршрутизатор]
 \index{маршрутизатор}
- - устройства обеспечивающие функционирование сетей, в частности, интернет. Их задача - разделение сетевого трафика в зависимости от типа и направления. В организации есть обычно как минимум один прибор выполняющий функции маршрутизатора - тот, через который организация соединена с internet. Самый примитивный пример маршрутизатора - ADSL modem. Маршрутизаторы работают, как минимум, с транспортными протоколами.
+ - устройства обеспечивающие функционирование сетей, в частности, интернет. Их
+задача - разделение сетевого трафика в зависимости от типа и направления. В
+организации есть обычно как минимум один прибор выполняющий функции маршрутизатора - тот,
+ через который организация соединена с internet. Самый примитивный пример маршрутизатора - ADSL modem.
+ Маршрутизаторы работают, как минимум, с транспортными протоколами.

 \item[коммутатор, свитч]
 \index{коммутатор}
@@ -700,7 +773,9 @@
 \item[IDS, Intrusion Detection System, ай-ди-эс ]
 \index{IDS}
 \index{Intrusion Detection System}
- - программно-аппаратный комплекс для обнаружения в сетевом трафике как следов успешных вторжений / нарушений политики использования локальной сети, так и попыток вторжений/атак на устройства сети (компьютеры, маршрутизаторы, коммутаторы).
+ - программно-аппаратный комплекс для обнаружения в сетевом трафике как следов успешных
+вторжений / нарушений политики использования локальной сети, так и попыток вторжений/атак
+ на устройства сети (компьютеры, маршрутизаторы, коммутаторы).

 \item[порт]
 \index{порт}
@@ -731,30 +806,48 @@
 \item[honeypot, хонипот]
 \index{honeypot}
 \index{хонипот}
- - ловушка. В контексте статьи программа или компьютер, работающие в качестве приманки для <<злоумышленников>>, наиболее продвинутые варианты honeypot'ов делаются профессионалами, причем есть проекты, где машина ловушка не просто пассивно ждет, когда на неё попадет незадачливый взломщик, а используется для простейших действий (например серфинга сети), что позволяет таким ловушками при определенном стечении обстоятельств получить закладки инсталлируемые автоматически только на компьютеры используемые в интерактивном режиме.
+ - ловушка. В контексте статьи программа или компьютер, работающие в качестве приманки для
+ <<злоумышленников>>, наиболее продвинутые варианты honeypot'ов делаются профессионалами,
+ причем есть проекты, где машина ловушка не просто пассивно ждет, когда на неё попадет
+ незадачливый взломщик, а используется для простейших действий (например серфинга сети),
+ что позволяет таким ловушками при определенном стечении обстоятельств получить закладки
+ инсталлируемые автоматически только на компьютеры используемые в интерактивном режиме.

 \item[интерактивный хонипот]
 \index{интерактивный хонипот}
- - ловушка в виде компьютера, который используется человеком для серфинга сети с целью подцепить закладки инсталлируемые пользователям во время серфинга.
+ - ловушка в виде компьютера, который используется человеком для серфинга сети с целью
+ подцепить закладки инсталлируемые пользователям во время серфинга.

 \item[ACL, access control list ]
 \index{ACL}
 \index{access control list}
- - обобщённое название ограничений которые могут в различных устройствах устанавливаться на некоторые действия совершаемые подконтрольным объектом (программой, потоком данных через сеть).
+ - обобщённое название ограничений которые могут в различных устройствах устанавливаться
+ на некоторые действия совершаемые подконтрольным объектом (программой, потоком данных через сеть).

 \item[adware]
 \index{adware}
- - термин используемый для названия программ, которые инсталлируются с каким либо прикладным ПО (иногда самовольно, иногда - согласно лицензионному соглашению о использовании программы, которое, впрочем, мало кто читает) и занимаются сбором и отправкой разработчику ПО различной не связанной с <<шпионскими функциями>> информации, например рапортуют статистику использования программного продукта и прочие не имеющие отношения к личности пользователя детали.
+ - термин используемый для названия программ, которые инсталлируются с каким либо
+ прикладным ПО (иногда самовольно, иногда - согласно лицензионному соглашению о
+ использовании программы, которое, впрочем, мало кто читает) и занимаются сбором
+ и отправкой разработчику ПО различной не связанной с <<шпионскими функциями>>
+ информации, например рапортуют статистику использования программного продукта и
+ прочие не имеющие отношения к личности пользователя детали.

 \item[spyware]
 \index{spyware}
- - термин используемый для обозначения закладок, которые инсталлируются вместе с каким либо прикладным ПО (иногда самовольно, иногда - согласно лицензионному соглашению о использовании программы, которое, впрочем, мало кто читает) и занимаются сбором и отправкой разработчику ПО различной информации личного характера о пользователе. Например, отсылают информацию о истории его серфинга сети.
+ - термин используемый для обозначения закладок, которые инсталлируются вместе с
+каким либо прикладным ПО (иногда самовольно, иногда - согласно лицензионному
+соглашению о использовании программы, которое, впрочем, мало кто читает) и
+занимаются сбором и отправкой разработчику ПО различной информации личного
+ характера о пользователе. Например, отсылают информацию о истории его серфинга сети.
 Используется, в лучшем случае, для контекстной рекламы.

 \item[клиент-серверная архитектура]
 \index{клиент-серверная архитектура}
- - термин применяемый для описания взаимодействия между объектами (в IT это компьютеры или программы), при котором, упрощённо говоря, один компьютер использует ресурсы другого. Чаще всего количество компьютеров(программ) серверов меньше, чем количество компьютеров клиентов(обычно, как
- минимум,  в несколько раз).
+ - термин применяемый для описания взаимодействия между объектами (в IT это компьютеры
+ или программы), при котором, упрощённо говоря, один компьютер использует ресурсы другого.
+ Чаще всего количество компьютеров(программ) серверов меньше, чем количество компьютеров
+ клиентов(обычно, как  минимум,  в несколько раз).

 \item [ сниффинг, sniffing ]
 \index{сниффинг}
@@ -763,7 +856,7 @@

 \item [ контент ]
 \index{контент}
- - наполнение. Чаще всего речь идёт о содержимом web-страницы.
+ - наполнение. Чаще всего речь идёт о содержимом web-страницы.

 \item [ хост, host ]
 \index{host}
@@ -772,12 +865,12 @@

 \item [ загрузка ]
 \index{загрузка}
- - в контексте спама - покупка заражений через hosting-сервера. Вообще в IT, в
-зависимости от
-контекста - либо используется для обозначения процесса запуска программы (загрузка в ОЗУ компьютера), либо для обозначения использования некоторого устройства в смысле синонимичном слову <<нагрузка>> (загрузка процессора, загрузка канала доступа к интернет).
+ - в контексте спама - покупка заражений через hosting-сервера. Вообще в IT, в
+зависимости от контекста - либо используется для обозначения процесса запуска программы (загрузка в ОЗУ компьютера), либо для обозначения использования некоторого устройства в смысле синонимичном слову <<нагрузка>> (загрузка процессора, загрузка канала доступа к интернет).

-\item [ covert channel, hidden channel, скрытый канал ]
+\item [ covert channel, covered channel,hidden channel, скрытый канал ]
 \index{covert channel}
+\index{covered channel}
 \index{скрытый канал}
 \index{hidden channel}
  - в IT подразумевается канал передачи информации путем
@@ -792,7 +885,7 @@
 \index{bruteforce}
 \index{лобовая атака}
 \index{полный перебор}
-- в криптографии/криптоанализе под этим подразумевается поочередная подстановка всех возможных паролей или ключей, которые могут быть использованы для расшифрования. Чаще всего это самый неэффективный метод, однако для некоторых алгоритмов зашифрования не опубликовано иных алгоритмов атаки.
+- в криптографии/криптоанализе под этим подразумевается поочередная подстановка всех возможных паролей или ключей, которые могут быть использованы для расшифрования. Чаще всего это самый неэффективный метод, однако для некоторых алгоритмов зашифрования не опубликовано иных алгоритмов атаки.

 \label{glossary_end}
 \end{description}
--- items.ver_1.0/9100_bibliography.utf8.tex	2009-05-26 03:29:39.000000000 +0400
+++ items/9100_bibliography.utf8.tex	2009-06-06 17:18:48.000000000 +0400
@@ -36,7 +36,35 @@
 \begin{verbatim}
 Олег Артемьев, Владислав Мяснянкин
 Опасные деревья в сетевых лесах
-Журнал "LAN" 01/2002
+Журнал "LAN" 01/2002
+-end{verbatim}
+
+\bibitem{Craig A. Schiller,Jim Binkley,David Harley,Gadi Evron,Tony Bradley,Carsten Willems,Michael Cross}
+Botnets - the Killer Web App (2007)- ничего особенного, но хороша тем, что есть обзор CC и общеизвестных альтернативных способов. Также есть обзор имевшихся на 2007 ботов. Также будет полезна описанием методов работы anomaly detection tools ( в частности ourmon tcp wight \& other). Также рассматриваются сандбоксы для исследования малвари.
+\begin{verbatim}
+Craig A. Schiller
+Jim Binkley
+David Harley
+Gadi Evron
+Tony Bradley
+Carsten Willems
+Michael Cross
+
+Botnets: The Killer Web App
+Syngress Publishing Inc, 2007
+-end{verbatim}
+
+\bibitem{Peter Szor}
+<<THE ART OF COMPUTER VIRUS RESEARCH AND DEFENSE>> - неплохой обзор технологий используемых для
+обнаружения вирусного и прочего зловредного кода и трюков используемых для того чтобы избежать этого в главе <<Armored Viruses>>.
+\begin{verbatim}
+THE ART OF COMPUTER VIRUS RESEARCH AND DEFENSE
+  By Peter Szor
+
+  Publisher: Addison Wesley Professional
+  Pub Date: February 03, 2005
+  ISBN: 0-321-30454-3
+  Pages: 744
 -end{verbatim}

 \end{thebibliography}
\end{verbatim}
