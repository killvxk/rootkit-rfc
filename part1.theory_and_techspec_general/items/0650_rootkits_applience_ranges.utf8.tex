\section{Ограничения применения rootkits} \label{applience_limits}

В этой главе рассматриваются ограничения накладываемые на rootkit,
как ограничения естественного характера(сеть, ПО, железо), так и 
ограничения связанные с средой исполнения (маскировка) .

\subsubsection{Естественные ограничения}

RootKit не панацея, и за кофе бегать не умеет..\\

Очевидно, что RootKit не сможет:

\begin{itemize}

\item{дать информацию о действиях пользователя не соотносящихся с компьютером.}

\item{
 дать информацию которой на компьютере пользователя нет 
\footnote{
 доступ к удаленной
 информации не всегда сопровождается с передачей на него этой информации, наоборот, чаще
 всего, доступ осуществляется посредством некоторого интерфейса к удаленной системе(например
 web-interface'а), когда локально доступны лишь трудносопоставимые с информацией движения
 и клики мышью.
 }.
}

\item{
Дать больше информации в единицу времени, чем позволяют установленные для данного 
RootKit естественные или програмные ограничения \footnote{в простейшем случае - 
невозможно передать информацию быстро на медленном канале}
}

\end{itemize}



\subsubsection{Ограничения среды исполнения}

Есть ряд существенных ограничений, которые должны соблюдаться
с целью маскировки деятельности закладки, в особенности сетевой ее деятельности.

Очевидно, что из за того, что это будет очень легко заметить на непривычном поведении компьютера,
rootkit не следуюет использовать чтобы:

\begin{itemize}
\item{получить представление о действиях пользователя в <<режиме реального времени>>}
\item{получить трансляцию video/audio с мультимедиа-устройств зараженного компьютера в
режиме реального времени}
\item{получать в режиме реального времени снимки с экрана}
\item{получить полный дамп трафика проходящего через сетевые интерфейсы }
\end{itemize}

Хочется заметить, что практически все из перечисленного в ограничениях среды исполнения реализовать
можно, но делать этого не стоит, поскольку обнаружение последует очень быстро, ставя под угрозу всю
систему.\footnote{и дело не в том, что описанная клиент-серверная система как либо особенно уязвима
к обнаружениям - наоборот, прилагается масса усилий затруднить получение полной картины по одному пойманному rootkit, - дело в том, что принципы функционирования сети rootkits - маскировка, а значит 
неправильное использование = создание себе проблем.}





