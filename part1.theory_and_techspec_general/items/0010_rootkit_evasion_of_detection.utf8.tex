\section{Скрытие rootkit}
\label{section_rk_evasion}

\paragraph{обзор главы\\}

В этой главе будут рассмотрены различные варианты скрытия рукткитов от
локального и сетевого обнаружения, то есть обнаружения "локальными"
средствами (антивирусным ПО, anti-spyware, antiadware and so on) и
сетевыми средствами - на маршрутизаторах, прокси, IDS.\\

Задача противодействия honeypots является задачей комплексной, в ней
присутствуют элементы локального и сетевого характера, поэтому она
выделена в отдельную главу номер \ref{honeypot_section} (см. на
странице \pageref{honeypot_section_page} ).\\

Задача по сокрытию rootkit делится на самом деле, как минимум, на три
задачи:

\begin{itemize}
\item{Локальное скрытие присутствия rootkit.}
\item{Скрытие трафика генерируемого rootkit.}
\item{Скрытие полной функциональности закладки при попадании в honeypot}
\end{itemize}

\subsection{Локальное скрытие rootkit}

\subsubsection{Общие рекомендации}

\paragraph{алгоритмы, проверки, противодействие сетевой трассировке\\}

\paragraph{Со стороны управляющего сервера\\}
Для наиболее успешного сокрытия результатов проверки на honeypots и
прочее контролирующий сервер должен внешне одинаково реагировать как на
<<плохого>>, так и на <<хорошего>> клиента.\\

Алгоритм противодействия антивирусным продуктам не должен быть реализован их отключением.
Отключение работы HIPS/Antivirus может быть замечено самим пользователем.

Более того, в случае, если на передачу закладкой данных, согласно
используемому транспортному протоколу возможно выдавать ошибку - это
следует делать.\footnote{Это поможет откреститься от обвинений обмена
данными с закладкой}

При работе с управляющим сервером закладка должна проверяться на точное
соблюдение инструкций сервера. Например, закладке можно по приходу за
заданием отдавать команду прийти еще раз через определенный промежуток времени.
Это позволит выявить закладки в процессе реверсинга и отрабатывая их по отдельному
алгоритму скрыть действительный алгоритм работы ботнета и полный протокол обмена
с сервером.

\paragraph{Со стороны бота}

По возможности часть алгоритма работы закладки должна быть реализована
на сервере - это усложнит и в ряде случаев сделает практически невозможным реверсинг,
 в частности при возможности установления соединения на timeoute'ах можно
реализовать протокол проверки на отладку.

В случае если закладка модульная необходимо шифрование модулей, при этом необходимо
использовать отдельный ключ для каждого модуля. При этом сервер должен выдавать
только ту часть ключей, которой достаточно для выполнения текущего набора
задач. Так делается невозможным реверсинг неиспользуемых модулей. Возможна
реализация последовательного получения ключей с сервера по мере прохождения
 проверок.

Для усложнения реверсинга возможна реализация интерпретатора своего байт-кода.\footnote{при этом однако всегда есть смысл делать оценку затраты/эффективность. То есть затраты оправданы когда труд противодействующей стороны умножается от вашего в разы, а лучше десятки и сотни раз.}

\subparagraph{работа с файловой системой\\}

Желательно создать следующую схему:

\begin{enumerate}
\item{Мониторинг присутствия пользователя (движение мыши/нажатие на кнопки)
или присутствия screen saver'а.}

\item{В случае присутствия пользователя: встраивание своего кода в
исполняющийся процесс, на каждое обращение процесса к диску
<<навешивается>> дополнительное обращение к диску одного из thread'ов
rootkit'а;
}

\item{В случае работы screen-saver'а: равномерное обращение к диску,
прекращаемое сразу по прекращению  screen-saver'а.}
\end{enumerate}

Таким образом можно замаскировать работу с диском под работу самого
пользователя, либо под работу  screen-saver'а - такая маскировка полезна,
поскольку активная работа с диском может быть замечена пользователем компьютера
(светодиодная индикация работы с диском на многих ноутбуках сделана на виду
у пользователя).


\subsubsection{Сокрытие при загрузке с внешнего носителя}

Бороться с обнаружением при загрузке с внешнего носителя можно
следующим образом:

\begin{itemize}
\item{Сохранение в BIOS компьютера или комплектующих - пока проверкой этих областей проверяющее ПО не озаботилось.}
\item{Сохранение легитимных ссылок на места в сети, открытие которых вызовет загрузку дроппера и повторное заражение. То есть сам рукит перезагрузку не переживает.}
\end{itemize}

\paragraph{хранение руткита в BIOS\\}
Не имеет пока очень широкого распространения, однако известны случаи <<in the wild>>, так, в Тайване были
обнаружены компьютеры в firmware которых находилась программная закладка пытавшаяся отсылать некие данные по сети. К сожалению подробностей инцидента пока нет\footnote{Хороший повод для внесения комментариев
аудиторией - у текущего релизера документа, к сожалению, мало времени чтобы указывать конкретные линки в сети на те или иные инциденты, информация о которых получена общением с специалистами занятыми в области computer/network security. Так что, если Вы можете подтвердить информацию здесь и в других местах линками на ресурсы в сети - присылайте ваши дополнения.}. При хранении в BIOS необходимо использовать шифрование как можно
большей части закладки, поскольку рано или поздно антивирусные компании и другие заинтересованные организации и персоны начнут проверять BIOS, как минимум, сигнатурным или другим методом.

\paragraph{Методы которых следует избегать\\}
Если будет найдена уязвимость в проверяющем файловую систему ПО возможно хранение загрузчика спец. областях файловой системы с загрузкой по факту проверки FS при старте. Однако это легко отслеживается, если проверяется отличие всего диска, а не только файлов в рамках FS, к тому же данное поведение зависимо от наличия ошибки в
проверяющем ПО, то есть ненадежно - исправленная версия будет рано или поздно выпущена производителем и закладка <<попадется>>.


\subsubsection{Вариант 1: Отсутствие внешней маскировки}
\paragraph{Описание}

Под отсутствием маскировки можно понимать такое поведение, при котором работа с
файловой системой и реестром не прячется а исполнение реализуется так, что нет
отдельного процесса и т.о. закладка не отображается в стандартном мониторе процессов.
В т.ч. позволяется произвести деинсталляцию закладки (с возможными вариантами в духе
перехода на другой метод маскировки работы в системе). Маскируется только трафик работы
с сетью, причем создается также легитимный трафик, который пользователю позволяется разрешить
или запретить. В идеале закладка должна нести также некую минимальную полезную нагрузку.

\subparagraph{Реестр\\}

Этот метод состоит в том, что загружающая rootkit часть не прячется от
просмотра во время работы rootkit, также не прячутся временные файлы.
Таким образом при проверке чтения реестра разницы просто нет. Такой метод
работы подразумевает, что после инсталляции rootkit реестр не меняется вообще.
Это не может не создавать определённых неудобств программирующему rootkit.

\subparagraph{файловая система\\}

Временные данные rootkit возможно размещать в swap файле,
временных системных файлах которые изменяются при каждой
загрузке ОС. Вариантов достаточно много.

Сам rootkit размещается среди прочих драйверов, которых в
современных ОС более чем достаточно. Предусматривается
деинсталляция штатными для ОС средствами деинсталляции драйверов
с последующей отработкой какой нибудь подпрограммы на это
<<недружественное>> действие.

\paragraph{Плюсы\\}

\begin{itemize}
\item{При правильной реализации должно быть очень похоже на какой нибудь обычный системный драйвер.}
\item{Возможно совместить с методом сокрытия от програмных мониторов контрольных сумм}
\end{itemize}

\paragraph{Минусы\\}

\begin{itemize}
\item{Сложность в программировании.}
\end{itemize}


\subsubsection {Скрытие от мониторов контрольных сумм}

Основой возможности скрытия от мониторов контрольных сумм служит тот
факт, что современные компьютерные системы весьма динамичны. А именно
подавляющее большинство пользователей инсталлирует программное
обеспечение, которое совершенно легально изменяет реестр и инсталлирует
свои файлы, иногда в системные каталоги.  При последующем  запуске
монитора контрольных сумм объёмы информации об изменениях произведённых
в системе может оказаться настолько большим, что пользователь, или
системный администратор, контролирующий данный ПК вынужден будет
просматривать имеющиеся изменения весьма поверхностно. Также весьма
характерно для подобных мониторов то, что  администратор  монитора
вынужден делать исключения для определённых несущественных файлов и
каталогов, в противном случае он просто утонет в объёмах совершенно
неинтересных изменений в, например, кэше internet explorer или swap
file'е.

Ещё один характерный нюанс современной сетевой инфраструктуры - привязка
ОС к обновлениям. Практически все современные системы подключённые к
интернет по каналам ADSL пользуются обновлениями. Например MS Windows
based системы пользуются сервисом windows update . Не будет
преувеличением сказать, что данным сервисом пользуются и многие
пользователи подключённые по менее высокоскоростным каналам.

Основная идея - сделать инсталляцию root-kit максимально похожей на
часть инсталляции стороннего ПО или процедуры обновления системы.

Основной нюанс этого метода в том, чтобы не модифицировать исполняемые и
прочие системные файлы вообще, либо стараться делать это как можно реже,
в идеале - только один раз, во время инсталляции, причем инсталляции
совмещённой с инсталляцией другого ПО.

Результатом такой методики при правильной реализации должны быть весьма
 успешны.

\paragraph{Плюсы\\}

\begin{itemize}
\item{При правильной реализации данные хранимые rootkit должны
 попадать в список исключений из мониторинга.}
\item{Возможно совместить с методом сокрытия от проверки с внешнего носителя}
\end{itemize}

\paragraph{Минусы\\}

\begin{itemize}
\item{Сложность в программировании.}
\item{Требует довольно кропотливой и длительной пользовательской работы с
установкой различных мониторов контрольных сумм и прогонкой в различных
вариантах возможных конфигураций доступных для установки мониторов}
\end{itemize}


\subsubsection{Маскировка под известные закладки.}
\paragraph{Описание\\}
Такой метод базируется на отказе от скрытия факта различий между
<<offline>> и  <<online>> реестром, т.е. покуда ОС загружена - ключ
реестра запускающий rootkit скрывается (либо вместо имеющегося отдается
пустой ключ, либо выдается инф-я о его отсутствии вообще). В то же время
загружающий rootkit бинарь на который указывает запись реестра делается
похожим на какой нибудь известный относительно безобидный вирус. На мой
взгляд этот метод не заслуживает существенного внимания, хотя может
применяться, когда необходимо создать у пользователя представление о
том, что утечка данных произошла через вирусное ПО или же, что он словил
относительно безобидное adware.

\paragraph{Плюсы\\}

Если пользователь таки обнаружит что-то, то, много шансов, что он
успокоится на обнаружении и удалении безопасного вируса.

\paragraph{Минусы\\}

Очевидно, что если rootkit не имеет альтернативных способов
<<реинкарнации>>, то его присутствие на обнаружении загружающего модуля
будет завершено с  перезагрузкой машины.

\subsubsection{Сокрытие при попытке обнаружения несоответствий}

Основоной метод борьбы в данном случае - устранение несоответствий, что
может быть достигнуто только при перехвате функций на уровне ядра (а не
только на уровне hook'ов <<обёрток>> к вызову ядерных функций) и модификации
внутренних структур данных с которыми оперируют функции ядра.


\subsection{Сетевое скрытие rootkit}

По итогам нескольких выставок современных достижений в области защиты
инф-ии я делаю следующие выводы:  преимущество (по эффективности борьбы
с вредоносным ПО) у технологий распознающих последствия, например
увеличение трафика, нехарактерный пользователю трафик и т.п. (т.е. в
аналогичные  несоответствия типичному поведению можно искать и в
поведении исполняемого кода на самом компьютере - жертве). Характерным
для всех продуктов реализующих защиту от заранее неизвестных проблем
является довольно большой порог срабатывания, так как поведение
пользователя в различные дни может колебаться. По моей оценке двух/пяти
процентные несоответствия должны быть незаметны на общем фоне в
большинстве систем. Таким образом медленные атаки (т.е. сцеживание
информации в час по чайной ложке) имеют гораздо больший шанс остаться
незамеченными, чем любые другие. Вообще говоря - чем  ближе активность
закладки к активности самого пользователя, тем больше шансов не вылезти
за рамки типичного поведения. Огрублённые оценки <<темплейта работы
пользователя>> на мой взгляд реализуемы достаточно просто. Фактически
речь идёт о наборе правил, на основании которых должна регулироваться
работа с сетью (трафик), причем набор  правил желательно формировать динамически
на основе наблюдения за пользователем.

\subsubsection{Доступ к сетевым дискам}

При работе с сетью в крупных учреждениях вероятна ситуация, когда доступ
к определённым ресурсам возможен только из <<представившейся>> программы
(процесса прошедшего аутентификацию).

Желательно создать следующую схему:

\begin{enumerate}
\item{Мониторинг обращения к сетевым дискам.}
\item{Встраивание своего кода в исполняющийся процесс, который обращается к сетевым ресурсам}
\item{Обращение к сетевым дискам <<от имени>> процесса, который к ним уже обращался}
\end{enumerate}

Таким образом можно обойти возможно-существующие ACL разрешающие
обращение к сетевому диску только из одной программы. В принципе ту же
практику можно применить по отношению к локальным дискам, однако
необходимостью для локальной работы она станет только в случае
установленных программ реализующих криптодиски и установленных программ
мониторов работы приложений - второй случай относительно редко, но может
встретиться.


\subsubsection{Сокрытие трафика}

Имеют место следующие рассуждения:

\begin{itemize}
\item{на основании того, что трафик может быть получен для анализа сетевыми средствами следует вывод что трафик rootkkit'а надо зашифровывать}
\item{на основании того, что нестандартные порты и протоколы могут вызвать подозрения у администратора просматривающего статистику, так и потому, что это однозначно вызовет срабатывание тревожного триггера на любой IDS такими приёмами пользоваться не следует - при возможности обмена с внешним миром по стандартным протоколам следует использовать именно эти протоколы.}
\item{на основании того, что IDS наверняка сгенерирует тревожное событие (<<алерт>>) на несоответствие портов протоколу не следует генерировать подобный трафик}
\item{На основании того, что существуют стандартные средства анализа трафика на правильную асимметрию следует генерировать трафик с правильной асимметрией (пояснения ниже).}
\item{На основании того, что сама по себе передача зашифрованных данных уже повод задуматься - канал передачи данных необходимо прятать (см. выше о тунелировании в картинках на  стр. \pageref{hidden_tunnel}).}
\item{на основании того, что за закладку могут взяться всерьёз следует встраивать в неё средства обнаружения(но не защиты от!) трассировки}
\end{itemize}

\paragraph{Удовлетворение требований к асимметрии трафика\\}
\index{асимметрия трафика}
Вкратце: для того чтобы удовлетворить нормальной статистике по
количеству принятых и переданных  байт для протокола http закладка
должна эмулировать поведение пользователя. То есть она должна скачивать
что то, поскольку чаще всего по http происходит именно download, а не
upload файлов. Наиболее подходящим для скачивания являются картинки.
Помимо прочего можно, усложнив алгоритм, приблизить похожесть контента к
виду нормальной http страницы, то есть некий случайный безобидный текст
плюс картинки. То есть иными словами закладка должна уметь серфить сеть.
При этом характер допустимой работы с сетью можно получить мониторингом
сетевой активности самого пользователя.

Вообще говоря асимметрия трафика актуальна только для тунелирования
в протоколах не использующих шифрование. Внутри ssl заботиться об асимметрии нет смысла.

\paragraph{Шифрование трафика\\}

Требования к алгоритму шифрования просты: скорость, минимальный размер.
Выбор остается за разработчиком соответствующего rootkit, могу лишь
сказать, что заслуживает внимания idea всвязи с
высокой скоростью его реализации на
mmx-совместимых процессорах - таких сейчас большинство.
Кроме того возможно применение разных алгоритмов на различных данных, т.е. если актуальность
данных для анализа закладки/ботнета в случае их расшифровки ничтожно мала - можно применять на таких
данных банальный xor. Разумеется payload должен быть зашифрован с использовавнием серьезных алгоритмов
шифрования.

\paragraph{Получение payload через скрытый канал\\}
payload передается закладке в зашифрованном виде. Ключ к расшифровке может быть
передан по условию (например после проверки на sanbox/honeypot и прочих проверок).
Желательна передача по каналам организованным согласно рекомендациям в \ref{hidden_tunnel} на стр. \pageref{hidden_tunnel}, при этом необходимо чтобы получить их с сервера иначе невозможно было невозможно.\\
Необходимо исключить возможность записи на диск расшифрованного payload
и ключа расшифрования.

\paragraph{Скрытие канала передачи данных\\}
\label{hidden_tunnel}
\index{скрытие канала передачи данных}
Одним из возможных способов для организации скрытого канала передачи данных закладке
является прием модифицированных картинок. Так, например, в стандарте на формат jpeg
определяются поля, которые могут использоваться только специальным
ориентированным на графику софтом (как то adobe photoshop, gimp и
подобные специальные редакторы графики), причем некоторые - только после
того как будут установлены соответствующие галочки в настройках этих
программных продуктов(по крайней мере photoshop ведет себя именно так).
Другой же софт, в том числе \underline{любой} браузер, проигнорирует эти
поля.\\

Еще один момент, который стоит учитывать - передача информации наружу может,
но не должна быть организована через картинки, поскольку upload файлов на сервер
по http происходит весьма редко. Для отправки данных на сервер можно использовать
запросы с зашифрованными данными внутри запроса (например POST). Чтобы это было
незаметно на фоне остального - подобные запросы, но со случайными данными должны
идти на случайные сервера.

Для того чтобы наличие шифротекста было тяжело отличить от просто
измененной картинки соответствующее поле в картинке должно
инициализироваться случайной строкой, тогда не зная алгоритма (то есть
не дизассемблировав закладку) невозможно будет отделить мусор от значащих данных.

Однако, следует понимать, что очистка картинок от таких дополнительных
полей легко автоматизируется (например есть утилита jpegclean). В то же
время существуют утилиты которые прячут данные в поля цветности картинки
без заметного глазу ухудшения качества. Выбор конкретной реализации
скрытого канала остается за реализатором закладки. По крайней мере очистка
jpeg от информации в специальных полях не является сейчас стандартным пунктом
в настройке прокси систем.

Также при организации тунелирования через картинки необходимо использовать картинки
с статистически часто попадающимися параметрами (размер, разрешение, цветность и т.д.).