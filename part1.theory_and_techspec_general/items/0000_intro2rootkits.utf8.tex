\begin{flushleft}
\ldots\\
Уходят волки в оптике прицела..\\
И все про все - твой выстрел на удачу..\\
\ldots\\
группа Би 2.\\
\end{flushleft}

\begin{flushright}
\ldots\\
Игра ума.. \\
Кончается расстрелом.\\
И здесь и там..\\
Все та же волчья стая.\\
\ldots\\
группа Би 2.\\
\end{flushright}


\section{Обзор статьи.}
\label{section_paper_overview}

\subsection{Насколько это законно?}
В разных странах существуют различные законодательные ограничения вплоть до уголовного преследования
разработки програмного обеспечения вредоносного характера. Данная статья не содержит програмного кода и
является аналитическим обзором, таким образом ее развитие не подлежит преследованию, по крайней мере,
 согласно законодательству РФ. Если Вы считаете иначе - присылайте maintainer'у
Ваши комментарии с ссылками на соответствующие статьи законодательства. 

\subsection{Использование.}

Изначально статья распространялась исключительно в узком кругу заинтересованных
лиц - специалистов IT, в особенности среди тех из них, чья работа связана с обеспечением
безопасности IT инфраструктуры, либо с ее тестированием.\\

Теперь же статья выложена в public domain в надежде на поддержку сообщества.\\

Email текущего релизера документа в рамках NetHack club: grey-olli@ya.ru,\\ PGP ключ доступен к поиску в интернет: gpg --search-keys grey\_olli , сервер ключей: hkp://keys.gnupg.net \\


Подразумевается использование в качестве ознакомительного материала для:

\begin{enumerate}

\item{Управляющего персонала среднего и высшего звена (главы: \ref{section_paper_overview},
 \ref{section_rk_payloads} ) }

\item{Персонала ответственного за безопасность сетевой
( internet(IPv4/IPv6), ЛВС (ethernet / wifi LANs / Corporate WANs) ) инфраструктуры.
(особое внимание главам \ref{section_rk_detection} и \ref{section_rk_evasion}).
}

\item{Заказчика практической реализации описываемых в данной статье принципов - стоит
 просмотреть все уделив особое внимание главам
\ref{section_rk_detection} (со стр. \pageref{section_rk_detection}),
\ref{honeypot_section} (со стр. \pageref{honeypot_section}),
\ref{section_rk_payloads} (со стр. \pageref{section_rk_payloads}) .}

\item{программиста или группы программистов, реализующих закладку класса <<rootkit>>
для сетей общего пользования, в частности, интернет (особенное внимание главе \ref{tech_spec} ).}

\item{Любых заинтересованных лиц - как
 обзорный материал по возможным способам скрытия и обнаружения закладок,
использующих сети общего доступа, в частности, интернет (если нет
конкретной цели и достаточно времени, то лучше просмотреть все ).}

\end{enumerate}

Разумеется, если Вы заинтересовались, есть смысл прочесть все.

\paragraph{Глоссарий\\}
Статья содержит глоссарий терминов,  составленный в рассчете на человека
мало знакомого с предметом, в том числе с интернет. Однако, в нем
присутствуют и термины, которые могут быть новыми и для некоторых
IT-специалистов, особенно <<прикладников>>. Так что пролистать глоссарий
(\ref{glossary} стр. \pageref{glossary} - \pageref{glossary_end})
рекомендуем всем. Для удобства сделан предметный указатель страниц
по встречающимся терминам (в основном в глоссарии).

\subsubsection{Порядок чтения}
\label{reading_notice}
В случае если Вы не знакомы с предметом, или не уверены, что знакомы
достаточно хорошо, имеет смысл начать с просмотра глоссария.\\

\paragraph{Ответственным за сетевую безопасность} желательно просмотреть все, уделив
особенное внимание обзору в главах \ref{section_rk_detection} и \ref{section_rk_evasion}\\

\paragraph{Техническим специалистам} рекомендуется уделить внимание главе \ref{tech_spec}
(стр. \pageref{tech_spec} - \pageref{tech_spec_end}), все остальное, если не возникает
вопросов по аргументации, можно лишь бегло просмотреть.

\subsubsection{Замечания по главам}

\paragraph{ Глава \ref{tech_spec} } (стр. \pageref{tech_spec} - \pageref{tech_spec_end})
выделена в отдельный документ, предназначенный только для технических специалистов.
Скорее всего у Вас имеются обе части данного RFC.

\paragraph{ Глава \ref{applience_limits}} (стр. \pageref{applience_limits}) дает
краткий обзор ограничений присущих rootkits.

\paragraph{ Глава \ref{section_paper_workflow}} (стр. \pageref{section_paper_workflow})
может быть интерсна только тем, кто участвует в развитии документа, она полностью посвещена редакторской правке и ведению версий документа, по смысловой нагрузке статьи в ней мало интересного,
однако, возможно стоит взглянуть на \ref{miss_of_attention} на стр. \pageref{miss_of_attention} - там есть ответы на кое-какие комментарии людей невнимательно читавших часть документа. Также \ref{wonnt_fix} на стр.
\pageref{wonnt_fix} описывает то, что предлагалось поменять, но релизер счел изменения не стоящими усилий.

\subsection{Цель}
%
%\sl - наклонный
%\bf - bold (полужирный)
%
{\sl\bf
 Данная статья ставит своей целью описание рекомендаций по написанию
rootkits. В идеале - описание минимально необходимых требований к
качественному rootkit, то есть документ класса RFC.
}

{\slДля разработки качественного ПО, требуются соответствующие подходы
 - выработка требований, построение архитектуры, планирование, а не
 кодинг на коленке сломя голову. Это все применимо и к руткитам, и
даже в большей степени - из-за специфичности разработки - код работающий
с ядром ОС (существенная часть кода rootkit) требует внимательости и продуманности.}\\

Данная статья развивается чтобы заполнить нишу в части описания правильной архитектуры и
выработки требований к реализации.


\paragraph{Сделано:\\}
Рассматриваются методы обнаружения (вкратце) и методы скрытия (в общих
чертах), модели угроз для закладки, управляющих серверов и ботнета в целом.
Примеры и описания утилит пока под только под Windows. Декларируется
минимально необходимый набор требований к реализации качественного rootkit:
наборы требований к протоколам, алгоритмам, языкам реализации, наборы требований при реализации
клиент-серверной (не peer2peer) модели взаимодействия.

\paragraph{В планах:\\}
см. \pageref{paper_2do_1}, \pageref{paper_2do_2} .

\paragraph{Что такое rootkit\\}

rootkit - <<самый продвинутый>> вариант реализации программных закладок.
Для того чтобы удовлетворять классу rootkit, закладка должна
реализовывать невидимость своего присутствия - как для средств
обнаружения программ включённых в комплект ОС, так и для утилит
сторонних производителей. Такой уровень невидимости достигается путём
перехвата внутренних функций ОС на уровне ядра\footnote{самый низкий,
или <<самый внутренний>> уровень работы ОС}, т.е. перехватом функций к
которым обращаются как процессы прикладного режима, так и функции ОС.
Существуют различные подходы к классификации руткитов, они будут вкратце рассмотрены
далее.\footnote{в частности, встречается понятие 'user-mode' rootkit. С учетом
развития современных средств противодействия нежелательному ПО такие rootkit'ы,
по большому счету, за rootkit'ы считать уже нельзя.}


\paragraph{Copyright\\}

Текущий \copyright - NetHack club .
Первоначальный вариант статьи был написан в 2006 году по мотивам обсуждения развития
одного из спамерских ботнетов за 2005й-2006й годы. В мае 2009 статья перешла к NetHack
club в team work, после чего прошла существенную правку - технологии сильно
ушли вперед. После 2009го года статья не обновлялась.

\paragraph{Спасибо\\}

Хочется сказать спасибо:

\begin{itemize}
\item{всем представителям компьютерномого undeground'а вообще.}
\item{virii/coding сцене.}
\item{Техническим специалистам, предоставившим свои комментарии к букве и сути данной работы.}
\end{itemize}
