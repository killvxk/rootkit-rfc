\section{Набор свойств необходимых к реализации в качественном rootkit}
\label{tech_spec}

Тут только анонс - краткое содержание (обзор) - второй части данного rfc, предназначенной только
для технических специалистов. Вторая часть данного rfc распространяется на анологичных условиях
 отдельным документом, скорее всего она у Вас есть, если же ее у Вас нет и Вы чуствуете в себе
силы принять участие в обсуждении технических тонкостей - .пишите текущему релизеру документа.

%
{\sl\bf
  Внимание: эта глава развиваться не будет.
  Детальная техническая разработка требований к качественному rootkit
  вынесена в отдельную статью, которая является логическим продолжением
  данной. Это сделано, в том числе, для облегчения развития и ведения версий
  данного RFC.
}

\paragraph{Обзор второй части rfc\\}

Во второй части техническим языком без аргументации (она как раз тут, в первой части):

\begin{enumerate}
\item{preface+editorial}
\item{требования к языкам программирования}
\item{требования к архитектуре}
\item{требования к протоколу}
\end{enumerate}

Все это поделено сервер/закладка/степень важности/прочие параметры.
