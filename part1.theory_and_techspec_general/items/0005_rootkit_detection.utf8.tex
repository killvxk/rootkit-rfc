\section{Обнаружение rootkit}
\label{section_rk_detection}
\subsection{Классификация обнаружения}

Поскольку современные руткиты это модульное ПО - обнаружение можно классифицировать по месту:
\begin{itemize}
\item{на зараженном компьютере}
\item{в сети}
\end{itemize}

и по модулю обнаружение которого требуется осуществить:
\begin{itemize}
\item{закладка}
\item{дроппер}
\item{управляющий сервер}
\end{itemize}

\paragraph{Обнаружение управляющих серверов} происходит исключительно по выявлению адресов и url к которым подключаются закоадки.
В основном происходит на хонипотах.

\paragraph{Обнаружение дропперов} сводится к выявлению хостящих серверов (купленных либо зараженных),
скачиванию тела и созданию сигнатур для определения противодействующим локальным и сетевым ПО -
антивирусами, сетевыми сенгсорами. Происходит, в основном, либо на активных хонипотах, либо на поисковых
серверах - например google предлагает фильтрацию найденного ссылдочного контента по степени опасности.

\paragraph{Обнаружение закладок} может быть либо сетевым либо локальным.
Под сетевым обнаружением понимается и активное и пассивное сетевое
обнаружение.

\subparagraph{Активное сетевое обнаружение закладок}\index{активное
сетевое обнаружение} это либо сканирование
компьютеров подключённых к защищаемой сети на предмет несоответсвия
открытых портов используемым на них сервисам, либо определение
нестандартной реакции на проходящий мимо тестируемых компьютеров
трафик\footnote{классическое определение глупого сниффера - по
взаимодействию с адресами  из трафика не относящегося к данному
компьютеру, хотя и доступному на нем, например по  resolving'у таких
адресов}.

\subparagraph{Пассивное сетевое обнаружение закладок}
\index{пассивное сетевое обнаружение} подразумевает под
 собой анализ трафика в сети. Либо на прокси, маршрутизаторах и
коммутаторах по статистическим аномалиям и сигнатурам(шаблонам), либо на IDS
(статистические аномалии плюс сигнатуры на трафик определённого типа).
HoneyPots  также используют пассивное определение по трафику, по крайней мере, на
первоначальном этапе.

\paragraph{Обзор\\}
\index{локальное обнаружение}
Вариантов локального обнаружения по большому счету два. Первый -
обнаружение инструкций для загрузки кода\footnote{например ключей реестра и т.п.}
и прочих изменений после загрузки с внешнего к системе носителя, например CDROM,
так можно провести, например, сравнение чтения реестров - <<offline>> и <<online>>.
Второй - обнаружения несоответствий результатов вызова тех или иных функций API
на уровне пользователя (ring3) и на уровне ядра (ring0). Возможен также вариант с
опросом через различные службы в пределах ring3 на предмет неких данных о системе
и сравнение полученного, однако это ненадёжно.

\subsection{Локальное обнаружение rootkit}
\index{локальное обнаружение}

\subsubsection{Используемое готовое ПО}

На данный момент рассматриваются только Windows ситемы. Желающие дополнить unix-like аналогами - присылайте краткие описания релизеру статьи.

\paragraph{RKDetect\\}
RKDetect - утилита обнаружения Windows rootkit по поведению.

\subparagraph{Общие принципы работы\\}
RKdetect\index{RKdetect} - небольшая утилита основанная на обнаружении  отклонений,
которая позволяет обнаруживать службы  скрытые распространенными
руткитами типа Hacker Defender. Утилита чрезвычайно проста - она
перечисляет службы  удалённого компьютера через WMI (на пользовательском
уровне) \footnote{Windows Management Instrumentation (WMI) - интерфейс,
 обеспечивающий взаимодействие с компонентами системы, в общем случае
доступными лишь через особые механизмы. WMI можно использовать в
различных целях, в частности для управления компьютерами с помощью
сценариев.} и через Service Control Manager (приложение работает на уровне пользователя - services.exe), сравнивает результаты и отображает различия. Таким образом находятся скрытые
службы, используемые обычно для запуска rootkit. Такой же подход может быть использован
для обнаружения процессов, разделов реестра и всего, что может скрыть rootkit.

\paragraph{RootkitRevealer\\}
Утилита для обнаружения руткитов от 2006го года, <<лечить>> не умеет. Сравнивает контент
 файловой системы и реестра полученный чтением на низком уровне и на уровне windows API.
На 2007 год в среде продвинутых писателей rootkit считался неактуальным старьем. Легко определяет
только руткиты доступные полностью или частично в исходных кодах в интернет на 2006й год - AFX,
Vanquish и HackerDefender .
\begin{verbatim}
Скачать можно тут:
http://technet.microsoft.com/en-us/sysinternals/bb897445.aspx
\end{verbatim}

\paragraph{Антивирусный монитор Adinf\\}
\index{Adinf}
Утилита устанавливается в проверяемую систему сразу же после её
инсталляции и создает контрольные суммы файлов имеющихся на диске
(конфигурабельно). В дальнейшем при запусках  контрольные суммы
сравниваются с оригинальными. Умеет обнаруживать некоторый набор
вирусов, умеет передавать измененные файлы в качестве параметров для
проверки антивирусным ПО.

\paragraph{Tripwire\\}
\index{Tripwire}
Утилита устанавливается на проверяемую систему сразу же после её
инсталляции  и создает контрольные суммы файлов имеющихся на диске
(конфигурабельно). В дальнейшем при запусках  контрольные суммы
сравниваются с оригинальными.

\paragraph{AVZ,GMER}
\label{antirootkittool}
\index{AVZ}
\index{GMER}
Наиболее серьезные из известных нам утилит специально нацеленных на обнаружение rootkit, активно развиваются\footnote{на лето 2009го года} авторами.

\paragraph{Антивирусные мониторы DrWEB, Kaspersky, NortonAntivirus, Panda antivirus и другие.\\}
\label{antiviruses}
\index{DrWEB}
\index{Kaspersky}
\index{NortonAntivirus}
\index{Panda}
На 2006-2007й годы всерьёз почти не рассматривались как антируткит-средства,
поскольку 99.(9) их функциональности состояло в обнаружении известных вирусов,
троянцев, червей и прочего деструктивного ПО. Обнаружение новых закладок у всех
антивирусов отнюдь не на лучшем уровне - максимум, что может сказать
антивирусный продукт - <<подозрительно>>.\\

Тем не менее, в 2009м антируткит-технологии различного качества предоставляет большинство
антивирусов. Например, на конец 2007го - KAV детектит скрытые процессы как руткит, DrWeb
стал детектить и лечить новый MBR Rootkit. Качество их работы требует отдельного рассмотрения.
Поскольку цель данной главы лишь обзор средств борьбы с руткитами - антивирусные продукты
упомянуты весьма поверхностно.\\

Также отдельного рассмотрения заслуживает проактивная защита рекламируемая многими производителями
антивирусов.\\

О методах противодействия классическим сигнатурным анализаторам можно сказать <<в двух словах>> следующее:

\begin{itemize}
\item{использование алгоритмов полиморфизма}
\item{шифрование тела закладки}
\end{itemize}

\index{полиморфизм}
Кстати, наиболее правильным является полиморфизм реализуемый при компиляции, когда
антивирусные компании не получают для последующего реверсинга алгоритм
полиморфизма и, таким образом, не способны выпустить версию антивируса,
который будет ловить следующий <<морф>> rootkit'а и его дроппера.

\subsection{Сетевое обнаружение rootkit}

Обнаружение любых закладок и вредоносного кода использующего возможности сети достаточно
просто, если не используются скрытые каналы передачи данных с маскировкой паразитного
трафика под легитимный. При этом сетевое обнаружение можно разделить на две
категории:\\

\begin{itemize}
\item {обнаружение локальное - средствами  Personal Firewall, как то: Kerio, Outpost и др., а также интегрированными в antivirus-software продукты программными модулями, предоставляющими функциональность personal firewall в качестве одной из опций (например  в комплекте у Norton, Panda, Kaspersky), функционал аналогичный Personal Firewall имеется у HIPS (host intrusion prevention systems), например Cisco Security Agent\index{Cisco Security Agent}}
\item{обнаружение на транзитном сетевом оборудовании.}
\end{itemize}

\index{firewall}
\index{personal firewall}

\subsubsection{Обнаружение на локальных firewall'ах и HIPS}

Доступно любому пользователю ПК, однако ненадёжно, поскольку использует
перехват обращений к функциям реализующим передачу данных по сети, то
есть, если rootkit осуществил перехват раньше personal firewall, то
весь трафик rootkit'а идёт мимо firewall (не блокируется, не
замечается) . В идеале очередность загрузки значения не имеет - руткит
за счет нестандартных приемов скрытия активности (в т.ч. сетевой) может
оставить персональный firewall <<в дураках>>.

Существуют закладки, в которых реализована "необнаружимость"
персональными firewall'ами любых производителей.\footnote{Это было абсолютно
справедливо в 2006м году - на подобных закладках работал достаточно приличный по
объёмам ботнет, rootkit-часть которого тестировалась на большинстве personal firewalls
того времени - достоверные данные <<из первых рук>>. В 2009м это утверждение можно
 считать спорным (особенно с учетом появления комплексных систем с использованием
 технологии гипервизоров), однако де факто борьба firewall vs malware в этом
контексте - классическая вечная борьба щита и меча - после появления новых способов
 взлома со временем появляются способы защиты от них.}

Обнаружение средствами HIPS более вероятно, так как HIPS чаще всего включает в себя функционал
локального firewall'а, но не ограничивается им. Чаще всего HIPS поставляется с некоторым набором
правил и средствами их управления и дополнения. В числе прочего HIPS мониторит обращения к файловым
системам, загрузку приложений и драйверов, возможные переполнения буфферов приложений.

Примеры HIPS: System Safety Monitor, AntiHook, CyberHawk,
DefenceWall,Sandboxie, Cisco Security Agent.

\subsubsection{Обнаружение на транзитных сетевых устройствах}

В общем случае обнаружение активности rootkits на транзитных устройствах
сводится к  трем вариантам:

\begin{itemize}
\item{обнаружение асимметрии трафика}
\item{обнаружение аномальной активности (нетипичные объёмы трафика, порты, протоколы)}
\item{обнаружение на основе сигнатур IDS}
\end{itemize}


\paragraph{Асимметрия трафика\\}
\index{асимметрия трафика}
Протоколы используемые для работы с сетью имеют статистические
характеристики доступные на точках через которые проходит трафик.
Например, http протоколу характерна нормальная асимметрия, когда
пользователь получает данных больше, чем отправляет\footnote{фактически
трафик пользователя это запросы, а ответный трафик (контент запрошенный
сервера) обычно существенно больше по объёму данных}. Существуют
анализаторы асимметрии трафика, способные сигнализировать нетипичную
статистику обмена данными по различным протоколам. Помимо этого возможно
 определение типа туннелируемого трафика и для зашифрованных туннелей по
статистическим характеристикам, так, некоторые современные IDS способны
по статистическим характеристикам передаваемых данных определить, например,
что внутри ssh туннеля идет http трафик. Фильтрация peer2peer сетей и приоритезация
peer2peer трафика на транзитном оборудовании в ISP построена в т.ч. на этом принципе.

\paragraph{маршрутизаторы\\}
\index{маршрутизаторы}
 Часто совмещают в себе задачи firewalling'а . Для обнаружения может
быть использована возможность снятия статистики с устройства. Причем,
поскольку трафик в интернет вещь платная, к анализу трафика во многих
организациях подходят отнюдь не спустя рукава. Любой мало-мальски
серьёзный маршрутизатор способен выдать статистику программе-агрегатору,
которая, в свою очередь, способна выводить различные репрезентационные
графики с выборкой за различные периоды времени (час, день, месяц).
Такое представление данных весьма наглядно и позволяет узнать о
установленных закладках по нетипичному трафику (нестандартные порты,
протоколы не используемые в данной организации), либо по пикам трафика
определённого типа, а также по графикам расходования трафика .
Для примера можно указать на маршрутизаторы компании cisco systems и
пакет netflow tools. Аналоги программных комплектов визуализаторов
трафика есть для любых производителей маршрутизаторов. Даже в случае,
если маршрутизатором является обычный ПК - у не слишком ленивого
администратора есть все необходимое чтобы добиться такой же наглядности
 представления трафика, как и при использовании маршрутизаторов той же
 cisco. \\

Обнаружение нетипичного поведения на основе трафика весьма опасно для
владельцев rootkit и может послужить поводом для дальнейших
разбирательств, что рано или поздно приведёт к обнаружению и, в, худшем
для владельцев, rootkit'а случае, реверсингу алгоритмов работы,
инсталляции и протоколов передачи информации . Это может привести
к выходу на владельцев, вплоть до выявления личности с вытекающими
из этого судебными преследованиями.

\paragraph{прокси\\}
\index{прокси}
\index{proxy}
Часть сетей (значительная часть сетей организаций) предоставляют доступ
к web ресурсам (и, иногда, к другим ресурсам) через proxy. Прокси
могут, как и маршрутизаторы, быть совмещены с программным комплексом по
визуализации трафика, то есть возможности обнаружения в общих чертах те
же, что и у маршрутизаторов. Кроме этого прокси обладают возможностью
хранения трафика проксируемого протокола, избирательной блокировки
ресурсов доступных в сети по проксируемому протоколу, а также позволяют
реализовать запуск программы по факту доступа к тому или иному ресурсу
доступному через проксируемый протокол и, разумеется фильтрации и вызова
скрипта по факту совпадения трафика с шаблоном какой либо атаки. То есть
потенциально более опасны для владельцев rootkit.\\
Существуют решения <<Web Application Firewall>>, основная их задача -
фильтрация атак на web-приложения по прокси-принципу (защищается web-сервер
перед которым ставится такая proxy система).

\paragraph{IDS и IPS\\}
\index{IDS}
\index{IPS}

Основных отличиий IDS от IPS  два:\\

Во первых, место установка в сети: IDS прослушивает трафик, а
IPS пропускает трафик через себя, что позволяет на IPS использовать правила фильтрации
для блокирования атак и изоляции атакующего. Преимуществом IDS является то, что он не
вносит задержки в сеть. Недостатком IDS является меньшее количество возможных реакций -
их всего два - посылка пакетов TCP RST и запуск некоего скрипта, который уже может
обрабатывать событие по различным сценариям (это заведомо медленнее реакции IPS).

Во вторых - скорость реакции - IPS реагируют <<real time>>, IDS с некоторой задержкой.\\

IDS и(или) IPS устанавливаются в любой организации всерьёз относящейся к собственной
безопасности. Представляют из себя мониторы трафика, <<заточенные>> под
выявление неправомерных действий в сети, как то:

\begin{itemize}
\item{атак извне сети}
\item{подозрительной сетевой активности}
\item{трафика генерируемого известными закладками}
\item{запрещённых типов трафика(по портам, протоколам, содержимому)}
\end{itemize}

Классические примеры:\\
snort\index{snort} (доступен всем, бесплатен\footnote{плюс доп. поддержка за деньги}),
cisco IDS (дорого, но очень производительно, доступно только организациям
способным потратить порядка 100 тыс долларов и затем тратить деньги на
обновления сигнатур),Ourmon\index{Ourmon}(доступен всем, бесплатен, специально нацелен на
противодействие ботнетам).\\
Так же широко известны:\\ IDS от Internet Security Systems
<<RealSecure>>:\\
система RealSecure - это интеллектуальный анализатор пакетов с расширенной
базой сигнатур атак действующий в реальном масштабе времени (real-time packet
 analysis), она относится к системам обнаружения атак, ориентированных на защиту
 целого сегмента сети.\\

IBM RealSecure Network IDS\\
Используемый сенсором модуль обнаружения атак Protocol Analysis Module (PAM) аналогичен модулю используемому в аппаратных устройства Proventia Network IPS. Кроме того, RealSecure Network Sensor может переконфигурировать правила межсетевого экрана Checkpoint Firewall-1 используя протокол OPSEC.

Также существуют различные коммерческие IDS доступные для фирм <<средней
руки>>.

IDS и IPS весьма опасны для владельцев rootkit. Для них существуют утилиты
агрегации и обработки статистики, которые могут выдать отчёт о событиях
связанных с безопасностью в аналогичном наглядном виде, как для
маршрутизаторов и прокси. Это позволяет например обращать внимание на
регулярные нарушения, которые в единственном экземпляре не привлекли бы
внимания. Также возможна запись трафика по выявлении атак для последующего анализа.

\paragraph{DarkNets}
\label{darknet}
 представляют из себя области адресации внутри ЛВС, в которых
ни один адрес не занят компьютером или другим оборудованием способным отвечать по
сети. Для DarkNets действуют нормальные правила маршрутизации, но трафик идет
через сенсор IDS. Наличие unicast трафика в darknet с большой вероятностью говорит
 о том, что хост инициировавший трафик заражен.\\

Помимо ЛВС такие же сети существуют и в области адресов реального интерннета, что
 позволяет выявлять аномалии в трафике, в частности, наличие нового типа трафика в
darknet может сигнализировать о появлении нового сетевого червя.\\


\paragraph{коммутаторы, концентраторы}
\index{коммутаторы}\index{концентраторы}
Существенным интеллектом для обнаружения не обладают, но могут быть
использованы для подключения монитора трафика (в простейшем случае -
tcpdump) и записи его на хранение для последующего анализа. Так, все
коммутаторы компании cisco поддерживают port mirror - зеркалирование
траффика одного порта в другой. Некоторые IPS и IDS могут предоставлять возможность
динамического управления конфигурацией по событиям (такие функции
также есть у специализированного ПО cisco). Последнее время концентраторы
 практически не выпускаются (сняты с производства).

\paragraph{honeypots\\}
\label{honeypots_in_general}
\index{honeypot}
\index{хонипот}
Самый эффективный и опасный для владельцев root-kit вариант. HoneyPot
переводится как <<ловушка>>. Это программа или компьютер, работающие в
качестве приманки.

Наиболее продвинутые варианты honeypot'ов делаются профессионалами,
причем есть проекты, где машина ловушка не просто пассивно ждет, когда
на неё попадет  незадачливый взломщик, а  используется для простейших
действий (например серфинга сети), что позволяет таким ловушками при
определённом стечении обстоятельств получить закладки инсталлируемые
автоматически только на компьютеры используемые для net-serfing'а.

\subparagraph{интерактивные honeypots\\}
\index{интерактивный хонипот}
\index{интерактивный honeypot}
Под интерактивным honeypot здесь понимается компьютер, которые
используется для обычного серфинга сети\footnote{Тут стоит отметить,
что классификация honeypot'ов в этой статье отличается от ставшей классической
Spitzner'овской классификации, где honeypot'ы деляться на <<High-interaction>>,
<<medium-interaction>> и <<Low-interaction>> - по глубине эмуляции сервиса.
Спицнер подразумевает под интерактивностью взаимодействие программ -
низкоинтерактивный хонипот отличается от высокоинтерактивного глубиной
 эмуляции сервиса атакуемого злоумышленниками, я же использую термин
 интерактивность для того чтобы отделить honeypot'ы без
активного участия человека (неинтерактивные) от honeypot'ов предполагающих
 серфинг сети пользователем (или эмуляцию поведения пользвателя скриптами).
Как показано в данной статье на не серфящий сеть компьютер руткит может
просто не попасть (практический пример - распространение так называемого
 bootkit'а (классификация команды Kaspersky lab) - заражения происходили
 только для тех компьютеров, которые кликали по линкам на сайтах с вредоносным
 контентом - открыть сайт было недостаточно), поскольку алгоритм заражения
 предполагает наличие пользователя за компьютером}. Таким образом, на
 этот компьютер попадают не только активные члены сети в виде червей, вирусов и
незадачливых хакеров, но так же закладки устанавливаемые исключительно
при входе на какой нибудь сайт (например многие сайты с порно продают
хостинг для спамеров, которые распространяя через этот хостинг закладки
организуют botnet'ы, которые затем используются для распространения
спама, фишинга и кражи персональных данных).

Примером такой интерактивной системы honeypots может служить открытый
недавно микрософтом проект: http://research.microsoft.com/HoneyMonkey/ .

После заражения компьютер поступает на анализ к команде специалистов,
которая  занимается, как минимум,выявлением методов заражения. В худшем
случае для владельцев root-kits пойманный экземпляр подвергается полному
реверсингу - дизассемблирование, выявление алгоритма работы, используемых
для обмена информацией серверов и ресурсов сети.

Позволить себе интерактивные хонипоты с участием людей могут только большие
компании либо большие  госучереждения. Однако системы с эмуляцией поведения человека
не так дороги в обслуживании. Различные организации поддерживают также автоматизированные
 системы анализа malware (например CWSandbox, )
Типичные <<заинтересованные лица>> в создании
хонипотов:

\begin{itemize}
\item{Компании производители ОС (Microsoft уже <<проставилась>>, кто следующий?)}
\item{Компании производители антивирусного ПО}
\item{Компании продающие различные security-related сервисы}
\item{Военные/разведывательные ведомства}
\end{itemize}

\subparagraph{Реакция на обнаружение закладки}

В случае попадания в ловушку с мониторингом на внешних к хонипоту устройствах,
закладка будет обнаружена по трафику гарантированно. Если за время
существования в режиме <<подопытного кролика>> закладка не проявила
видимых  деструктивных действий, то дальнейшее его исследование
зависит от типа honey-pot'а. Во многих honeypot'ах проигнорируют
безвредного серфера, или же просто занесут его в базу adware ПО.
В редких антивирусных компаниях, возможно, займутся боле-менее плотным
reversing'ом и в конечном итоге закладка появится в
наборе сигнатур какой либо антивирусной компании(а значит, рано или
поздно - у всех разработчиков антивирусов). Крайне редким, на мой взгляд
будет случай детального разбора закладки с выяснением подробностей
алгоритма  её работы. Большинство антивирусных компаний удовлетворится
списком перехватываемых функций и созданием сигнатуры в базу - для
лечения на других компьютерах.
