\section{Противодействие антивирусному ПО.}
\label{honeypot_section}
\label{honeypot_section_page}

\subsection{Общие замечания.}

\subsubsection{Особенности антивирусного ПО}

\paragraph{Безусловные:\\}

\begin{itemize}
\item{при применении эмуляции в момент анализа бинарника не может ждать существенное время}
\end{itemize}

\paragraph{Возможные:\\}

\begin{itemize}
\item{Реализация personal firewall}
\end{itemize}


\subsection{Возможности противодействия}

Резюмируя вышесказанное,  можно сказать, что противодействие антивирусному ПО
 можно разбить на следующие направления:

\begin{itemize}
\item{антиэвристические приемы - antiheuristics}
\item{антиэмуляционные приемы -antiemulation}
\item{обнаружение антивирусного ПО + индивидуальный подход к антивирусу}
\item{антиотладка - antidebugging}
\item{противодействие дизассемблированию - antidisassembly}
\end{itemize}

%Antigoat  ??????????

\paragraph{противодействие дизассемблированию}

Помимо шифрование блоков закладки\ref{antihoneypot_crypto} для скрытия алгоритмов используемого payload необходимо использовать шифрование для  исключения определения антивирусом последовательностей кода используемых для эксплуатации уязвимостей ОС и перехвата управления. Это аналогично шифрованию используемому вирусами.

\paragraph{обнаружение антивирусного ПО + индивидуальный подход к антивирусу} - антивирусного ПО выпускается не так уж и много - не более десятка популярных производителей. Соответственно можно обнаруживать соответствующие им ключи реестра, имена процеесов, глобальные системные семафоры и прочие объекты.

\paragraph{антиэвристические приемы}

\subparagraph{Обфускация кода.\\} Как минимум - подготовка регистров арифметическими операциями вместо прямой загрузки перед вызовом функций их использующих. Пример:
\begin{verbatim}
Вместо:

MOV   CX, 100h ; this many bytes
MOV   AH, 40h  ; to write
INT   21h      ; use main DOS handler

Используется:

MOV          CX,003Fh  ; CX=003Fh
INC          CX        ; CX=CX+1 (CX=0040h)
XCHG         CH,CL     ; swap CH and CL (CX=4000h)
XCHG         AX,CX     ; swap AX and CX (AX=4000h)
MOV          CX,0100h  ; CX=100h
INT          21h
\end{verbatim}

\subparagraph{CRC\\} Использование CRC вместо имен функций для поиска.

\subparagraph{non-writabble entry point \& other code\\}
Writeable сегмент кода - добавка подозрений эвристику. Альтернатива - расшифрование в стеке.

\paragraph{антиэмуляционные приемы\\}

\subparagraph{медленный вход\\}
Подразумевает использовааание длительных вычислений перед началом работы кода который может быть признан эмулятором как подозрительный. Эмулятор кода в момент проверки файлов не может ждать несколько минут на каждый файл.

\subparagraph{использоваание random encryption key\\}
\label{random_enc_key}
В момент запуска закладки она может расшифровывать часть себя используя подбор ключа шифрования. Это одновременно и медленный вход и позволит избежать хранения ключа на закладке. Однако такой способ не должен использоваться для зашифрования всего кода, поскольку для того чтобы закладка смогла расшифровывать себя в разумное время (10 - 30 минут) схема шифрования должна быть относительно слабой.

\subparagraph{использоваание расшифровывающего кода как части ключа\\}
\label{decryptor_as_a_part_of_key}
Некоторые эмуляторы оптимизируют код подбирающий ключи, т.е. этот прием нарушит расшифрование.


%\subsubsection{Выявление эмуляции кода}

%\paragraph{Блокировка доступа к сети}
%
%Характерна для антивирусного ПО включающего personal firewall.


\subsubsection{Инкубационный период}
\begin{enumerate}
 \item {пауза от начала выполнения - никакой активности }
 \item {пауза перед инсталляцией в автозапуск на диск - сбор информации о системе,
тесты на известные антивирусы}
 \item {регистрация на сервере с отправкой информации о системе}
\end{enumerate}
