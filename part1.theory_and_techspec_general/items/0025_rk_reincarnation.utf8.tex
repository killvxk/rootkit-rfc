\section{Различные приёмы возвращения rootkit после того как основной модуль был удалён}

\subsection{Установка закладок <<реинкарнаторов>>}

После успешного окончания инкубационного периода rootkit устанавливает
несколько автономных  программных модулей, единственным payload которых
является загрузка текущей версии rootkit-инсталлятора  из интернет.
Модуль стартует при загрузке системы или, что лучше, при загрузке
определённых приложений.  Скачивание происходит раз в месяц плюс минус
несколько дней по случайному принципу. После того как rootkit скачан он
обращается к серверу за инструкциями и, в зависимости от решения
принимаемого на сервере получает к исполнению тот или иной бинарь или
получает инструкцию выгрузиться из памяти.


\subsubsection{Описание закладки реинкарнатора.}

Закладка-инсталлятор должна представлять собой простой http/irc/(другие
протоколы?) downloader/executor. Задача такой закладки - скачать и
выполнить новый модуль первой ступени в случае удаления закладки из
системы.

Таким образом мы можем добиться того, что данный модуль сможет повторить
инфицирование новой  версией rootkit данной машины через достаточный
промежуток времени. Разумеется только сервер всегда сможет отдать любой
бинарь, вместо текущей версии опираясь на собственные данные о
запрашивающей машине.

\subsubsection{Предостережение.}

Вообще говоря, делать возрат на машину, на которой модуль был удален
опасно. Уж очень велика возможность попасть в грамотно сделанный хонипот.
Ведь владелец как то смог отловить и деактивировать rootkit. Так что лучше
10 раз подумать прежде чем вступать на такой опасный очередным разоблачением
путь. Единственный вариант, при котором такое может быть нужным, если сеть
закладок осталась <<на автопилоте>> без контроля и за время, которое она так
работала все или часть антивирусных пакетов вдруг прозрели и стали определять
наличие некоторых модулей закладки, производя лечение не полностью, раз
независимая закладка-реинкарнатор сохранилась (если это не ловушка конечно).



\subsubsection{Минимально необходимый список проверок.}

К возвращению rootkit на компьютер, с которого он был ранее удален
требует особенно жестких проверок. Ведь, в данном случае
{\bfчерезвычайно велик риск напороться на активное
противодействие и подготовленную ловушку}. Решение о возврате желательно
принимать индивидуально, а хосты на которые rootkit инсталируется
повторно должны быть, во первых, под особым контролем, а во вторых,
обслуживаться на  отдельных серверах.

\begin{itemize}
\item{отдельные сервера}
\end{itemize}

\paragraph{Отдельные сервера}

Это требование связано с тем, что если за сеть закладок возьмутся
всерьез, то соответствующие  заинтересованные госструктуры могут обязать
к принятию мер, как минимум, следующих  юридических лиц:

\begin{itemize}
\item{Провайдера хостинга}
\item{Провайдера трафика}
\end{itemize}

\subparagraph{провайдера хостинга} можно обязать  предоставить
физический доступ к данным на компьютере , что решаемо со стороны
владельца ботнета использованием зашифрованных файловых систем (частично,
поскольку если сервер виртуальный - может быть применен дамп памяти, то
есть так  можно получить ключи к файловой системе, следовательно к данным).
Тем более это решаемо в случае, если покупается хостинг физического
компьютера - например датчик открытия корпуса завязывается на мобильник,
 который  висит на зарядке питающейся от Б/П. По срабатыванию датчика
отправляется СМС. Однако, это отдельный след в real life: предоставить
компьютер должен либо живой человек, либо  служба доставки, которая,
соответственно, должна принять его отнюдь не у виртуального персонажа,
а у реального человека с паспортом той или иной страны - это слабое звено,
поскольку анонимизация в реальном мире доступна лишь ОПГ\index{ОПГ} и различным
боле-менее курпным компаниям и специфическим госструктурам.

\subparagraph{провайдера трафика} могут обязать, с какого либо момента, сохранять
лог обращений и даже трафик работы с сервером по некоторым или всем протоколам
(СОРМ\index{СОРМ} это предусматривает), таким образом это может помочь в выявлении управляющего
сервером персонала и, затем, владельца ботнета. Противодействовать этому можно лишь
используя для административного доступа специальные средства анонимизации - цепочки
 анонимных прокси и сети подобные TOR\index{TOR}.
