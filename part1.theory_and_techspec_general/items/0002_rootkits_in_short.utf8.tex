
\section{Краткий обзор .}

\subsection{Текущее положение дел}

\paragraph{Современный RootKit}
 это закладка работающая с сетью и позволяющая осуществлять удалённое
управление компьютером на котором он установлен. Конечно возможны
варианты с исключительно локальным использованием, однако удалённое
управление более чем удобно - зачастую это просто необходимость.\\

Это часто не индивидуальная закладка, а  массово применяемая, но
достаточно гибкая для индивидуальной настройки. В первую очередь это
связано с тем,  что, так или иначе, но для rootkit желательна
отдельная серверная машина в сети.\footnote{ведь не у каждой
машины носителя, подключенной к сети, есть адрес доступный всему
интернет.}  Если такая машина есть, то разница между управлением с нее
одним компьютером или тысячей не слишком велика.\\

В этой главе рассматривается пример существующих сетей
зараженных машин.

\subsection{Массовость}
Разработка алгоритма массовых заражений окупается гораздо лучше
индивидуальных закладок. Более того, разумно построенный алгоритм
массовых заражений не исключает возможности индивидуальной работы с
клиентом. Плюс контроль над значительным числом компьютеров позволяет
реализовывать схемы распределённых вычислений  и DDoS атаки, что весьма
популярно в современности.


\subsubsection{Примеры}

Для того чтобы проще воспринимать суть вещей начнём с примера.
Подходящий пример - распространение спама.

\paragraph{Пример: спамерские bot-net'ы}

Современные сети изобилуют спамом. Спам рассылается при помощи сетей
зараженных компьютеров. Спамеры или создают собственные сети зараженных
компьютеров, или покупают доступ к сети уже зараженных закладками
компьютеров. Вторых - большинство. Спамеры (и не только они) используют
термин  <<bot>> для обозначения заражённого компьютера\footnote{точнее
этот термин часто употребляется как по отношению к заражённому
компьютеру, так и по отношению к заразившей программе}, а <<botnet>>
для обозначения сети таких компьютеров.

\paragraph{Боты(bots)},
фактически представляют собой программные закладки различной сложности.
Работа спамерского бота заключается либо в предоставлении анонимного
транзита для рассылки спама(см. \pageref{bot_as_relay}),  либо в
собственно рассылке спама.

\paragraph{Проникновение в систему}
 спамерских ботов происходит за счёт действий самих пользователей.
Фактически, схема проста - спамеры покупают хостинг (обычно - на порно
серверах - cамый популярный ресурс в интернет де факто - порно
контент) для exploit'ов . При посещении порно сервера у пользователя
пользующегося Internet Explorer\footnote{Браузер по умолчанию в windows
системах, соответственно самый часто используемый браузер. Разумеется IE постоянно
обновляется, но так или иначе находятся новые серьезные ошибки. Также возможно
проникновение через ошибки в других браузерах, однако всвязи с популярностью (IE
лидирует, за ним FireFox, затем Safari и остальные) большинство эксплойтов
пишется для IE и его расширений.} срабатывает уязвимость браузера:
скачивается и запускается программа, которая собственно и инсталирует
закладку (т.е. бота) в ОС пользователя.

Такова картина для Win32 систем. Возможно разумеется и проникновение ботов в linux
 и другие unix-like системы, но опять-же из-за популярности лидируют боты на Windows
 системах.

\subparagraph{прочие популярные методы проникновения\\}
 Получила также распространение методика установки ботов методами аналогичными распространению сетевых червей - через
 email рассылки с вредоносным контентом, через использование уязвимостей в ОС по сети.\\
 Алтернативой работы с собственным хостингом может быть взлом web серверов с подменой линков или внедрением дополнительных фреймов с эксплойтами, при этом, если web страницы генерируются динамически из БД в которой
присутствует sql injection моодифицируют саму БД так чтобы генерируемый код вызывал эксплуатацию уязвимостей
обратившихся к сайту.\\


\subparagraph{Загрузки\\}
Хостинг уязвимостей среди спамеров называется хостингом <<загрузок>>,
говоря <<загрузки>> подразумевают покупку хостинга для эксплойтов к
браузерам, поскольку, фактически, это обеспечивает загрузку закладок в
компьютеры незадачливых пользователей. Загрузки не подразумевают покупку
хостинга (для сайта например), они могут использоваться как услуга.

\subparagraph{Предоставление ботом транзита }
\label{bot_as_relay}
для рассылки осуществляется следующим образом: бот\footnote{т.е.
закладка}  устанавливает на зараженном компьютере открытый прокси
сервер, позволяя подключаться к нему с любого адреса в сети, при этом
протокол работы не ведётся. После установки свободного прокси сервера
закладка каким либо  образом рапортует о себе, после чего спамер, зная
адрес закладки,  может подключаться к ней и,  используя её, отправлять
сообщения <<от имени>> заражённого компьютера.\\  Де факто, компьютер,
предоставляющий сервис свободно  доступного прокси, интересует многих
пользователей сети по различным причинам\footnote{Кому то, так же как и
владельцу закладки, интересно использовать её для отсылки спама, кому
то, в силу особенностей подключения, выгоднее работать через прокси, чем
напрямую, а кому то нужна анонимность при работе с сетью.} Поскольку бот
позволяет использовать себя как прокси всем желающим, открытый им порт
достаточно быстро находит множество заинтересованных людей. Таким образом, из за
большого количества соединений, становится труднее вычленить из общего
потока данных трафик хозяина закладки.  При такой схеме спамеры обычно
используют один или несколько компьютеров объединённых в сеть для
рассылки почты через тысячи зараженных <<транзитных>> машин .\\
Вариантом этого случая является предоставление транзита после аутентификации
 клиента - чаще всего в этом случае доступ к транзиту продается владельцем
сети зараженных машин (аренда ботов).

\subparagraph{Рассылка с бота}
осуществляется в общих чертах так: после установки и осуществления ряда
проверок бот обращается к серверу спамера, забирает у него список
заданий и осуществляет отправку почты самостоятельно.

Владельцев рассылающих ботов, на 2006й год было не много, однако рано
или поздно  их станет больше, поскольку выгода такого решения в
производительности очевидна - тысячи компьютеров  отсылая спам дают
 скорость бОльшую, чем один или несколько
компьютеров, рассылающих через тысячи. Это решение  эффективнее, хотя и
сложнее - сложнее сам алгоритм работы. Кроме того оно расширяет
<<целевую  аудиторию>>. Дело в том, что компьютеры организаций находятся
обычно за шлюзовым компьютером и если открыть  на них прокси свободного
доступа, то этот прокси из интернет будет напрямую
недоступен\footnote{например, компьютеры за NAT'ом}, т.е. им смогут
пользоваться только члены этой локальной сети, поскольку нет корректного
способа установить соединение с компьютером за шлюзом. Как следствие -
такие компьютеры неинтересны тем,  кто рассылает спам используя
транзитные соединения. Однако для алгоритма с рассылкой с бота таких
ограничений нет - бот приходит к спамеру на сервер за заданием сам, не
требуя никаких предварительных соединений извне и не открывая сервисов
на зомбированном компьютере. У такого бота есть различные плюсы и минусы.
Например: плюс в том,  что отсутствие открытых портов затрудняет активный
поиск через сеть, минус в том, что нет маскирующего  трафика от случайно
нашедших открытый сервис членов глобальной сети.

\subsection{Клиент-серверная модель}

Поскольку с одного или нескольких серверов возможно управлять многими
тысячами компьютеров логично называть управляющие компьютеры серверами,
в основном исходя из того, что они раздают задания для клиентов
(ботов/закладок). Вообще отношения закладки и рассылающего сервера
соответствуют тому, что принято называть <<клиент-серверная
технология>>:
\begin{verbatim}
см. http://ru.wikipedia.org/wiki/Технология_«клиент-сервер»
\end{verbatim}

Здесь и далее взаимодействие между rootkit и управляющим сервером называется для
удобства  клиент-серверным\footnote{при этом, в
зависимости от контекста, rootkit может выступать и клиентом (получение
заданий) и сервером(выдача файлов и прочей информации по запросу с
управляющего компьютера)}. В различной литературе управляющие сервера часто
называют <<C\&C>> от словосочетания <<Command \& Control>>.

\subsubsection{Преимущества схемы клиент-сервер}

\begin{itemize}
\item{масштабируемость}
\item{экономичность кодирования}
\end{itemize}

\paragraph{Масштабируемость}

\begin{itemize}
\item{
Возможность распределения серверной части алгоритма на несколько серверов.}
\item{
Возможность распределения нагрузки при росте числа контролируемых машин.}
\end{itemize}

\paragraph{Экономичность кодирования}

Однажды написав базовую функциональность сервера и клиента далее ее
можно улучшать и доводить до ума не тратя время на разработку программ
целиком для каждого конкретного случая.

