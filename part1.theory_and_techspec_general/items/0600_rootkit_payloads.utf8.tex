\section{Примеры payload современных rootkits}
\label{section_rk_payloads}

\subsection{Paid proxy}
Владелец ботнета продает доступ к прокси: доступ к транзиту на боте осуществляется по ключу,
по окончании срока аренды ключ меняется - так называемая <<аренда ботов>>.

\subsection{DDoS}

\paragraph{Массовое единомоментное открытие соединений\\}
Сотня тысяч компьютеров послав одновременно запрос на использование
затруднят работу любого сервиса. Использование сетей зомбированных
компьютеров для организации DDoS атак уже давно стало привычным
событием.\\
Раскрытие взаимодействия в рамках сети зомбированных компьютеров - очевидное
следствие участия в DDoS атаках. Поэтому, если сохранение rootkit на
данном компьютере существенно, то его не следует использовать при
организации DDoS атак.

\subsection{Distriibuted Calculations}

\subsubsection{Distributed Net ( dnet ) }\label{dnet} - широко известная
 инициатива  интернет сообщества по оценке стойкости алгоритмов шифрования и
алгоритмов хеширования. Суть её состоит в том, что каждый желающий
может заставить свой компьютер работать над некой вычислительной задачей
совместно с компьютерами других энтузиастов. По умолчанию этот проект
использует остаточную вычислительную мощность компьютера, т.е. то, что
осталось после выполнения задач ОС и пользователя.

Пример организации distributed net энтузиастами на территории exUSSR
можно посмотреть на http://bugtraq.ru/dnet/ .

Инфицированные компьютеры, точно так же, как и любые другие,
используются большую часть времени не на полную мощность, так что
возможно написать модуль к rootkit, который будет использовать время
простоя этих машин на пользу владельцу root-kit сети.

\subsubsection{Intelligent Distributed Calculations}

Совершенно необязательно производить перебор <<подряд>>, как это
делается в реализации Distributed Net\footnote{такой перебор называют перебором грубой силы - <<bruteforce>>} (см. выше, стр. \pageref{dnet}
глава \ref{dnet}). Дело в том, что проект Distributed Net
не слишком заинтересован в оптимизации алгоритма вычислений и
поддерживается энтузиастами, которые зачастую почти не имеют
отношения к криптографии. Однако, согласно Шнайеру, при вычислениях
связанных, например, с выявлением закрытых ключей существенную роль
может играть объем доступной памяти. RootKit'у вполне по силам незаметно
отъесть десяток и более мегабайт памяти на многомегабайтной системе
\footnote{редкий современный компьютер не имеет хотя бы 128 мегабайт памяти},
а на системах со старенькими компьютерами можно использовать меньше памяти.
Допустим на каждой зараженной машине отъедается, в среднем, 10Mb ОЗУ. Таким образом,
при объемах <<ботнета>> порядка 300 тысяч компьютеров\footnote{не самая большая сеть}
количество доступного ОЗУ составляет 3 терабайта. Однако, с учетом необходимых
накладных расходов (резервирование участвующих вычислительных единиц), существенно меньшая
скорость доступа к распределенной памяти выигрышь может быть не столь значителен или вовсе
может отсутствовать - это требует математической оценки.

\subsection{Псевдополезное ПО}
Одним из возможных вариантов работы закладки на зараженном компьютере может быть сценарий, когда закладка выдает себя за антивирус - без реальной антивирусной работы выдает предупреждения о якобы присутствующих в системе вирусах с предложениями купить ее для лечения. У доверчивых пользователей таким образом могут быть не только зря потрачены деньги, но и украдены данные необходимые для работы с их кредитными картами. Возможен также выпуск ПО, которое действительно будет делать что-то полезное + участвовать в ботнете..

\subsection{Spamming}

Весьма выгодная деятельность, описанная мной в начале этой статьи в
качестве примера зачаточной реализации rootkit (как спамерского
узкоспециализированного бота). Спамерство как одна из движущих сил экономической
оправданности разработки ПО для организации ботнетов обсуждается во многих источниках.
Однако,  следует учитывать, что если данный компьютер участвует в рассылке СПАМа это
очень быстро будет  выявлено - существует множество служб
выявления спама, борьбы со спамом, включая  регистрацию хостов в
публично доступных базах данных. Жалобы на рассылку и прочее. Фактически
спамеры постоянно покупают новые загрузки, как минимум два-три раза в
меся. Т.е. рассылка спама с большой  вероятностью повлечёт переустановку ОС
пользователем. Что в свою очередь повлечёт потерю закладки. Также массовая рассылка спама
достаточно быстро приведет к пристальному вниманию к самой закладке и значительным усилиям
по реверсингу ее алгоритма, выявления управляющих серверов (с попытками блокирования серверов и попытками отследить управляющий серверами персонал, а через них выйти на создателя ПО с дальнейшим уголовным преследованием).

\subsection{Накручивание баннерной рекламы}
Еще один вариант зарабатывания денег - пользователь регистрирует нормальный web сайт и размещает на нем баннерную рекламу. Боты периодически загружают сайт и таким образом накручивают счетчик посещений. Каждая
загрузка банера дает небольшую денежную сумму, но поскольку посещений много, они постоянные и с случайных
 IP - уличить в мошенничестве организатора махинации крайне сложно.

\subsection{влияние на торговые отношения на биржах}
Было актуально на 2006й год для США. Позднее рекламу торгуемых лотов запретили.
Возможно данный тип махинаций еще где-то действует. Суть в рекламе через спам
определенного набора акций с предсказанием их роста или падения. Поскольку
количество получателей такого спама огромно - график продаж меняется за счет
поверивших рекламе и тех, кто хоть и представляет суть процесса, но все равно не
прочь <<сыграть в рулетку>>. Заказчик рассылки может, например, заранее закупить
 акции по низким ценам и сбросить их на пике роста. Выигрывает тот, кто успел сыграть
на дельте до возврата цен в нормальный для данного типа акций диапазон. Заказчик
рассылки выигрывает больше за счет больших объемов вложений в участвующие в афере акции.

\subsection{Network analysis}

Возможность получить из удалённой точки информацию о маршрутизации
иногда бывает очень кстати.  Вообще для разных IP-сетей некоторый
хост может иметь разный список правил в своем firewall software,
причем это может быть реализовано в том числе на уровне принадлежности
к определенным автономным системам. В частности, многие зарубежные
системы реализующие grey-listing (<<взвешивание>> IP адресов) заведомо
отдают свое мнение о сетях принадлежащих РФ как о потенциально-опасных.

\subsection{File Grabbing}

Передача файлов пользователя без его ведома - вполне доступная для
rootkit задача. Даже если пользователь использует зашифрованные диски.
\footnote{Однако, если используется шифрование и зашифрованные данные
не подмонтированы на момент заражения - придется дождаться когда
система сама (по требованию пользователя вводящего пароль или предоставившего
носитель с ключами) обратится к зашифрованному диску, то есть дождаться
входа пользователя в систему с монтированием криптодисков}
Вообще говоря любые файлы компьютера, на котором установлен rootkit
помимо пользователя компьютера становятся доступны владельцу
rootkit.\footnote{см. также \ref{local_data_access} на стр.
\pageref{local_data_access}}


\subsection{Account Grabs}

Современные пользователи чрезвычайно доверчивы и хранят на компьютерах
очень много информации связанной с доступом в различные места, включая
то, что не имеет отношения непосредственно к интернет. Владельцу rootkit
доступна и эта информация. Кроме того rootkit может брать пароли
непосредственно из памяти приложения когда оно запущено\footnote{Например,
в Mozilla есть функция мастер пароль для хранения паролей к сайтам требующим
ввода пароля. Пока не введен мастер пароль все пароли хранятся на диске в
зашифрованном виде. Когда мастер пароль введен пароли расшифровываются и
загружаются в память, которую, в свою очередь, уже может прочитать закладка}.
Кроме того закладка может считывать пароль в момент ввода (если пользователь
не хранит пароли в легко расшифровываемом хранилище).

\paragraph{Примеры}
\begin{itemize}
\item{пароли к icq и подобным internet pager'ам}
\item{пароли к почтовым аккаунтам, в том числе пароли к webmail сервисам}
\item{пароли к управляющим аккаунтам к различному оборудованию}
\item{параметры банковских карточек}
\item{пароли на доступ к различным системам электронных денег(paypal,webmoney и др.)}
\item{пароли доступа к базам данных, в том числе удаленным базам данных}
\item{pgp ключи}
\end{itemize}

\subsection{Traffic Sniffing}

Некоторые компьютеры подключены к хабам, а не коммутаторам, что
позволяет видеть трафик предназначенный другим компьютерам. Реализация
такой функции возможна. Также (в индивидуальном порядке) возможно
применение атак на коммутирующее оборудование для получения доступа к
транзитному трафику\footnote{Имеются ввиду атаки на ARP и STP.} Однако
в сетях крупных компаний часто используются интеллектуальные устройства
коммутации и маршрутизации совместно с программами мониторинга, т.е. выполнение
атак подобного рода в автоматическом (да и вручную) режиме может быстро
выявить наличие заражения.


\subsubsection{KeyBoard sniffing}

Очень часто применяемый метод слежения за пользователем - запись всего
что он набирает на клавиатуре. Помимо прочего так можно красть пароли
во время ввода.

\subsection{User Stats}

Возможно получать различную статистику по действиям пользователя.
Например, список и статистику загружаемых программ (список, время,
частота использования), историю его серфинга интернет (список, пароли
доступа и прочее).

\subsubsection{Personal Data} \label{local_data_access}

Очень часто пользователи доверяют компьютеру хранение личных данных.
Разумеется, если информация есть в компьютере - она доступна для rootkit
тем или иным образом.

\paragraph{Доступность данных на зашифрованных носителях}

Даже если пользователь хранит данные на зашифрованных дисках (BestCrypt, PGP-диск и прочее) они тоже
могут быть доступны:

\begin{itemize}

\item{
Во первых, не всегда программа реализующая криптодиски позволяет
ограничить доступ к ним на  уровне приложений, а значит после
монтирования зашифрованного диска файлы на нем доступны для  просмотра и
копирования любым приложением с достаточными локальными правами
}

\item{Во вторых, даже если программа организующая криптодиск ограничивает
доступ  к нему определенным набором программ -  можно либо открыть
собственный поток в разрешенном приложении или просто перехватив пароль
при вводе эмулировать <<законный>> доступ.}

\end{itemize}

Тем не менее, несмотря на то, что доступность данных с криптодисков для
rootkit очевидна, необходимо  тестирование доступных программных
продуктов во избежание досадных ошибок из за которых пользователь сможет
по неадекватному поведению приложений понять, что компьютер находиться
под контролем.

\subsection{Сбор информации с multimedia устройств компьютера}

При доступе к системе на уровне OS/драйвера (ring0 access) не
составляет проблем доступ к, например, видеокамере или фотоаппарату
подключенным к компьютеру. Вопрос лишь в том чтобы написать
соответствующий модуль к rootkit, который смог бы воспользоваться
удаленным устройством.

\paragraph{Список устройств которые могут быть использованы rootkit\\}

Такие устройства на мой взгляд следует поделить на те, к которым следует
обращаться косвенно и те, к которым можно обращаться напрямую.

\subparagraph{Устройства, которые можно использовать непосредственно\\}

\begin{itemize}

\item{ микрофон, если подключен к звуковой карте }

\item{usb video камеры, если подключены и поддерживаются rootkit(хотя возможно обращение
через драйвер, если установлен в системе)}

\item{удаленные (ethernet/wifi/IP) video камеры}

\end{itemize}

При этом следует осознавать, что факт доступа к некоторым утройствам
может журналироваться,  поэтому желательно использовать косвенный доступ
к удаленным устройствам (см. ниже).\\

Также устройства типа web камеры часто имеют светодиодную индикацию, по которой
пользователь может понять, что камера используется не им. Для таких устройств
нужно совмещать доступ с использованием устройства самим пользователем.

\subparagraph{Устройства, доступ к которым можно осуществлять косвенным образом}

Под <<доступ косвенным образом>> подразумевается перехват результатов работы с устройством.

\begin{itemize}

\item{Принтеры (любые) - возможно получать задания в момент их отправки на печать с
последующей обработкой и возможной отправкой
}

\item{ Сканеры(любые) - возможно получение результатов сканирования с последующей их обработкой и возможной отправкой }

\item{Все устройства с которыми можно работать непосредственно}

\end{itemize}

Необходимость работы с этими устройствами косвенным образом очевидна -
обращение к таким устройствам заметно (индикация на корпусе, шум работы,
движение механических элементов). Весьма желательно осуществлять
косвенным образом и доступ к удаленным multimedia устройствам, поскольку
обращение к ним может журналироваться. При необходимости можно использовать
косвенный доступ и к устройствам которые можно использовать непосредственно.

\subsection{ScreenShots}

В числе прочего, возможно получать снимки экрана. Однако, следует
учитывать, что при сборе информации средствами Windows API происходит
затормаживание графического интерфейса, хоть и кратковременное  (доли
секунды), но все же заметное внимательному пользователю\footnote{Один из
моих знакомых  озаботился проверкой наличия закладок на своем компьютере
именно благодаря таким вот симптомам и, действительно, обнаружил на
домашнем шлюзе трафик сгенерированный закладкой}. Таким образом, если
данный тип  payload актуален, есть смысл подумать о реализации
snapshot'ов <<собственными силами>> ставя  целью, в первую очередь, не
качество полученного изображения\footnote{четкость, цветность}, а в
первую очередь, незаметность процесса <<фотографирования>> desktop'а и
малые размеры полученных <<фото>> .


\subsection{Zero Knowledge System (zks) any trafic relay}

\paragraph{Обзор}

Существовала когда то контора которая организовывала платный сервис
доставки пакетов с гарантией  анонимности. Ныне контора
перепрофилировалась и больше таких услуг не поставляет - есть
мнение - <<большой брат настоял>>. Идея проста - каждый пакет
шифруеутся на 3х ключах, отправляется на ближайший сервер сети
zks . Сервер расшифровывает пакет, отправляет следующему известному
ему серверу. Не зная адреса и порта назначения. Второй сервер делает то
же самое. Также не имея понятия куда это все идет и откуда. Далее только
на 3м сервере пакет  отправляется уже на хост назначения.\\

Резюме: очень сложно выяснить пути пакета, отправителя и получателя,
поскольку сервера раскиданы на разных континентах и под разной
юрисдикцией.


\paragraph{Подробнее}

Приведу цитату из доступных опубликованных материалов:

\begin{verbatim}
Цитата:

=========

Одним из интересных средств обеспечения конфиденциальности является
система Freedom ("Свобода"), разработанная канадской корпорацией
Zero-Knowledge Systems. Система Freedom предназначена для анонимного
просмотра страниц в Internet, обмена электронной почтой и участия в
конференциях Usenet (группах новостей). Система функционирует на базе
специальных серверов, разбросанных по всему миру. Когда кто-то хочет
послать сообщение в Internet, просмотреть веб-страницу или принять
участие в другой электронной транзакции, зашифрованное сообщение
посылается с компьютера этого человека на один из серверов Freedom.
Первый сервер пересылает сообщение на второй сервер, который, в свою
очередь, пересылает его на третий, который, наконец, отправляет его по
назначению. Каждое отправляемое сообщение зашифровано три раза, с
последовательным использованием ключей серверов. Устройство системы не
дает возможности человеку, перехватывающему сообщения (или имеющему
контроль над одним из серверов Freedom) одновременно узнать и личность
отправителя сообщения, и его содержимое. Фактически, Zero-Knowledge
разместила сервера по всему миру, чтобы максимально затруднить для
отдельно взятого правительства возможность изъятия содержимого всех трех
серверов, задействованных в пересылке конкретного сообщения.

=========
\end{verbatim}


