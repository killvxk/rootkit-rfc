\section{Глоссарий}
\label{glossary}

\begin{description}

\item[СОРМ]
\index{СОРМ}
- Система Оперативно Розыскных Мероприятий. Для провайдеров соблюдение требований СОРМ обязательно.

\item[ компьютер, машина, комп, ПК, ПЭВМ, ЭВМ ]
\index{компьютер}
\index{машина}
\index{комп}
\index{ПК}
\index{ЭВМ}
\index{ПЭВМ}
 - синонимичные понятия, которые,
надеюсь, не требуют пояснений. Тем не менее, в данной статье (в том числе в этом
глоссарии) термин компьютер
используется иногда для описания любого устройства существенно умнее калькулятора (то
есть имеющего возможность обеспечить работу операционной системы, пусть даже специализированной,
как, например, CISCO IOS. )
( alternate4fun: computer - a device designed to speed up and automate errors. )

\item[ИТ, IT, information technologies]
\index{ИТ}
\index{IT}
\index{information technologies}
 - информационные технологии. Термин очень широко
распространившийся в современности и не менее широко обобщаемый. Зачастую под IT понимают
вообще все что связано с компьютерами и, в частности, компьютерными сетями.

\item[тревожное событие, алерт, alert]
\index{тревожное событие}
\index{алерт}
- в контексте этой статьи - некоторое событие которое воспринимается системой защиты как тревожное,
при наступлении такого события могут быть произведены какие либо действия (например изменение правил firewall).

\item[ОПГ]
\index{ОПГ}
- Организованная Преступная Группа .

\item [ interface, интерфейс ]
\index{interface}
\index{интерфейс}
 - набор характеристик для взаимодействия между чем либо. Например,
для взаимодействия человека и компьютера чаще всего используется интерфейс в виде совокупности
монитора, манипулятора типа <<мышь>>, клавиатуры и набора программ для обслуживания событий
(нажатие кнопки на клавиатуре, например) с этих устройств. В свою очередь, человек, для
работы с этим интерфейсом,должен иметь интерфейс, как минимум, в виде рук и глаз . В IT
контексте под интефрейсом подразумевается, обычно, одно из следующих:
\begin{enumerate}
\item{внешний вид и характеристики(ток, сопротивление, вольтаж )разъемов на устройстве
(hardware context)}
\item{набор правил работы с программой реализуемых ее внешним видом - меню и т.п. (user context)}
\item{набор функций приложения или операционной системы доступных для программиста}
\end{enumerate}
Примеры интерфейсов:
\begin{enumerate}
\item{сетевые интерфейсы (плата ethernet, modem(в том числе adsl))}
\item{физические интерфейсы (ethernet (IEEE802.3), wifi(IEEE 802.11b, IEEE802.11g)), ps2, COM(RS232)}
\item{пользовательские интерфейсы (внешний вид и средства управления/настройки любых програм,
например IE, DrWEB,NortonComander)}
\item{програмные интерфейсы (набор API для работы с экраном, клавиатурой, мышью)}
\end{enumerate}

\item[виртуальная машина, virtual PC]
\index{виртуальная машина}
\index{virtual PC}
 - программа способная эмулировать компьютер.

\item[уязвимость, vulnerability]
\index{уязвимость}
\index{vulnerability}
 - возможность использования программы не по назначению, чаще
всего подразумевается использование либо без ведома пользователя программного продукта, либо, плюс
к тому, ещё и использование для получения дополнительных привилегий в системе разграничения
полномочий операционной системы в которой работает программа с уязвимостью.

\item[носитель, зомби]
\index{носитель}
\index{зомби}
 - заражённая машина или операционная система (в зависимости от контекста).

\item[anti-forensics]
- в контексте этой статьи, некие способы затруднения анализа закладки в процессе расследования инцидентов связанных с нарушением политики безопасности, в т.ч. в расследованиях связанных с попытками применить к владельцам закладки меры предусмотренные законодательством.

\item[offline, оффлайновый]
\index{offline}
\index{оффлайновый}
 - термин подразумевает, в первую очередь, отсутствие подключения к сети.
Часто применяется по отношению к программе до запуска (например, когда диск компьютера подключается
к другому компьютеру для исследования его содержимого, все программы на подключаемом диске находятся
 в offline'овом режиме).

\item[online, онлайн]
\index{online}
\index{онлайн}
 - термин подразумевает возможность взаимодействия с сетью, часто используется
по отношению к программе, подразумевая, что программа запущенна.

\item[гипервизор, hypervisor]
\index{гипервизор}
\index{hypervisor}
- виртуализатор, програмный или програмно-аппаратный комплекс позволяющий исполнять несколько операционных
систем на одном физическом компьютере, см:
\begin{verbatim}
http://en.wikipedia.org/wiki/Hypervisor
\end{verbatim}

\item[protocol, протокол]
\index{protocol}
\index{протокол}
 - набор соглашений об обмене информации. В it-контексте
подразумевается обычно протокол обмена данными по компьютерной сети или между
программными модулями (в т.ч. определяет объем данных и их логическое
представление на каждых приёме/передаче).
То есть протокол определяет форму
запрос/ответ или, иными словами, как понимать данные, которые приходят по
сети и как их в сеть <<говорить>> (в общем-то, человеческая речь это тоже
протокол)

\item[smtp]
\index{smtp}
 - simple mail transfer protocol, протокол для передачи почтового трафика.

\item[сервис,service]
\index{сервис}
\index{service}
 - практически то же самое, что и <<в миру>>, но по отношению к
приему информации. Если программа может предоставить пользователю или другой
программе информацию по приходу запроса, то эта программа предоставляет сервис.
Название сервиса чаще всего созвучно названию протокола который используется
для приёма и отправки запросов.

\item[сервер, server]
\index{сервер}
\index{server}
 - с точки зрения пользователя - <<какой-то>> компьютер, стоящий
<<где-то>> (хоть в соседней комнате, хоть на другом континенте), к которому
можно обратится по сети при помощи той или иной программы. В общем случае
термин применяется как к компьютеру в целом, так и к программе, которая на
нем установлена (работает) и предназначена для обслуживания клиентов, т.е.
<<предоставления сервисов>>.  Исторически сложилось, что аналогично сервисам
и клиентам, сервера именуются созвучно названию протокола - например http-сервер.
Бывает также, что слово сервер опускают, как бы, подразумевая его. Например
когда говорят <<прокси>> подразумевают прокси сервер.

\item[C\&C, управляющий сервер]
\index{C\&C}
\index{управляющий сервер}
в контексте данной статьи - сервера (возможно не единственный) для управления ботнетом.

\item[botnet herder]
\index{botnet herder}
такой <<ярлык>> часто встречается в сети, когда имеют ввиду либо владельца ботнета, либо административный
 персонал им управляющий. Перевод herder на русский - пастух.

\item[клиент, client]
\index{клиент}
\index{client}
 - с точки зрения пользователя - он сам. В общем случае,
в it-контексте, это либо компьютер, либо программа, которые сами сервисов
не предоставляют, но к ним обращаются. См. <<сервер>>, <<сервис>>.
Исторически слолжилось, что также как сервисы и сервера, клиенты именуются
созвучно названию протокола - например http-клиент.

\item[атака, attack]
\index{атака}
\index{attack}
 - в IT -  воздействие на некоторую вычислительную систему или сеть
или набор данных или алгоритм их обработки. Атаки используются для:
\begin{enumerate}
\item{нарушения работы данной
системы;}
\item{модификации алоритма
ее работы;}
\item{модификации или фальсификации данных;}
\item{получения данных или же модификации/фальсификации за счет
 недостатков заложенных в алгоритме обработки данных.}
\end{enumerate}

\item[DoS, Denial of Service, отказ в обслуживании ]
\index{DoS}
\index{Denial of Service}
\index{отказ в обслуживании}
 - широко распространенный вариант атак
в сети, суть которых - вывод из строя некоторого сервиса для того чтобы он,
временно, не мог обслуживать клиентов.


\item[DDoS, Distributed DoS, Distriuted Denial of Service]
\index{DDoS}
\index{Distributed DoS}
\index{Distriuted Denial of Service}
 - DoS атака реализуемая
большим количеством
компьютеров через сеть. Чаще всего DoS осуществляется либо за счет превышения пропускной
способности канала в интернет атакуемого сервера, либо за счет превышения максимального
поддерживаемого сервером количества открытых соединений.

\item[транспорт, transport]
\index{транспорт}
\index{transport}
 - почти как и <<в миру>> - средство переноса, в it-контексте
речь идёт о переносе данных. Сеть интернет задумана как иерархическая
структура, в том числе это касается и передачи данных, а именно: классическая
(как у письма) структура каждого протокола
может быть представлена как две части - заголовок (с адресом) и данные
(все что осталось); в свою очередь интерпретировать эти данные можно как
угодно - в том числе можно придумать какой нибудь дополнительный протокол
для обработки этих данных. Если данные передаются по протоколу <<А>> и для
обработки полученных данных используется протокол <<Б>>, то
получается, что один протокол как бы вложен в другой. В таких случаях
говорят что протокол A является транспортным для <<Б>> или <<Б>> ходит
поверх <<А>>.

\item[туннелирование, tunneling]
\index{туннелирование}
\index{tunneling}
 - под этим термином понимается вложение данных одного
протокола в другой (см. пример для транспорт). Туннели применяются как в
<<мирных целях>> в повседневной практике (например для создания защищённых
сетей установлением соединений поверх шифрующего протокола), так и в целях
получения/передачи информации из сетей с ограниченным доступом\footnote{чаще
всего это делают в нарушение правил} - например, если пользователь может
легально ходить в интернет только по протоколу http, написав соответствующую
программу он сможет сконвертировать в http трафик любой другой.\footnote{детальное
описание этого процесса можно найти в интернете}

\item[программный модуль, programm module]
\index{программный модуль}
\index{programm module}
 - исполняемый элемент, чаще всего - исполняемая
программа в формате машинных инструкций нечитабельных человеком, (пример
для dos и windows - setup.exe),  однако в общем случае программный модуль
это любой исполняемый набор команд (например autoexec.bat) .

\item[закладка]
\index{закладка}
 - некая сущность, которая превносится в объект <<злоумышленником>>
 для получения информации и, возможно, реализации других возможностей не
предусмотренных нормальной работой объекта. Под закладкой может, в зависимости от контекста,
подразумеваться как электрически пассивное(например действующее как резонатор или усиливающая
антена), так и активное (передающее) устройство (например жучок).
См. также <<програмная закладка>>.


\item[программная закладка]
\index{программная закладка}
 - исполняемый модуль, созданный для <<нелегальной>>
работы на компьютере, как то: слежение за <<законным>> пользователем
компьютера, воровство идентификационной информации, использование
процессорного времени без ведома владельца, воровство файлов владельца
компьютера и прочее.

\item[ОС, Операционная Система]
\index{ОС}
\index{Операционная Система}
 - <<системный>> набор программных модулей, обеспечивающих минимум возможностей для работы с
их помощью других программ и, собственно, пользователя.
Типичные примеры: MS DOS, Free DOS, MS Windows 3.11, MS Windows 95, MS Windows 98, MS Windows XP,
MS  Windows 2000, Debian Linux, Slackware Linux, Red Hat Linux, ALT Linux, ASP Linux, Solaris, IBM DOS,
IBM OS/2, IBM AIX, Free BSD, Open BSD, Net BSD .

\item[ПО, Программное Обеспечение, софт ]
\index{ПО}
\index{Программное Обеспечение}
\index{софт}
 - синонимичные понятия, которые, надеюсь,
не требуют пояснений.

\item[C/CPP, C, CPP]
\index{C/CPP}
\index{C}
\index{CPP}
 - класс родственных языков программирования.

\item[platform, платформа]
\index{platform}
\index{платформа}
 - термин используемый как для определения сродства компьютерной
техники (по типу и организации комплектующих), так и для определения сродства
операционных систем (по набору и организации предоставляемых пользователю и
программам возможностей, производителю ПО, ОС). Например UNIX и WINDOWS
платформы. Платформы различных производителей зачастую имеют несовместимый
формат исполняемых файлов, например исполняемый файл Linux не запускается в
Windows.

\item[кроссплатформенный]
\index{кроссплатформенный}
 - используемый на нескольких платформах, ПО работающее более чем на одной аппаратной
 платформе и/или операционной системе. Кроссплатформенность может быть либо на уровне
 компиляции (из исходного кода можно собрать бинарь более чем для одной платформы или ОС),
 либо на уровне выполнения - исполняемый файл может быть запущен на разных ОС (платформах).

\item[червь, worm]
\index{червь}
\index{worm}
 - самораспространяющийся, после запуска на компьютере подключённом
к ЛАН или интернет, программный модуль, использующий для распространения сеть
(в частности - интернет) и ошибки в ПО обрабатывающем информацию из сети на
конечных точках сети (как серверах, так и рабочих станциях). В общем случае
черви - кроссплатформенное явление. Разумеется, возникает не сам по себе, а как
результат творчества программиста.

\item[LAN, Local Area Network, ЛВС, Локальная Вычислительная Сеть, локалка ]
\index{LAN}
\index{Local Area Network}
\index{ЛВС}
\index{Локальная Вычислительная Сеть}
\index{локалка}
 - сеть
из компьютеров, территориально расположенная на небольшом (географически) пространстве.
Классические примеры - сеть офиса, дома, микрорайона.

\item[WAN, Wide Area Network, Глобальная Сеть ]
\index{WAN}
\index{Wide Area Network}
\index{Глобальная Сеть}
 -  сеть
из компьютеров, территориально расположенная на большом (географически) пространстве.
Классические примеры - сеть internet, связанные внутренние сети корпораций с филиалами
в разных странах.

\item [ ARP, Address Resolution Protocol ]
\index{ARP}
\index{Address Resolution Protocol}
протокол используемый в локальных сетях ethernet для преобразования IP адресов в адреса
сетевых плат. Атаки с использованием протокола ARP на коммутирующее оборудование
используют тот факт, что таблицы коммутации имеют конечный размер и после переполнения
коммутации больше не производится - свитч начинает работать как хаб.

\item [ internet, интернет ]
\index{internet}
\index{интернет}
 - глобальная (WAN) сеть публичного доступа, объединяет
компьютеры частных лиц и организаций, используется для досуга, работы, рекламы, публикации
информации технического и гуманитарного характера, как развлекательного, так и познавательного
характера. Вряд ли есть область деятельности человека не упомянутая в интернет. Большинство
организаций имеет подключения к сети интернет в том или ином виде с разной степенью
ограничений на доступ к сети и извне сети - в зависимости от политики безопасности
данной организации. Сеть интернет зачастую используется для объединения частных сетей организаций
через так называемые туннели.

\item[RFC, Request for Comments]
\index{RFC}
\index{Request for Comments}
 - общеупотребительное название стандартов на составляющие
при помощи которых организована сеть Internet. RFC свободно доступны на соответствующих
серверах интернета и я вляются, по сути, открытыми стандартами, в поиске и исправлении ошибок
в которых может принять участие каждый желающий.

\item[вирус, virus]
\index{вирус}
\index{virus}
 - самораспространяющийся после запуска на компьютере программный
модуль, использующий для распространения модификацию других программных модулей,
в том числе поставляемых в комплекте ОС и в комплекте с ПО сторонних
производителей. В последнее время в связи с широким распространением ЛВС
используют в том числе методы распространения червей (сеть). Разумеется,
возникает не сам по себе, а как результат творчества программиста.

\item[троян, троянский конь]
\index{троян}
\index{троянский конь}
 - типичное название программной закладки.  В общем случае может
распространятся автоматически - и как вирус, и как червь; классическое
применение - индивидуальная инсталляция <<вручную>>. Разумеется, возникает не
сам по себе, а как результат творчества программиста.

\item[spam, спам]
\index{spam}
\index{спам}
 - незапрошенная получателем email корреспонденция рекламного характера.

\item[spamer, спамер]
\index{spamer}
\index{спамер}
 - человек рассылающий СПАМ.

\item[спамерский]
\index{спамерский}
 - имеющий отношение к рассылкам СПАМа.

\item[адрес, address]
\index{address}
\index{адрес}
 - значение слова практически такое же, как и <<в миру>>, однако в
ИТ-контексте необходимо понимать следующее: в любой широко используемой
компьютерной сети как глобального, так и локального масштаба используются
<<адреса>>, отличие от обычного смысла - адрес указывает не физическое
положение компьютера, а его место в логике сети (её логической структуре).
Адреса в IT области ориентированы не на людей, а на работу программ.
Также, необходимо понимать, что поскольку компьютер может участвовать в
различных сетях - адресов у него может быть несколько, кроме того они
могут со временем меняться (типичный пример - дозвон поставщику услуг интернет
- чаще всего при повторном дозвоне провайдер назначит уже другой адрес).
Очевидно, что знание адресов требуется для практически любого обмена
информацией.

\item[Mbit]
\index{Mbit}
\index{Мегабит}
 - Мегабит, сокращение используемое для описания пропускной
способности. Для сравнения типичное для Москвы пользовательское соединение
с интернет типа <<ADSL>> предоставляет максимальную скорость приёма до 7Mbit,
но в то же время максимальная скорость отправки всего лишь в 0.7 Mbit - именно
поэтому ADSL это асимметричное (или же асинхронное) соединение.

\item[DNS]
\index{DNS}
\index{domain name system}
\index{трансляция имен в адреса}
 - domain name system - система трансляции из символьных имен вида
www.pornoserver.com в адрес вида 121.123.23.54 и обратно. <<Держится>> на взаимодействии
пользователей(запускаемых ими программ) с dns серверами и взаимодействии dns серверов между
собой.

\item[adress resolving, adress resolution, ресолвинг, преобразование адресов, трансляция адресов ]
\index{adress resolving}
\index{adress resolution}
\index{ресолвинг}
\index{преобразование адресов}
\index{трансляция адресов}
- процедура перевода из машинного представления адреса в легко понимаемый человеком. Не для всех адресов
в сети зарегистрировано удобное для человека представление. Процесс resolving'а подразумевает
обращение к серверу DNS.

\item[traffic, трафик]
\index{traffic}
\index{трафик}
 - объем передаваемых данных. Основной доход от деятельности провайдеры
доступа в интернет получают за счет взимания платы за объем переданных
данных через своё оборудование для подписчиков услуг (то есть оплачивается как
трафик клиентов, так и трафик серверов) - расходы обсчитываются на каждый адрес
(из списка адресов принадлежащих сети провайдера) используемый при работе
в сети интернет.\\
 - поток данных, чаще всего имеется ввиду объем данных в единицу
времени (например загрузка канала), причем трафик именуется зачастую по
названию протокола согласно которому\footnote{исторически принято говорить
<<по которому>>} передаются данные, например http-трафик.

\item[функция]
\index{function}
\index{функция}
\index{процедура}
 - некая часть программы, которую можно использовать (вызывать)
многократно. Функции зачастую называют <<процедурами>>. Современные ОС реализуют
для программ возможность использовать как свои собственные функции, так и
функции находящиеся внутри программ ОС и других программ.

\item[машинное представление]
\index{машинное представление}
 - способ хранения чисел в ячейках памяти компьютера.

\item[железо, harware]
\index{железо}
\index{harware}
 - любая аппаратная часть компьютера либо компьютер <<отдельно
от программ>>.

\item[ядро,kernel]
\index{ядро}
\index{kernel}
 - в контексте этой статьи - часть кода ОС, исполняемая с максимумом
возможных прав, основная часть ОС, обеспечивающая её функциональность по работе
с железом, выделению временных и прочих ресурсов программам, а также разделение
прав пользователей.

\item[низкий, низкоуровневый]
\index{низкий}
\index{низкоуровневый}
 - в контексте этой статьи, чаще всего, - максимально  приближенный к внутренним
 особенностям работы ОС или компьютера, например  низкоуровневое программирование -
 либо программирование очень близкое к работе с ядром ОС, либо программирование в
 непосредственно инструкциях процессора (то есть фактически <<в коде>> (циферками
 то есть), а не командами, которые затем переводятся в код (цифры в том или ином
 машинном представлении)).

\item[rootkit, руткит, RootKit]
\index{rootkit}
\index{руткит}
\index{RootKit}
 - <<самый продвинутый>> вариант реализации программных закладок.
Для того чтобы удовлетворять классу rootkit закладка должна реализовывать
невидимость своего присутствия как для средств обнаружения программ включённых
в комплект ОС, так и для утилит сторонних производителей. Такой уровень
невидимости достигается путём перехвата внутренних функций ОС на уровне ядра (самый
низкий, или <<самый внутренний>> уровень работы ОС - функций к которым
обращаются как процессы прикладного режима, так и функции ОС).

\item[машинные коды(машинные инструкции)]
\index{машинные коды}
\index{машинные инструкции}
 - исполняемые команды процессора, самый низкий уровень
инструкций компьютеру.

\item[компилятор, compiler]
\index{компилятор}
\index{compiler}
 - программа создающая машинные инструкции из инструкций
языка программирования с записью их в результирующий файл, который затем может самостоятельно
быть  использован для запуска.

\item[интерпретатор (interpretator)]
\index{интерпретатор}
\index{interpretator}
 - программа создающая машинные инструкции из инструкций
языка программирования с последующим их исполнением в процессе своей работы по интерпретации,
то есть инструкции интерпретатора не могут быть использованы для самостоятельного запуска в отсутствии
интерпретатора.

\item[бинарь, binary, application, приложение, программа]
\index{бинарь}
\index{binary}
\index{application}
\index{приложение}
\index{программа}
 - исполняемый модуль программы в виде инструкций процессора на данной
 вычислительной
системе. Не путать со скриптами и исходными текстами, которые тоже
есть набор инструкций,
 но не для
процессора, а для некоей программы (компилятора или интерпретатора).
Синонимичные определения: программа, приложение.

\item[исходник, исходный текст, source code, source, сырец ]
\index{исходник}
\index{исходный текст}
\index{source code}
\index{source}
\index{сырец}
 - синонимичные названия текста программы, после
обработки которого компилятором получается бинарник.

\item[library, библиотека, <<либа>>]
\index{library}
\index{библиотека}
 - набор функций в программе или ОС, которые
подразумевают
возможность независимого использования . Библиотеки бывают статические (тело библиотеки
вставляется в тело программы при компиляции, не требуется наличие библиотеки в ОС) и
 динамические (библиотека загружается из комплекта ОС).

\item[payload]
\index{payload}
 - нагрузка, ради которой работает rootkit (или закладка вообще). См. \ref{payload_term}

\item[сцена]
\index{сцена}
В контексте этой статьи - понятие объединяющее людей интересующихся определенной информацией
и создающих некий контент интересный, в основном, в рамках этого круга. Существует демо-сцена,
vx-сцена и другие типа сцены. В рамках таких объединений люди творят ради искусства,
самореализации и обмена идеями -  <<just for fun>> .

\item[vx-сцена,вирусная сцена]
\index{vx-сцена}
\index{вирусная сцена}
Сцена вирмейкеров. Людей, которые пишут вирусы. Не для того (по крайней мере не обязательно) чтобы навредить другим, а <<just for fun>>.

\item[29A]
\index{29A}
 - журнал команды вирмейкеров. Широко известен в рамках вирусной сцены, см. сноску в \ref{29A_mag}

\item[алгоритм]
\index{алгоритм}
 - формализованная последовательность действий или событий поддающаяся описанию словестно и графически.

\item[реверсинг, reversing, reverse engenearing\label{reversing}]
\index{реверсинг}
\index{reversing}
\index{reverse engenearing}
 - процесс восстановления
 алгоритма программы по её исполняемому
файлу. Занятие трудоемкое и длительное, кроме того требующее высокой квалификации и знания
низкоуровнего программирования (ассемблера и машинных кодов).

\item[отладчик, debugger]
\index{отладчик}
\index{debugger}
 - программ для пошагового выполнения некоего кода, не обязательно машинного (в т.ч. бывают отладчики
для скриптовых языков программирования) с возможностью просмотра значений переменных используемых в программе - используется для исправления ошибок (отладки) программ.\\
 - программа для трассировки. Позволяет пошагово выполнять машинные инструкции трассируемого
 приложения с просмотром значений в регистрах процессора, памяти.

\item[трассировка, отладка, debugging]
\index{трассировка}
\index{отладка}
\index{debugging}
 - процесс пошагового прохождения работы программы с использованием
специальной программы отладчика(debugger'а), которая в свою очередь использует для перевода исполняемого
файла в пошаговый режим специальные возможности ОС и процессора. Существуют методики определения программой того, что её трассируют, равно как и методы сокрытия этого факта от программы. Вообще говоря трассировка и отладка - не одно и тоже, но часто эти понятия используются именно в указанном контексте.
Точные значения этих понятий таковы:\\
Отладка - этап разработки компьютерной программы, на котором обнаруживают, локализуют и устраняют ошибки.\\
Трассировка - пошаговое выполнение программы с остановками на каждой команде (assembler) или строке (языки более высокого уровня).

\item[реверсер (reverser)\label{reverser}]
\index{реверсер}
\index{reverser}
 - человек занимающийся процессом reverse engenearing.

\item[jpeg, джипег]
\index{jpeg}
 - название формата хранения картинок.
 - формат файла, который содержит сжатые данные обычно также называют именем JPEG, наиболее
 распространённые расширения для таких файлов .jpeg, .jfif, .jpg . формат jpg использует сжатие с
 потерями, при сохранении JPEG-файла можно указать степень качества, а значит и степень сжатия, которую
 обычно задают в некоторых условных единицах, например, от 1 до 100 или от 1 до 10. Большее число
 соответствует лучшему качеству, но при этом увеличивается размер файла. Обыкновенно, разница в
 качестве между 90 и 100 на глаз уже практически не воспринимается. Следует помнить, что побитно
 восстановленое изображение всегда отличается от оригинала.

\item[гиф, gif]
\index{gif}
 - название формата хранения картинок.
- формат GIF способен хранить сжатые данные без потери качества в формате до 256 цветов. Независящий от аппаратного обеспечения формат GIF был разработан в 1987 году (GIF87a) фирмой CompuServe для передачи растровых изображений по сетям. В 1989-м формат был модифицирован (GIF89a), были добавлены поддержка прозрачности и анимации. GIF использует LZW-компрессию, что позволяет неплохо сжимать файлы, в которых много однородных заливок (логотипы, надписи, схемы).

\item[ресурс]
\index{ресурс}
 - некий сервис в сети доступный для использования всем или некоторым клиентам.

\item[web]
\index{web}
 - альтернативное название http протокола.

\item[web-ресурс]
\index{web-ресурс}
 - http-ресурс, т.е. ресурс доступный по http протоколу, см. url, browser.

\item[web designer, веб-дизайнер ]
\index{web designer}
\index{веб-дизайнер}
 - человек занимающийся созданием web-сайтов (страниц в интернете).

\item[ браузер, browser ]
\index{браузер}
\index{browser}
 - http клиент. смотрелка страниц в интернете.

\item[ exploit, эксплойт, сплойт ]
\index{exploit}
\index{эксплойт}
\index{сплойт}
 - программный код для использования уязвимости.

\item[dropper, дроппер]
\index{дроппер}
\index{dropper}
чаще всего загрузка и запуск закладки осуществляется небольшим программным модулем, который включается в себя функционал эксплойта, downloader'а (загрузчика), executor (выполнение скачанного основного модуля).

\item[IE, Internet Explorer, ИЕ]
\index{IE}
\index{Internet Explorer}
 - браузер используемый в windows системах <<из коробки>>, то есть, если пользователь не
 инсталлировал другой бродилки по сети IE окажется бродилкой <<по умолчанию>>. Де факто
 самый используемый в сети браузер. Де факто один из самых <<дырявых>> браузеров: под
IE и расширения к нему пишется большинство exploits. Однако, это объясняется в
первую очередь его популярностью - в других браузерах тоже находят уязвимости.

\item[ урл, url, линк, ссылка, сайт, web-ресурс ]
\index{урл}
\index{url}
\index{линк}
\index{ссылка}
\index{сайт}
\index{web-ресурс}
 - Указатель (ссылка) на ресурс в сети, - в частности - строка, которую, например, браузер (например IE, Netscape, Mozilla, Opera) может использовать как адрес.

\item[баннер]
\index{баннер}
 - небольшая картинка рекламного или информационного характера. Баннеры используются,
для рекламы, в том числе контекстной, а так же для привлечения внимания и как легко заметные ссылочные элементы (в таком случае к баннеру <<привязывается>> web-дизайнером некий ссылочный url).

\item[баннерная сеть]
\index{баннерная сеть}
 - термин используется для группировки размещаемых в сети баннеров по некоторому общему признаку.
 В частности баннерные сети организуют компании предоставляющие услуги рекламного характера и услуги
 по номинированию рейтингов сайтов. За использование баннерной сетью может взиматься плата. Создание
 и использование бесплатных баннерных сетей может окупаться за счет показа рекламы в теле баннер
(изображение обновляется с серверов баннерной сети). Баннерные сети работают поверх протокола http
 и его расширений.

\item[ нулевое кольцо, ring3, ring0 ]
\index{нулевое кольцо}
\index{ring3}
\index{ring0}
 - термины характерные для описания программ работающих на intel PC - совместимых процессорах (x86,
 начиная с 286). Указывают на уровень привилегий которые использует в работе программа. ring3 - самый
 бесправный код, ring0 - самый привилегированный. В ring0 обычно работает ядро ОС, в ring3 -
 пользовательское ПО и часть ПО ОС не требующая привилегий ring0.

\item[ конфигурабельно, настраиваемо ]
\index{конфигурабельно}
 - подразумевается возможность настройки поведения программы.

\item [ система, системный ]
\index{система}
\index{системный}
 - в контексте этой статьи система это чаще всего ОС, системный - относящийся к встроенным в ОС функциям, базам данных, файлам.

\item[ API, Application Programming Interface, АПИ ]
\index{API}
\index{Application Programming Interface}
\index{АПИ}
 - набор функций предоставляемых ОС, в частности, для пользовательских программ. Бывают системные
 API (для использования модулями ОС) и пользовательские API (для использования прикладными программами).

\item [ реестр ]
\index{реестр}
 - в контексте данной статьи и вообще в контексте Windows совместимых систем: системная база данных,
 для использования которой каждое приложение может использовать пользовательское API. Для реестра
 windows характерно использование значений в машинном представлении (то есть значений не понятных не
 профессионалу). Реестр активно используется большинством windows приложений.

\item[драйвер, driver]
\index{драйвер}
\index{driver}
 - ПО из комплекта ОС или поставляемое сторонним производителем для поддержки работы ОС с
 некоторым устройством (железом).

\item[полиморфизм, полиморфность, мутации кода, пермутация]
\index{полиморфизм}
\index{полиморфность}
\index{мутации кода}
\index{пермутация}
 - подразумевается методика изменения бинарного кода основанная на том, что
 <<на языке процессора можно сделать одно и то же разными словами>>. Практически
 представляет собой вариацию бинарного кода с заменой инструкции и блоков
инструкций на эквивалентные по смыслу (действию).

\item[морф]
\index{морф}
 - в контексте этой статьи - очередная версия <<программы мутанта>> которая делает то же самое, но выглядит по другому.

\item[ фича, feature, возможность, наворот, фичастость ]
\index{фича}
\index{feature}
 - подразумевается наличие неких <<продвинутых>>, расширенных возможностей, по сравнению
 с чем то типичным. Может применяться как к софту, так и к железу.

\item[firewall, файервол, brandmouer, брэндмауэр]
\index{firewall}
\index{файервол}
\index{brandmouer}
\index{брэндмауэр}
 - средство контроля трафика в сети. Позволяет, в зависимости от уровня разработки
(<<фичастости>>) продукта контролировать работу с транспортными протоколами (IP,udp),
 так и работу на уровне приложений (http, ftp клиенты и прочее).

\item[proxy,прокси, проксик, прокся]
\index{proxy}
\index{прокси}
 - некоторая программа, которая работает в режиме посредника при работе с сетью.
Чаще всего software решение. Прокси классифицируются по типам протоколов.
Наиболее распространены http прокси. Классический прокси требует настройки
браузера для того чтобы его использовать.

\item[прозрачный прокси]
\index{прозрачный прокси}
 - прокси, который работает с перенаправлением трафика и таким образом не требует
 настройки в браузере. Классическое использование - на серверах через которые
 организация подключена к интернет.

\item[personal firewall, персональный файрволл]
\index{personal firewall}
\index{персональный файрволл}
 - программное обеспечение контролирующее обращения программ к ресурсам сети,
 настройка доступна администратору компьютера, в простейшем случае настраивает
 тот кто пользуется. Настройка производится через графический интерфейс, чаще
 всего по факту обращения программы в сеть или обращения к ресурсам компьютера из сети.

\item[botnet,ботнет]
\index{botnet}
\index{ботнет}
 - сеть образуемая инфицированными компьютерами. Управляется владельцем ботнет.
Чаще всего владелец ботнет является его создателем. Но при этом отнюдь не
всегда владелец ботнета автор закладок при помощи которых botnet функционирует.

\item[маршрутизатор]
\index{маршрутизатор}
 - устройства обеспечивающие функционирование сетей, в частности, интернет. Их
задача - разделение сетевого трафика в зависимости от типа и направления. В
организации есть обычно как минимум один прибор выполняющий функции маршрутизатора - тот,
 через который организация соединена с internet. Самый примитивный пример маршрутизатора - ADSL modem.
 Маршрутизаторы работают, как минимум, с транспортными протоколами.

\item[коммутатор, свитч]
\index{коммутатор}
\index{свитч}
 - устройства обеспечивающие функционирование, в основном, локальных сетей. Их задача - доставка трафика между конечными рабочими станциями. Обладают минимальным интеллектом для определения, например, того из какого порта в какой порт направить трафик.

\item[STP, Spanning Tree Protocol]
\index{STP}
\index{Spanning Tree Protocol}
- протокол применяемый в локальных сетях с топологией звезда для выявления
 непозволительных в таких
сетях закольцовываний на уровне кабельных соединений. Стандарт на этот протокол
не содержит средств проверки источника данных протокола, что позволяет атаковать
его.


\item[IDS, Intrusion Detection System, ай-ди-эс, ИДС ]
\index{IDS}
\index{Intrusion Detection System}
 - программно-аппаратный комплекс для обнаружения в сетевом трафике как следов успешных
вторжений / нарушений политики использования локальной сети, так и попыток вторжений/атак
 на устройства сети (компьютеры, маршрутизаторы, коммутаторы).

\item[IPS, Intrusion Prevention System, ай-пи-эс, ИПС ]
\index{IPS}
\index{Intrusion Prevention System}
 - практически то же, что и IDS, но с real-time реакцией на обнаржение атаки (блокировка трафика
и прочие возможные способы реакции).

\item[сигнатура, шаблон]
- в контексте этой статьи - некоторое правило, применение которого позволяет выделить тот или иной тип трафика или атаки при анализе потока данных в сети.

\item[порт]
\index{порт}
 - часть адреса в транспортных протоколах. Наглядная аналогия - номер квартиры в многоэтажном доме.

\item[концентратор, хаб]
\index{концентратор}
\index{хаб}
 - устройства обеспечивающие функционирование, в основном, локальных сетей. Их задача - доставка трафика между конечными рабочими станциями. Не обладают интеллектом для определения, например, того из какого порта в какой порт направить трафик, в результате чего трафик <<тупо>> дублируется между портами устройства. Устройства относятся к классу устаревших и постепенно снимаются с производства и в экономически развитых странах почти не используются.

\item[проксируемый]
\index{проксируемый}
 - сервис, который предоставляется через прокси.

\item[firewalling]
\index{firewalling}
 - процесс фильтрации трафика по заданным критериям.

\item[интерактивный]
\index{интерактивный}
 - в контексте статьи - подразумевается взаимодействие пользователей с сетью.

\item[serfing, net-serf, net-serfing, сёфринг сети]
\index{serfing}
\index{сёфринг}
 - просмотр информации из сети, в основном, с использованием браузера.

\item[honeypot, хонипот]
\index{honeypot}
\index{хонипот}
 - ловушка. В контексте статьи программа или компьютер, работающие в качестве приманки для
 <<злоумышленников>>, наиболее продвинутые варианты honeypot'ов делаются профессионалами,
 причем есть проекты, где машина ловушка не просто пассивно ждет, когда на неё попадет
 незадачливый взломщик, а используется для простейших действий (например серфинга сети),
 что позволяет таким ловушками при определенном стечении обстоятельств получить закладки
 инсталлируемые автоматически только на компьютеры используемые в интерактивном режиме.

\item[интерактивный хонипот]
\index{интерактивный хонипот}
 - ловушка в виде компьютера, который используется человеком (или программой эмулирующей его поведение) для серфинга сети с целью
 подцепить закладки инсталлируемые пользователям во время серфинга.

\item[ACL, access control list ]
\index{ACL}
\index{access control list}
 - обобщённое название ограничений которые могут в различных устройствах устанавливаться
 на некоторые действия совершаемые подконтрольным объектом (программой, потоком данных через сеть).

\item[adware]
\index{adware}
 - термин используемый для названия программ, которые инсталлируются с каким либо
 прикладным ПО (иногда самовольно, иногда - согласно лицензионному соглашению о
 использовании программы, которое, впрочем, мало кто читает) и занимаются сбором
 и отправкой разработчику ПО различной не связанной с <<шпионскими функциями>>
 информации, например рапортуют статистику использования программного продукта и
 прочие не имеющие отношения к личности пользователя детали.

\item[spyware]
\index{spyware}
 - термин используемый для обозначения закладок, которые инсталлируются вместе с
каким либо прикладным ПО (иногда самовольно, иногда - согласно лицензионному
соглашению о использовании программы, которое, впрочем, мало кто читает) и
занимаются сбором и отправкой разработчику ПО различной информации личного
 характера о пользователе. Например, отсылают информацию о истории его серфинга сети.
Используется, в лучшем случае, для контекстной рекламы.

\item[клиент-серверная архитектура]
\index{клиент-серверная архитектура}
 - термин применяемый для описания взаимодействия между объектами (в IT это компьютеры
 или программы), при котором, упрощённо говоря, один компьютер использует ресурсы другого.
 Чаще всего количество компьютеров(программ) серверов меньше, чем количество компьютеров
 клиентов(обычно, как  минимум,  в несколько раз).

\item [ сниффинг, sniffing ]
\index{сниффинг}
\index{sniffing}
 - прослушивание, в частности, трафика или клавиатурного ввода или любого другого доступного программно устройства.

\item [ контент ]
\index{контент}
 - наполнение. Чаще всего речь идёт о содержимом web-страницы.

\item [ хост, host ]
\index{host}
\index{хост}
 - в общем случае - компьютер подключенный к сети.

\item [ загрузка ]
\index{загрузка}
 - в контексте спама - покупка заражений через hosting-сервера. Вообще в IT, в
зависимости от контекста - либо используется для обозначения процесса запуска программы (загрузка в ОЗУ компьютера), либо для обозначения использования некоторого устройства в смысле синонимичном слову <<нагрузка>> (загрузка процессора, загрузка канала доступа к интернет).

\item [ covert channel, covered channel,hidden channel, скрытый канал ]
\index{covert channel}
\index{covered channel}
\index{скрытый канал}
\index{hidden channel}
 - в IT подразумевается канал передачи информации путем
непредусмотренного(альтернативного) использования каналов связи. В качестве примеров использования
скрытых каналов можно дать следующий, далеко не полный, список:
\begin{itemize}
\item{туннелирование полное или частичное(один и более протоколов) IPv4/IPv6 в один из его <<подпротоколов>> - ICMP, UDP и др.}
\item{туннелирование полное или частичное(один и более протоколов) IPv4/IPv6 в протоколы работающие на OSI application layer: http, dns и др.}
\end{itemize}

\item{bruteforce, полный перебор, лобовая атака}
\index{bruteforce}
\index{лобовая атака}
\index{полный перебор}
- в криптографии/криптоанализе под этим подразумевается поочередная подстановка всех возможных паролей или ключей, которые могут быть использованы для расшифрования. Чаще всего это самый неэффективный метод, однако для некоторых алгоритмов зашифрования не опубликовано иных алгоритмов атаки.

\item{fishing,фишинг} получение обманом конфиденциальных данных пользователя. Данные в дальнейшем используются для различных махинаций, например - воровство денег из электронных систем оплаты. Довольно часто используется перенаправление (например трояном) на сайт внешне похожий или даже идентичный тому, куда пользователь вводит свои данные.

\label{glossary_end}
\end{description}
