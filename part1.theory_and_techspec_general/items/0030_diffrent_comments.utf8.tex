\section{Различные комментарии}

Данная глава на данный момент - разрозненные заметки на полях.


\label{payload_term}

Термин payload я подобрал в журнале 29A\index{29A}, в котором он использовался для
обозначения функций выполняемых вирусом не относящихся к заражению и
распространению. В нашем случае этот термин употребляется как
обозначение задач rootkit выполняемых по указанию владельца в
автоматическом режиме (то  есть без специальных на то команд от
владельца). Пользуясь случаем, хочу  высказать огромное спасибо 29A и
всем участникам virus scene, кто помог мне своими опубликованными
идеями.\footnote{Журнал 29A это один из современных (на лето 2004)
журналов вирусной сцены, на момент начала написания этой статьи
с ним можно было ознакомится по url: http://www.29a.host.sk/\label{29A_mag}. На 2009й год
команда 29A это уже, к сожалению, история сцены . }
Обзор истории 29A: http://bugtraq.ru/library/underground/29a.html

\subsection{Reversing}

Reversing - восстановление алгоритма по <<бинарю>> (см. \ref{glossary}).
Очевидно, что владелец rootkit заинтересован в максимально возможной
секретности алгоритма  его работы, так как знание алгоритма помогает
противодействию экземпляров rootkit и их сетей (kitnet/botnet),
помогает понять цели установки и исполняемых им в процессе работы
действий а также может помочь в преследовании владельца ботнета
по законам страны проживания.

\subsection{Возможные варианты реализации}
Чтение различных таблиц из юзер и кернелмоды и сравнение можно попробовать обойти за счет анализа откуда
идет операция чтения. То бишь ядро грузится в определенный диапазон адресов. Драйвера тоже рядом, но они после ядра. Можно попробовать вычислить диапазон физических адресов из которого давать на чтение полную таблицу IDT например, а из всего что дальше только часть ее. То есть драйвер загруженный позже будет доступаться из памяти дальше определенной границы, если на чтение области памяти в районе IDT можно поставить хук, то там эту ситуацию
можно попробовать отработать.

